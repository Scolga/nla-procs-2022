\begin{englishtitle} % Настраивает LaTeX на использование английского языка
% Этот титульный лист верстается аналогично.
\title{About Exponential Stability of Solutions to
Systems of Differential Equations of Neutral Type with Distributed
Delay\thanks{Работа выполнена в рамках государственного задания Института математики им. С.Л. Соболева СО РАН (проект \textnumero~FWNF-2022-0008).}}
% First author
\author{Timur Yskak  
}
\institute{Sobolev Institute of Mathematics SB RAS, Novosibirsk, Russia\\
  \email{istima92@mail.ru}
% etc
}
\maketitle

\begin{abstract}
We consider a class of systems of nonlinear non-autonomous differential equations of neutral type with distributed delay. For this class of systems, the Lyapunov~-- Krasovskii functional is proposed, with the help of which it is possible to obtain sufficient conditions for exponential stability of the zero solution, estimates of the norms of solutions that characterize the rate of decrease, and estimates of the attraction set.

\keywords{exponential stability, distributed delay, Lyapunov~-- Krasovskii functional, nonlinear differential equations, estimates of solutions} % в конце списка точка не ставится
\end{abstract}
\end{englishtitle}

\iffalse
%%%%%%%%%%%%%%%%%%%%%%%%%%%%%%%%%%%%%%%%%%%%%%%%%%%%%%%%%%%%%%%%%%%%%%%%
%
%  This is the template file for the 6th International conference
%  NONLINEAR ANALYSIS AND EXTREMAL PROBLEMS
%  June 25-30, 2018
%  Irkutsk, Russia
%
%%%%%%%%%%%%%%%%%%%%%%%%%%%%%%%%%%%%%%%%%%%%%%%%%%%%%%%%%%%%%%%%%%%%%%%%

%  Верстка статьи осуществляется на основе стандартного класса llncs
%  (Lecture Notes in Computer Sciences), который корректируется стилевым
%  файлом конференции.
%
%  Скомпилировать файл в PDF можно двумя способами:
%  1. Использовать pdfLaTeX (pdflatex), (LaTeX+DVIPS не работает);
%  2. Использовать LuaLaTeX (XeLaTeX будет работать тоже).
%  При использовании LuaLaTeX потребуются TTF- или OTF-шрифты CMU
%  (Computer Modern Unicode). Шрифты устанавливаются либо пакетом
%  дистрибутива LaTeX cm-unicode
%              (https://www.ctan.org/tex-archive/fonts/cm-unicode),
%  либо загрузкой и установкой в операционной системе, адрес страницы:
%              http://canopus.iacp.dvo.ru/%7Epanov/cm-unicode/
%  Второй вариант не будет работать в XeLaTeX.
%
%  В MiKTeX (дистрибутив LaTeX для ОС Windows):
%  1. Пакет cm-unicode устанавливается вручную в программе MiKTeX Console.
%  2. Для верстки данного примера, а именно, картинки-заглушки необходимо,
%     также вручную, загрузить пакет pgf. Этот пакет используется популярным
%     пакетом tikz.
%  3. Тест показал, что остальные пакеты MiKTeX грузит автоматически (если
%     ему разрешено автоматически грузить пакеты). Режим автозагрузки
%     настраивается в разделе Settings в MiKTeX Console.
%
%
%  Самый простой способ сверстать статью - использовать pdfLaTeX, но
%  окончательный вариант верстки сборника будет собран в LuaLaTeX,
%  так как результат получится лучшего качества, благодаря пакету microtype и
%  использованию векторных шрифтов OTF вместо растровых pdfLaTeX.
%
%  В случае возникновения вопросов и проблем с версткой статьи,
%  пишите письма на электронную почту: eugeneai@irnok.net, Черкашин Евгений.
%
%  Новые варианты корректирующего стиля будут доступны на сайте:
%        https://github.com/eugeneai/nla-style
%        файл - nla.sty
%
%  Дальнейшие инструкции - в тексте данного шаблона. Он одновременно
%  является примером статьи.
%
%  Формат LaTeX2e!

\documentclass[12pt]{llncs}  % Необходимо использовать шрифт 12 пунктов.

% При использовании pdfLaTeX добавляется стандартный набор русификации babel.
% Если верстка производится в LuaLaTeX, то следующие три строки надо
% закомментировать, русификация будет произведена в корректирующем стиле автоматом.
\usepackage{iftex}

\ifPDFTeX
\usepackage[T2A]{fontenc}
\usepackage[utf8]{inputenc} % Кодировка utf-8, cp1251 и т.д.
\usepackage[english,russian]{babel}
\fi

% Для верстки в LuaLaTeX текст готовится строго в utf-8!

% В операционной системе Windows для редактирования в кодировке utf-8
% можно использовать программы notepad++ https://notepad-plus-plus.org/,
% techniccenter http://www.texniccenter.org/,
% SciTE (самая маленькая по объему программа) http://www.scintilla.org/SciTEDownload.html
% Подойдет также и встроенный в свежий дистрибутив MiKTeX редактор
% TeXworks.

% Добавляется корректирующий стилевой файл строго после babel, если он был включен.
% В параметре необходимо указать russian, что установит не совсем стандартные названия
% разделов текста, настроит переносы для русского языка как основного и т.п.

\usepackage{todonotes} % Этот пакет нужен для верстки данного шаблона, его
                       % надо убрать из вашей статьи.

\usepackage[russian]{nla}

% Многие популярные пакеты (amsXXX, graphicx и т.д.) уже импортированы в корректирующий стиль.
% Если возникнут конфликты с вашими пакетами, попробуйте их отключить и сверстать
% текст как есть.
%
%


% Было б удобно при верстке сборника, чтобы названия рисунков разных авторов не пересекались.
% Чтоб минимизировать такое пересечение, рисунки можно поместить в отдельную подпапку с
% названием - фамилией автора или названием статьи.
%
% \graphicspath{{ivanov-petrov-pics/}} % Указание папки с изображениями в форматах png, pdf.
% или
% \graphicspath{{great-problem-solving-paper-pics/}}.


\begin{document}

% Текст оформляется в соответствии с классом article, используя дополнения
% AMS.
%
\fi

\title{Об экспоненциальной устойчивости решений систем дифференциальных уравнений нейтрального типа с распределенным запаздыванием}
% Первый автор
\author{Т.~Ыскак   % \inst ставит циферку над автором.
%  \and  % разделяет авторов, в тексте выглядит как запятая.
% Второй автор
%  И.~О.~Фамилия\inst{2}
 % \and
% Третий автор
%  И.~О.~Фамилия\inst{1}
} % обязательное поле

% Аффилиации пишутся в следующей форме, соединяя каждый институт при помощи \and.
\institute{ИМ СО РАН, Новосибирск, Россия \\
  \email{istima92@mail.ru}
 % \and   % Разделяет институты и присваивает им номера по порядку.
%Институт (название в краткой форме), Город, Страна\\
 % \email{email@example.com}
% \and Другие авторы...
}

\maketitle

\begin{abstract}
В работе рассматривается класс систем нелинейных неавтономных дифференциальных уравнений с распределенным запаздыванием нейтрального типа. 
Для данного класса систем предложен функционал Ляпунова~-- Красовского, с помощью которого удается получить достаточные условия экспоненциальной устойчивости нулевого решения, оценки норм решений, которые характеризуют скорость убывания, и оценки множества притяжения.  

\keywords{экспоненциальная устойчивость, распределенное запаздывание, функционал Ляпунова~-- Красовского, нелинейные дифференциальные уравнения, оценки решений} % в конце списка точка не ставится
\end{abstract}

\section{Основные результаты} % не обязательное поле

Рассмотрим класс систем дифференциальных уравнений с распределенным запаздыванием следующего вида 
$$
\frac{d}{dt}(y(t)+D(t) y(t-\tau)) = A(t)y(t)
+\int\limits^t_{t-\tau}B(t,t-s)y(s)ds, 
$$
$$
+F\left(t,y(t),\int\limits^t_{t-\tau}y(s)ds\right),
\eqno(1)
$$
где
$D(t)$~--- матрица размера $n\times n$ с непрерывно дифференцируемыми
$T$-периодическими элементами,
$A(t)$~--- матрица размера $n\times n$ с непрерывными
$T$-периодическими элементами,
$B(t,s)$~--- матрица размера $n\times n$ с непрерывными элементами,
$T$-периодическими по первой переменной, т.~е.
$$
D(t+T)\equiv D(t),\quad A(t+T)\equiv A(t),\quad B(t+T,s)\equiv B(t,s),
$$
$F(t,u_1,u_2)$~--- вещественнозначная непрерывная вектор-функция, удовлетворяющая локальному условию Липшица по $(u_1,u_2)$  и следующей оценке:
$$
\|F(t,u_1,u_2)\| \le q_1 \|u_1\|^{1+\omega_1}+
q_2\|u_2\|^{1+\omega_2}, \quad t\ge 0,
$$
где
$$
q_1,\,q_2\ge 0,\quad \omega_1,\,\omega_2>0\, -\, const.
$$



При исследовании устойчивости нулевого решения системы (1)  используется
 функционал Ляпунова~-- Красовского, предложенный в [2]: 
$$
v(t,y) = \langle H(t)\bigl(y(t)+ D(t) y(t-\tau) \bigr),
\bigl(y(t)+D(t) y(t-\tau)\bigr) \rangle
$$
$$ +\int\limits^\tau_0 \int\limits^t_{t-\eta}
 \langle K(t-s)y(s), y(s) \rangle \, dsd\eta
 +\int\limits^t_{t-\tau}
 \langle M(t-s,s)
y(s), y(s) \rangle \, ds.
$$
Данный функционал является аналогом функционала Ляпунова~-- Красовского из [1].

Используя этот функционал, получены достаточные условия экспоненциальной устойчивости нулевого решения в терминах матричных и интегральных неравенств, указаны оценки норм решений, которые характеризуют скорость убывания, и оценки множества притяжения.  

% Рисунки и таблицы оформляются по стандарту класса article. Например,

% Современные издательства требуют использовать кавычки-елочки << >>.

% В конце текста можно выразить благодарности, если этого не было
% сделано в ссылке с заголовка статьи, например,

%

% Список литературы оформляется подобно ГОСТ-2008.
% Примеры оформления находятся по этому адресу -
%     https://narfu.ru/agtu/www.agtu.ru/fad08f5ab5ca9486942a52596ba6582elit.html
%

\begin{thebibliography}{9} % или {99}, если ссылок больше десяти.
\bibitem{DemMat_2005} Демиденко\,Г.\,В., Матвеева\,И.\,И.
	 Асимптотические свойства решений дифференциальных уравнений
	с запаздывающим аргументом~//
	Вестник НГУ. Серия: Математика, механика, информатика.
	2005. Т.~5, \textnumero~3.\ С.~20--28.
 

\bibitem{Yskak_2020} Ыскак Т.   Оценки решений одного класса систем уравнений нейтрального типа с распределенным запаздыванием~// Сиб. электрон. матем. изв. 2020. T.~17. С.~416--427.


\end{thebibliography}

% После библиографического списка в русскоязычных статьях необходимо оформить
% англоязычный заголовок.




%\end{document}

%%% Local Variables:
%%% mode: latex
%%% TeX-master: t
%%% End:
