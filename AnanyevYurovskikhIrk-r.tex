\begin{englishtitle} % Настраивает LaTeX на использование английского языка
% Этот титульный лист верстается аналогично.
\title{Estimation Problem for Discrete Systems with Information Delays\thanks{Работа выполнена в рамках исследований, проводимых в Уральском математическом центре
при финансовой поддержке Министерства науки и высшего образования Российской Федерации
(номер соглашения \textnumero~075-02-2022-874).}}
% First author
\author{Boris Ananyev  \and Polina Yurovskikh 
}
\institute{IMM UB RAS, Yekaterinburg, Russia\\
  \email{abi@imm.uran.ru, polina2104@list.ru}}
% etc

\maketitle
\begin{abstract}
We consider an estimation problem for a linear stationary discrete system under
uncertainty with measurement delay. The problem is an approximation of the
corresponding problem for a continuous system. Restrictions on disturbances are
integral, on the initial state are geometric. The problem consists in constructing
information sets based on measurement data containing the true state of the system, and
is solved using cone programming methods. Convergence in the Hausdorff metric of the
set of the discrete problem to the corresponding set of the continuous system is
proved.
\keywords{discrete system, delay, guaranteed estimation, support function, conic optimization} % в конце списка точка не ставится
\end{abstract}
\end{englishtitle}

\iffalse


\documentclass[12pt]{llncs}
\usepackage[T2A]{fontenc}
\usepackage[utf8]{inputenc}
\usepackage[english,russian]{babel}
\usepackage[russian]{nla}
\usepackage[active]{srcltx}
\begin{document}
\fi

\title{Задача оценивания дискретных систем с запаздыванием в измерении
}
\author{Б.~И.~Ананьев   \and П.~А.~Юровских 
} % обязательное поле
\institute{ИММ УрО РАН, Екатеринбург, Россия\\
  \email{abi@imm.uran.ru, polina2104@list.ru}}
\maketitle
\begin{abstract}
Рассматривается задача оценивания для линейной стационарной дискретной системы в
условиях неопределенности с запаздыванием в измерении. Данная задача представляет собой
аппроксимацию соответствующей проблемы для непрерывной системы. Ограничения на
возмущения -- интегральные, на начальное состояние -- геометрические. Рассматриваемая
задача состоит в построении информационных множеств по данным измерения, содержащих
истинное состояние системы, и она решается с помощью методов конического
программирования. Доказывается сходимость в метрике Хаусдорфа множеств дискретной
задачи к соответствующему множеству непрерывной системы. \keywords{дискретная система,
запаздывание, гарантированное оценивание, опорная функция, коническая оптимизация}
\end{abstract}

Пусть задана дискретная система с наблюдением
\begin{equation}\label{Yurovskikh-eq1}
	\begin{gathered}
		x_k=Ax_{k-1} + bv_k, \ \ x_k\in\mathbb{R}^n, \ \ k\in1:N, \\
		y_k=Gx_{k-1} + cv_k + G_1x_{k-s},\ \ y_k\in\mathbb{R}^m,
	\end{gathered}	
\end{equation}
где $s\in \mathbb{N}$ --- задержка в получении наблюдений, $A,b,G,G_1$ --- матрицы
соответствующих размерностей. Неизвестные возмущения $v_k$ и начальное состояние $x_0$
подчинены ограничениям
\begin{equation}\label{Yurovskikh-eq2}
	\sum_{k=1}^N|v_k|^2\leq 1, \ \ x_0\in X_0,
\end{equation}
где $X_0 = \{x\in\mathbb{R}^n:|x_{(j)}-x_{(j)}^c| \leq \mu_{(j)}\}$ --- бокс,  $x_{(j)}^c
> 0 $,  $ \mu_{(j)} > 0$, $j = \{1,\ldots,n\}$.
\begin{definition}\label{Yurovskikh-Dfn2}
Семейство $\mathcal{X}^N(y)\subset\mathbb{R}^n$ назовем \emph{информационным множеством
(сокращенно ИМ)}, если оно состоит из всех векторов $x_N$, для каждого из которых
существует порождающая совместимая пара $(x_0,v_\bullet)$ при $1:N$, удовлетворяющая
ограничениям \eqref{Yurovskikh-eq2}.
\end{definition}

Из определения \ref{Yurovskikh-Dfn2} вытекает, что
$
	\mathcal{X}^N (y) = \{x_N \in \mathbb{R}^n | \min_v J^N(x_N,v_{1:N})\leq 1 - \sum_{k=1}^N |y_k|^2_C\},
$
где функционал представим в виде $J^N(x_N,v_{1:N}) = \sum_{k=1}^N |v_k|^2_{C_1} -|Gx_{k-1}|^2_C -
 |G_1x_{k-s}|^2_C - 2(cv_k)'CGx_{k-1} - 2(cv_k)'CG_1x_{k-s} - 2(G_1x_{k-s})'CGx_{k-1}  \leq p$. К виду функционала
можно прийти, используя ортогональное разложение $v_k=c'Ccv_k+C_1v_k$ в пространстве
$\mathbb{R}^q$, где $C_1=I_q-c'Cc$ --- ортогональная проекция на подпространство
ker$\,c$.

Поскольку замкнутое выпуклое множество однозначно определяется своей опорной функцией,
определим для  информационного множества системы \eqref{Yurovskikh-eq1} с ограничениями
\eqref{Yurovskikh-eq2} величину
\begin{equation}\label{Yurovskikh-eq5}
	\rho(l,\mathcal{X}^N(y)) = \max_{x_N} l'x_N, \ \ x_N \in \mathcal{X}^N(y), \ l  \in\mathbb{R}^{n}.
\end{equation}
При фиксированном $l$ значение опорной функции можно найти как решение задачи конического
программирования
\begin{equation}\label{Yurovskikh-eq7}
	\begin{gathered}
		||Q r|| \leq p, \ \
		Mr \leq w, \ \
		Hr = h,
	\end{gathered}
\end{equation}
где $r = (x_0, x_1, \dots, x_N, v_1,\dots,v_N )' \in \mathbb{R}^{n+(n+q)N}$. Суммарным
ограничениям \eqref{Yurovskikh-eq2} соответствует первое неравенство из
\eqref{Yurovskikh-eq7}, геометрическим ограничениям --- второе неравенство, а последнее
равенство описывает динамику системы  \eqref{Yurovskikh-eq1}.

В докладе приводится связь параметров непрерывной и дискретной задач и рассматриваются
примеры. Результаты работы продолжают исследования \cite{Yurovskikh-Ananyev} и используют
методы из \cite{Yurovskikh-Luenberger}.

\begin{thebibliography}{9} % или {99}, если ссылок больше десяти.
\bibitem{Yurovskikh-Ananyev} {   Ананьев~Б.И., Юровских~П.А.} О задаче оценивания с
    раздельными ограничениями на начальные состояния и возмущения~// Тр. Ин-та математики
    и механики УрО РАН. 2022. Т.~28, №~1. C.~27--39.

\bibitem{Yurovskikh-Luenberger} {  Luenberger~D.G, Ye~Y.} Conic Linear Programming. In:
    Linear and Nonlinear Prog\-ramming~// International Series in Operations Research and
    Management Science, 2016, vol 228. Springer, Cham.
    https://doi.org/10.1007/978-3-319-18842-3 6
\end{thebibliography}


%\end{document}
