\begin{englishtitle} % Настраивает LaTeX на использование английского языка
% Этот титульный лист верстается аналогично.
\title{Spectral Analysis of the Stability of Fluid Flow in an Annular Channel\thanks{Работа выполнена при поддержке РНФ, проект \textnumero~22-21-00915.}}
% First author
\author{A.~A.~Mukhutdinova\inst{1}
  \and
  A.~D.~Nizamova\inst{1}
	\and
  V.~N.~Kireev\inst{2}
	\and
  S.~F.~Urmancheev\inst{1}
 }
\institute{Mavlyutov Institute of Mechanics, Ufa Investigation Center, RAS, Russia\\
  \email{adeshka@yandex.ru}
  \and
Bashkir state university, Ufa, Russia\\
\email{kireev@anrb.ru}}
% etc

\maketitle

\begin{abstract}
The work is a numerical study of the influence of the parameters of the dependence of viscosity on temperature on the spectral characteristics of the stability of a fluid flow in an annular channel with a non-uniform temperature field.

\keywords{spectrum of eigenvalues, eigenfunctions,flow stability, thermoviscous liquid, annular channel} % в конце списка точка не ставится
\end{abstract}
\end{englishtitle}

\iffalse

%%%%%%%%%%%%%%%%%%%%%%%%%%%%%%%%%%%%%%%%%%%%%%%%%%%%%%%%%%%%%%%%%%%%%%%%
%
%  This is the template file for the 6th International conference
%  NONLINEAR ANALYSIS AND EXTREMAL PROBLEMS
%  June 25-30, 2018
%  Irkutsk, Russia
%
%%%%%%%%%%%%%%%%%%%%%%%%%%%%%%%%%%%%%%%%%%%%%%%%%%%%%%%%%%%%%%%%%%%%%%%%

%  Верстка статьи осуществляется на основе стандартного класса llncs
%  (Lecture Notes in Computer Sciences), который корректируется стилевым
%  файлом конференции.
%
%  Скомпилировать файл в PDF можно двумя способами:
%  1. Использовать pdfLaTeX (pdflatex), (LaTeX+DVIPS не работает);
%  2. Использовать LuaLaTeX (XeLaTeX будет работать тоже).
%  При использовании LuaLaTeX потребуются TTF- или OTF-шрифты CMU
%  (Computer Modern Unicode). Шрифты устанавливаются либо пакетом
%  дистрибутива LaTeX cm-unicode
%              (https://www.ctan.org/tex-archive/fonts/cm-unicode),
%  либо загрузкой и установкой в операционной системе, адрес страницы:
%              http://canopus.iacp.dvo.ru/%7Epanov/cm-unicode/
%  Второй вариант не будет работать в XeLaTeX.
%
%  В MiKTeX (дистрибутив LaTeX для ОС Windows):
%  1. Пакет cm-unicode устанавливается вручную в программе MiKTeX Console.
%  2. Для верстки данного примера, а именно, картинки-заглушки необходимо,
%     также вручную, загрузить пакет pgf. Этот пакет используется популярным
%     пакетом tikz.
%  3. Тест показал, что остальные пакеты MiKTeX грузит автоматически (если
%     ему разрешено автоматически грузить пакеты). Режим автозагрузки
%     настраивается в разделе Settings в MiKTeX Console.
%
%
%  Самый простой способ сверстать статью - использовать pdfLaTeX, но
%  окончательный вариант верстки сборника будет собран в LuaLaTeX,
%  так как результат получится лучшего качества, благодаря пакету microtype и
%  использованию векторных шрифтов OTF вместо растровых pdfLaTeX.
%
%  В случае возникновения вопросов и проблем с версткой статьи,
%  пишите письма на электронную почту: eugeneai@irnok.net, Черкашин Евгений.
%
%  Новые варианты корректирующего стиля будут доступны на сайте:
%        https://github.com/eugeneai/nla-style
%        файл - nla.sty
%
%  Дальнейшие инструкции - в тексте данного шаблона. Он одновременно
%  является примером статьи.
%
%  Формат LaTeX2e!

\documentclass[12pt]{llncs}  % Необходимо использовать шрифт 12 пунктов.

% При использовании pdfLaTeX добавляется стандартный набор русификации babel.
% Если верстка производится в LuaLaTeX, то следующие три строки надо
% закомментировать, русификация будет произведена в корректирующем стиле автоматом.
\usepackage{iftex}

\ifPDFTeX
\usepackage[T2A]{fontenc}
\usepackage[utf8]{inputenc} % Кодировка utf-8, cp1251 и т.д.
\usepackage[english,russian]{babel}
\fi

% Для верстки в LuaLaTeX текст готовится строго в utf-8!

% В операционной системе Windows для редактирования в кодировке utf-8
% можно использовать программы notepad++ https://notepad-plus-plus.org/,
% techniccenter http://www.texniccenter.org/,
% SciTE (самая маленькая по объему программа) http://www.scintilla.org/SciTEDownload.html
% Подойдет также и встроенный в свежий дистрибутив MiKTeX редактор
% TeXworks.

% Добавляется корректирующий стилевой файл строго после babel, если он был включен.
% В параметре необходимо указать russian, что установит не совсем стандартные названия
% разделов текста, настроит переносы для русского языка как основного и т.п.

\usepackage{todonotes} % Этот пакет нужен для верстки данного шаблона, его
                       % надо убрать из вашей статьи.

\usepackage[russian]{nla}

% Многие популярные пакеты (amsXXX, graphicx и т.д.) уже импортированы в корректирующий стиль.
% Если возникнут конфликты с вашими пакетами, попробуйте их отключить и сверстать
% текст как есть.
%
%


% Было б удобно при верстке сборника, чтобы названия рисунков разных авторов не пересекались.
% Чтоб минимизировать такое пересечение, рисунки можно поместить в отдельную подпапку с
% названием - фамилией автора или названием статьи.
%
% \graphicspath{{ivanov-petrov-pics/}} % Указание папки с изображениями в форматах png, pdf.
% или
% \graphicspath{{great-problem-solving-paper-pics/}}.


\begin{document}

% Текст оформляется в соответствии с классом article, используя дополнения
% AMS.
%
\fi

\title{Спектральный анализ устойчивости течения жидкости в кольцевом канале}
% Первый автор
\author{А.~А.~Мухутдинова\inst{1}  % \inst ставит циферку над автором.
  \and  % разделяет авторов, в тексте выглядит как запятая.
% Второй автор
  А.~Д.~Низамова\inst{1}
  \and
	В.~Н.~Киреев\inst{2}
  \and
	С.~Ф.~Урманчеев\inst{1}
} % обязательное поле

% Аффилиации пишутся в следующей форме, соединяя каждый институт при помощи \and.
\institute{ИМех УФИЦ РАН, Уфа, Россия\\
  \email{adeshka@yandex.ru}
  \and   % Разделяет институты и присваивает им номера по порядку.
БашГУ, Уфа, Россия\\
  \email{kireev@anrb.ru}
% \and Другие авторы...
}

\maketitle

\begin{abstract}
Работа является численным исследованием влияния параметров зависимости вязкости от температуры на спектральные характеристики устойчивости течения жидкости в кольцевом канале с неоднородным температурным полем.

\keywords{спектр собственных значений, собственные функции, устойчивость течения, термовязкая жидкость, кольцевой канал} % в конце списка точка не ставится
\end{abstract}

Экспериментальное, теоретическое и численное исследование задач, связанных с устойчивостью течения жидкости является одной из актуальных задач современной гидродинамики [1-3].
Рассмотрим течение вязкой несжимаемой жидкости под действием постоянного перепада давления в кольцевом канале.
Используя стандартное для линейной теории устойчивости представление малых возмущений гидродинамических параметров в виде бегущей волны
$$
\nu=\varphi(r)\cdot e^{ik(z-ct)},
$$
где $k>0$ – волновое число, $c =c_r + ic_i$ – комплексная скорость распространения возмущений, подставляя возмущенные величины в уравнения Навье-Стокса и проводя процедуру линеаризации, получается уравнение для осесимметричного течения в кольцевом канале
$$
\begin{array}{l}
\varphi^{IV}+\frac{2}{r}\varphi'''-\frac{3}{r^2}\left(\varphi''-\frac{1}{r}\right)-2k^2\left(\varphi''+\frac{1}{r}\varphi'\right)-\\[1em]
-ikRe\left(\nu_z^0-c\right)\left(\varphi''+\frac{1}{r}\varphi'-\left(k^2+\frac{1}{r^2}\right)\varphi\right)+\\[1em]
+\left(k^4+\frac{2k^2}{r^2}-\frac{3}{r^4}+ikRe\left(\nu_z^0{''}-\frac{\nu_z^0{'}}{r}\right)\right)\varphi_1=0.
\end{array}
$$
где $Re$ – число Рейнольдса.

Граничные условия для уравнения имеют вид
$$
\varphi(a)=\varphi(b)=\varphi'(a)=\varphi'(b)=0.
$$

Постановка задачи в настоящей работе отличается от классической постановки тем, что температуры внутреннего и внешнего цилиндров кольцевого канала предполагаются различными, что приводит к неоднородному распределению температуры внутри канала. Поскольку вязкость жидкости является тем параметром, который достаточно чувствителен к изменению температуры, то следует ожидать изменение режимов течения и, в частности, характеристик гидродинамической устойчивости.

Для изотермического случая и с учетом температурной зависимости вязкости были получены спектры собственных значений, собственные функции, возмущения скорости течения жидкости для различных значений параметра термовязкости.

В результате исследования показано, что крити-ческое число Рейнольдса уменьшается при увеличении параметра термовязкости.

\begin{thebibliography}{9} % или {99}, если ссылок больше десяти.


\bibitem{Muh_Hat}	Hattori~H., Wada~A., Yamamoto~M., Yokoo~H., Yasunaga~K., Kanda~T., Hattori~K. Experimental study of laminar-to-turbulent transition in pipe flow. Physics of Fluids.~2022. Vol.~34.~034115.

\bibitem{Muh_Kir}	Kireev~V.N., Nizamova~A.D., Urmancheev~S.F. Some features of hydrodynamic instability of a plane channel flow of a thermovis-cous fluid. Fluid Dynamics. 2019. Vol.~54, \textnumero~7. Pp.~978--982.


\bibitem{Muh_Chang}	Chang~T.Y., Chen~F., Chang~M.H. Stability of plane Poiseuille–Couette flow in a fluid layer overlying a porous layer. Journal of Fluid Mechanics. 2017. Vol.~826. Pp.~376--395.



\end{thebibliography}

% После библиографического списка в русскоязычных статьях необходимо оформить
% англоязычный заголовок.




%\end{document}

%%% Local Variables:
%%% mode: latex
%%% TeX-master: t
%%% End:
