\begin{englishtitle}
\title{On Solvability of the Cauchy Problem for one Pseudohyperbolic System}
% First author
\author{Lina Bondar\inst{1},
Sanzhar Mingnarov\inst{2}}
\institute{IM SO RAN, Novosibirsk, Russia\\
  \email{b\underline{ }lina@ngs.ru}
  \and
NSU, Novosibirsk, Russia\\
\email{s.mingnarov@g.nsu.ru}}
% etc

\maketitle

\begin{abstract}
The paper considers a system that occurs when modeling bending-torsional vibrations in a rod.
We study the solvability of the Cauchy problem for such a system in the Sobolev space.

\keywords{equations unsolvable for the highest derivative, pseudohyperbolic system, the Cauchy problem, anisotropic Sobolev space}
\end{abstract}
\end{englishtitle}


\iffalse
\documentclass[12pt]{llncs}
\usepackage[T2A]{fontenc}
\usepackage[utf8]{inputenc}
\usepackage[english,russian]{babel}
\usepackage[russian]{nla}

%\usepackage[english,russian]{nla}

% \graphicspath{{pics/}} %Set the subfolder with figures (png, pdf).

%\usepackage{showframe}
\begin{document}
%\selectlanguage{russian}
\fi

\title{О разрешимости задачи Коши для одной псевдогиперболической системы}
% Первый автор
\author{Л.~Н.~Бондарь\inst{1}
  \and
% Второй автор
  С.~Б.~Мингнаров\inst{2}
  \and
% Третий автор
 % И.~О.~Фамилия\inst{2}
} % обязательное поле
\institute{ИМ СО РАН, Новосибирск, Россия\\
  \email{b\underline{ }lina@ngs.ru}
  \and
НГУ, Новосибирск, Россия\\
\email{s.mingnarov@g.nsu.ru}}
% Другие авторы...

\maketitle

\begin{abstract}
В работе рассматривается система, возникающая при моделировании изгибно-крутильных колебаний в стержне. 
Исследуется разрешимость задачи Коши для такой системы в соболевском пространстве.

\keywords{уравнения, не разрешенные относительно старшей производной по времени, псевдогиперболическая система, задача Коши, анизотропное соболевское пространство}
\end{abstract}

В докладе рассматривается 
задача Коши для одной системы дифференциальных уравнений, не разрешенной относительно старшей производной по времени:
$$
\begin{array}{l}
\left(
\begin{array}{ll}
(I-\alpha D_x^2)& \varepsilon D^2_x \\
\varepsilon D^2_x &  (I-\beta D_x^2)
\end{array} \right)
 D_t^2U 
+ 
c^2\left(
\begin{array}{ll}
1 & -\varepsilon \\
-\varepsilon &  1
\end{array} \right)
 D_x^4U 
-\delta^2 
\left(
\begin{array}{ll}
1 & 0 \\
0 &  0
\end{array} \right)D_x^2U = F(t, x)
\end{array} 
\eqno(1)
$$
в полуплоскости $R^2_+=\{t>0,\,x\in R\}$,
где $\alpha$, $\beta> 1$, $c> 0$, $0<\varepsilon <1$, $\delta\in R$. 
Такие системы возникают при описании волновой динамики в стержне 
(см., например, [1, 2]). 


Рассматриваемая система (1) относится к классу псевдогиперболических уравнений, введенных Г.В. Демиденко в [3]. 

В вырожденном случае, когда 
$\varepsilon=0$, 
система (1)
распадается на два псевдогиперболических уравнения:
$$
(I-\alpha D_x^2)D_t^2u + c^2 D_x^4 u -\delta^2 D^2_x u =f_1(t,x),
\eqno(2)
$$
$$
(I-\beta D_x^2)D_t^2v + c^2 D_x^4 v =f_2(t,x).
$$
Уравнение (2) в литературе называется уравнением Власова (см.~[1, 2]), а также~--- уравнением Рэлея--Бишопа (см.~[4, 5]).
Теоремы о разрешимости задачи Коши для псевдогиперболических уравнений см., например,~[3, 6--8].

В работе доказывается однозначная разрешимость задачи Коши для псевдогиперболической системы (1) в анизотропных соболевских пространствах с весом.

\begin{thebibliography}{99}

\bibitem{1}                          
Власов В.З. Тонкостенные упругие стержни. Стройиздат, 1940.

\bibitem{2}                          
Герасимов~С.~И., Ерофеев~В.~И. Задачи волновой динамики элементов конструкций. Саров: ФГУП ``РФЯЦ-ВНИИЭФ'', 2014.

\bibitem{3}
Демиденко~Г.~В., Успенский~С.~В. Уравнения и системы, не разрешенные относительно старшей производной. Новосибирск: Научная книга, 1998. 

\bibitem{4}                          
Bishop R.E.D. {\it Longitudinal waves in beams}. Aeronautical Quarterly. 1952. Vol.~3, Issue 4. Pp.~280--293.

\bibitem{5}                          
Rao J.S. Advanced Theory of Vibration. John Wiley \& Sons, 1992. 

\bibitem{6}                          
Demidenko~G.~V. {\it On solvability of the Cauchy problem for pseudohyper\-bolic equations}. Sib. Adv. Math. 2001. Vol~.11, No~4. Pp.~25--40. 

\bibitem{7}                          
Fedotov I., Volevich L. R. {\it The Cauchy problem for hyperbolic equations not resolved with respect to the highest time derivative}. Russ. J. Math. 
Phys. 2006. Vol.~13, No~3. Pp.~278--292.

\bibitem{8}                          
Демиденко~Г.~В. {\it Условия разрешимости задачи Коши для псевдогиперболических уравнений}. Сиб. мат. журн. 
2015. Т.~56, \textnumero~6. С.~1289--1303. 


\end{thebibliography}





%\end{document}

%%% Local Variables:
%%% mode: latex
%%% TeX-master: t
%%% End:
