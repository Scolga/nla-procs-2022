\begin{englishtitle} % Настраивает LaTeX на использование английского языка
% Этот титульный лист верстается аналогично.
\title{On a Model of Population Dynamics  
with Several Delays \thanks{Работа выполнена в рамках государственного задания
Института математики им.~С.\,Л.~Соболева СО РАН 
(проект \textnumero~FWNF-2022-0008).
}}
% First author
\author{Maria Skvortsova%\inst{1,2}
}
\institute{Sobolev Institute of Mathematics SB RAS, Novosibirsk, Russia \\
%  \and
\email{sm-18-nsu@yandex.ru}}
% etc

\maketitle

\begin{abstract}
We consider a system of delay differential equations
describing the interaction of
$n$
species of microorganisms.
We study the asymptotic stability of stationary solutions to the system.
We establish estimates of solutions characterizing the stabilization rate at infinity
and found estimates of the attraction domains.
Lyapunov--Krasovskii functionals are used to obtain the results.

\keywords{model of interaction of populations, delay differential equations,
asymptotic stability, estimates of solutions, attraction domains,
Lyapunov--Krasovskii functionals} % в конце списка точка не ставится
\end{abstract}
\end{englishtitle}

\iffalse

%%%%%%%%%%%%%%%%%%%%%%%%%%%%%%%%%%%%%%%%%%%%%%%%%%%%%%%%%%%%%%%%%%%%%%%%
%
%  This is the template file for the 6th International conference
%  NONLINEAR ANALYSIS AND EXTREMAL PROBLEMS
%  June 25-30, 2018
%  Irkutsk, Russia
%
%%%%%%%%%%%%%%%%%%%%%%%%%%%%%%%%%%%%%%%%%%%%%%%%%%%%%%%%%%%%%%%%%%%%%%%%

%  Верстка статьи осуществляется на основе стандартного класса llncs
%  (Lecture Notes in Computer Sciences), который корректируется стилевым
%  файлом конференции.
%
%  Скомпилировать файл в PDF можно двумя способами:
%  1. Использовать pdfLaTeX (pdflatex), (LaTeX+DVIPS не работает);
%  2. Использовать LuaLaTeX (XeLaTeX будет работать тоже).
%  При использовании LuaLaTeX потребуются TTF- или OTF-шрифты CMU
%  (Computer Modern Unicode). Шрифты устанавливаются либо пакетом
%  дистрибутива LaTeX cm-unicode
%              (https://www.ctan.org/tex-archive/fonts/cm-unicode),
%  либо загрузкой и установкой в операционной системе, адрес страницы:
%              http://canopus.iacp.dvo.ru/%7Epanov/cm-unicode/
%  Второй вариант не будет работать в XeLaTeX.
%
%  В MiKTeX (дистрибутив LaTeX для ОС Windows):
%  1. Пакет cm-unicode устанавливается вручную в программе MiKTeX Console.
%  2. Для верстки данного примера, а именно, картинки-заглушки необходимо,
%     также вручную, загрузить пакет pgf. Этот пакет используется популярным
%     пакетом tikz.
%  3. Тест показал, что остальные пакеты MiKTeX грузит автоматически (если
%     ему разрешено автоматически грузить пакеты). Режим автозагрузки
%     настраивается в разделе Settings в MiKTeX Console.
%
%
%  Самый простой способ сверстать статью - использовать pdfLaTeX, но
%  окончательный вариант верстки сборника будет собран в LuaLaTeX,
%  так как результат получится лучшего качества, благодаря пакету microtype и
%  использованию векторных шрифтов OTF вместо растровых pdfLaTeX.
%
%  В случае возникновения вопросов и проблем с версткой статьи,
%  пишите письма на электронную почту: eugeneai@irnok.net, Черкашин Евгений.
%
%  Новые варианты корректирующего стиля будут доступны на сайте:
%        https://github.com/eugeneai/nla-style
%        файл - nla.sty
%
%  Дальнейшие инструкции - в тексте данного шаблона. Он одновременно
%  является примером статьи.
%
%  Формат LaTeX2e!

\documentclass[12pt]{llncs}  % Необходимо использовать шрифт 12 пунктов.

% При использовании pdfLaTeX добавляется стандартный набор русификации babel.
% Если верстка производится в LuaLaTeX, то следующие три строки надо
% закомментировать, русификация будет произведена в корректирующем стиле автоматом.
%\usepackage{iftex}

%\ifPDFTeX
\usepackage[T2A]{fontenc}
\usepackage[utf8]{inputenc} % Кодировка utf-8, cp1251 и т.д.
\usepackage[english,russian]{babel}
%\fi

% Для верстки в LuaLaTeX текст готовится строго в utf-8!

% В операционной системе Windows для редактирования в кодировке utf-8
% можно использовать программы notepad++ https://notepad-plus-plus.org/,
% techniccenter http://www.texniccenter.org/,
% SciTE (самая маленькая по объему программа) http://www.scintilla.org/SciTEDownload.html
% Подойдет также и встроенный в свежий дистрибутив MiKTeX редактор
% TeXworks.

% Добавляется корректирующий стилевой файл строго после babel, если он был включен.
% В параметре необходимо указать russian, что установит не совсем стандартные названия
% разделов текста, настроит переносы для русского языка как основного и т.п.

%\usepackage{todonotes} % Этот пакет нужен для верстки данного шаблона, его
                       % надо убрать из вашей статьи.

\usepackage[russian]{nla}

% Многие популярные пакеты (amsXXX, graphicx и т.д.) уже импортированы в корректирующий стиль.
% Если возникнут конфликты с вашими пакетами, попробуйте их отключить и сверстать
% текст как есть.
%
%


% Было б удобно при верстке сборника, чтобы названия рисунков разных авторов не пересекались.
% Чтоб минимизировать такое пересечение, рисунки можно поместить в отдельную подпапку с
% названием - фамилией автора или названием статьи.
%
% \graphicspath{{ivanov-petrov-pics/}} % Указание папки с изображениями в форматах png, pdf.
% или
% \graphicspath{{great-problem-solving-paper-pics/}}.


\begin{document}

% Текст оформляется в соответствии с классом article, используя дополнения
% AMS.
%
\fi
\title{Об одной модели динамики популяций \\
с несколькими запаздываниями}
% Первый автор
\author{М.~А.~Скворцова%\inst{1,2}  % \inst ставит циферку над автором.
} % обязательное поле

% Аффилиации пишутся в следующей форме, соединяя каждый институт при помощи \and.
\institute{Институт математики им. С.\,Л. Соболева СО РАН, Новосибирск, Россия \\
%  \and   % Разделяет институты и присваивает им номера по порядку.
  \email{sm-18-nsu@yandex.ru}
% \and Другие авторы...
}

\maketitle

\begin{abstract}
Рассматривается система дифференциальных уравнений с запаздывающим аргументом,
описывающая взаимодействие
$n$
видов микроорганизмов.
Проведены исследования асимптотической устойчивости стационарных решений системы.
Установлены оценки решений, характеризующие скорость стабилизации на бесконечности,
и найдены оценки на области притяжения.
При получении результатов использовались функционалы Ляпунова~-- Красовского.

\keywords{модель взаимодействия популяций, уравнения с запаздывающим аргументом,
асимптотическая устойчивость, оценки решений, области притяжения,
функционалы Ляпунова~-- Красовского} % в конце списка точка не ставится
\end{abstract}

%\section{Основные результаты} % не обязательное поле

Рассматривается система дифференциальных уравнений с запаздывающим аргументом следующего вида:
$$
\left\{
\begin{array}{l}
\displaystyle
\frac{d}{dt} S(t)=(S^0-S(t))D-\sum\limits_{i=1}^{n} p_i(S(t)) N_i(t),
\\
\\
\displaystyle
\frac{d}{dt} N_i(t)=-D_iN_i(t)+\alpha_i p_i(S(t-\tau_i)) N_i(t-\tau_i),
\quad i=1,2,\dots,n.
\end{array}
\right.
\eqno (1)
$$
Система описывает взаимодействие
$n$
видов микроорганизмов~[1], при этом
$N_i(t)$~---
численность популяции
$i$-го
вида,
$S(t)$~---
концентрация питательного вещества.
Коэффициенты системы
$S^0$, $D$, $D_i$, $\alpha_i$
и параметры запаздывания
$\tau_i$
предполагаются положительными и постоянными.
Также предполагается, что функции
$p_i(S)$
локально липшицевы, монотонно возрастающие и
$p_i(0)=0$.

В работе проведены исследования асимптотической устойчивости стационарных решений системы (1).
Установлены оценки решений, характеризующие скорость стабилизации на бесконечности,
и найдены оценки на области притяжения.
При получении результатов использовались функционалы Ляпунова~-- Красовского~[2].

% Современные издательства требуют использовать кавычки-елочки << >>.

% В конце текста можно выразить благодарности, если этого не было
% сделано в ссылке с заголовка статьи, например,
%Работа выполнена при поддержке РФФИ (РНФ, другие фонды), проект \textnumero~00-00-00000.
%

% Список литературы оформляется подобно ГОСТ-2008.
% Примеры оформления находятся по этому адресу -
%     https://narfu.ru/agtu/www.agtu.ru/fad08f5ab5ca9486942a52596ba6582elit.html
%

\begin{thebibliography}{9} % или {99}, если ссылок больше десяти.
\bibitem{Wolkowicz1997}
Wolkowicz~G.S.K., Xia~H.
Global asymptotic behavior of a chemostat model with discrete delays.
SIAM J. Appl. Math. 1997. Vol.~57, No.~4. Pp.~1019--1043.

\bibitem{Demidenko2005}
Демиденко~Г.В., Матвеева~И.И.
Асимптотические свойства решений дифференциальных уравнений с запаздывающим аргументом~//
Вестник НГУ. Серия: математика, механика, информатика. 2005. Т.~5, \textnumero~3. С.~20--28.

\end{thebibliography}

% После библиографического списка в русскоязычных статьях необходимо оформить
% англоязычный заголовок.



%\end{document}

%%% Local Variables:
%%% mode: latex
%%% TeX-master: t
%%% End:
