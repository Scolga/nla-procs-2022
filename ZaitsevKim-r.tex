\begin{englishtitle} % Настраивает LaTeX на использование английского языка
% Этот титульный лист верстается аналогично.
\title{On Matrix Eigenvalue Spectrum Assignment for High-order Linear Systems by Static Output Feedback\thanks{Работа выполнена при поддержке Минобрнауки РФ в рамках государственного задания \textnumero~075-01265-22-00, проект FEWS-2020-0010.}}
% First author
\author{Vasilii Zaitsev 
  \and
  Inna Kim 
 % \and
 % Name FamilyName3\inst{1}
}
\institute{UdSU, Izhevsk, Russia\\
  \email{verba@udm.ru, kimingeral@gmail.com}
  %\and
%Affiliation, City, Country\\
%\email{email}
}
% etc

\maketitle

\begin{abstract}
For linear time-invariant control systems defined by a linear differential equation of the $n$th order with a multidimensional state, input and output, necessary and sufficient conditions for the solvability of the problem of assigning an arbitrary matrix spectrum by means of static output feedback are obtained.

\keywords{linear control system, eigenvalue spectrum assignment,
linear static output feedback} % в конце списка точка не ставится
\end{abstract}
\end{englishtitle}

\iffalse

%%%%%%%%%%%%%%%%%%%%%%%%%%%%%%%%%%%%%%%%%%%%%%%%%%%%%%%%%%%%%%%%%%%%%%%%
%
%  This is the template file for the 6th International conference
%  NONLINEAR ANALYSIS AND EXTREMAL PROBLEMS
%  June 25-30, 2018
%  Irkutsk, Russia
%
%%%%%%%%%%%%%%%%%%%%%%%%%%%%%%%%%%%%%%%%%%%%%%%%%%%%%%%%%%%%%%%%%%%%%%%%

%  Верстка статьи осуществляется на основе стандартного класса llncs
%  (Lecture Notes in Computer Sciences), который корректируется стилевым
%  файлом конференции.
%
%  Скомпилировать файл в PDF можно двумя способами:
%  1. Использовать pdfLaTeX (pdflatex), (LaTeX+DVIPS не работает);
%  2. Использовать LuaLaTeX (XeTeX не будет работать).
%  При использовании LuaLaTeX потребуются TTF- или OTF-шрифты CMU
%  (Computer Modern Unicode). Шрифты устанавливаются либо пакетом
%  дистрибутива LaTeX cm-unicode
%              (https://www.ctan.org/tex-archive/fonts/cm-unicode),
%  либо загрузкой и установкой в операционной системе, адрес страницы:
%  http://canopus.iacp.dvo.ru/%7Epanov/cm-unicode/
%
%  В MiKTeX (дистрибутив LaTeX для ОС Windows):
%  1. Пакет cm-unicode устанавливается вручную в программе MiKTeX Console.
%  2. Для верстки данного примера, а именно, картинки-заглушки необходимо,
%     также вручную, загрузить пакет pgf. Этот пакет используется популярным
%     пакетом tikz.
%  3. Тест показал, что остальные пакеты MiKTeX грузит автоматически (если
%     ему разрешено автоматически грузить пакеты). Режим автозагрузки
%     настраивается в разделе Settings в MiKTeX Console.
%
%
%  Самый простой способ сверстать статью - использовать pdfLaTeX, но
%  окончательный вариант верстки сборника будет собран в LuaLaTeX,
%  так как результат получится лучшего качества.
%
%  В случае возникновения вопросов и проблем с версткой статьи,
%  пишите письма на электронную почту: eugeneai@irnko.net, Черкашин Евгений
%
%  Новые варианты корректирующего стиля будут доступны на сайте:
%        https://github.com/eugeneai/nla-style
%        файл - nla.sty
%
%  Дальнейшие инструкции - в тексте данного шаблона. Он одновременно
%  является примером статьи.
%
%  Формат LaTeX2e!

\documentclass[12pt]{llncs}  % Необходимо использовать шрифт 12 пунктов.

% При использовании pdfLaTeX добавляется стандартный набор русификации babel.
% Если верстка производится в LuaLaTeX, то следующие три строки надо
% закомментировать, русификация будет произведена в корректирующем стиле автоматом.
\usepackage[T2A]{fontenc}
\usepackage[cp1251]{inputenc} % Кодировка utf-8, win1251 (cp1251) не тестировалась.
\usepackage[english,russian]{babel}

% Для верстки в LuaLaTeX текст готовится строго в utf-8!

% В операционной системе Windows для редактирования в кодировки utf-8
% можно использовать программы notepad++ https://notepad-plus-plus.org/,
% techniccenter http://www.texniccenter.org/,
% SciTE (самая маленькая по объему программа) http://www.scintilla.org/SciTEDownload.html
% Подойдет также и встроенный в свежий дистрибутив MiKTeX редактор
% TeXworks.

% Добавляется корректирующий стилевой файл строго после babel, если он был включен.
% В параметре необходимо указать russian, что установит не совсем стандартные названия
% разделов текста, настроит переносы для русского языка как основного и т.п.

\usepackage{todonotes} % Удрать из вашей статьи, нужен для верстки данного шаблона.

\usepackage[russian]{nla}

% Многие популярные пакеты (amsXXX, graphicx и т.д.) уже импортированы в корректирующий стиль.
% Если возникнут конфликты с вашими пакетами, попробуйте их отключить и сверстать
% текст как есть.
%
%


% Было б удобно при верстке сборника, чтобы названия рисунков разных авторов не пересекались.
% Чтоб минимизировать такое пересечение, рисунки можно поместить в отдельную подпапку с
% названием - фамилией автора или названием статьи.
%
% \graphicspath{{ivanov-petrov-pics/}} % Указание папки с изображениями в форматах png, pdf.
% или
% \graphicspath{{great-problem-solving-paper-pics/}} % Указание папки с изображениями в форматах png, pdf.


\begin{document}

% Текст оформляется в соответствии с классом article, используя дополнения
% AMS.
%
\fi

\title{О назначении матричного спектра \\ линейных систем высших порядков \\ статической обратной связью по выходу}
% Первый автор
\author{В.~А.~Зайцев   % \inst ставит циферку над автором.
  \and  % разделяет авторов, в тексте выглядит как запятая.
% Второй автор
  И.~Г.~Ким 
 % \and
% Третий автор
  %И.~О.~Фамилия\inst{1}
} % обязательное поле

% Аффилиации пишутся в следующей форме, соединяя каждый институт при помощи \and.
\institute{УдГУ, Ижевск, Россия\\
  \email{verba@udm.ru, kimingeral@gmail.com}
%  \and   % Разделяет институты и присваивает им номера по порядку.
%Институт (название в краткой форме), Город, Страна\\
%\email{email@examle.com}
% \and Другие авторы...
}

\maketitle

\begin{abstract}
Для линейных стационарных управляемых систем высших порядков получены необходимые и достаточные условия разрешимости задачи назначения произвольного матричного спектра
посредством статической обратной связи по выходу.

\keywords{линейная система управления, управление спектром, обратная связь по выходу} % в конце списка точка не ставится
\end{abstract}

Пусть $\mathbb{K}=\mathbb{C}$ или $\mathbb{K}=\mathbb{R}$; $\mathbb{K}^n=\{x={\rm col}\,(x_1,\ldots,x_n): x_i\in\mathbb{K}\}$~--- линейное пространство вектор-столбцов над полем $\mathbb{K}$; $M_{m,n}(\mathbb{K})$~--- пространство матриц размерности $m\times n$ над полем $\mathbb{K}$; $M_n(\mathbb{K}):=M_{n,n}(\mathbb{K})$; $I\in M_n(\mathbb{K})$~--- единичная матрица.

Рассмотрим линейную управляемую систему \cite{LAIA}
\begin{gather}\label{010}
x^{(n)}+\sum\limits_{i=1}^n
A_{i}x^{(n-i)}=\sum\limits_{\alpha=1}^m\sum\limits_{l=p}^n
B_{l\alpha}u_{\alpha}^{(n-l)},\\
\label{020}
 y_{\beta}=\sum\limits_{\nu=1}^p C_{\nu \beta}x^{(\nu-1)},\quad
\beta=\overline{1,k}.
\end{gather}
Здесь $s,n,m,k\in\mathbb{N}$ --- заданные числа, $p\in\{\overline{1,n}\}$;
$x\in\mathbb{K}^s$ --- фазовый вектор,
$u_{\alpha}\in\mathbb{K}^s$~--- векторы управления,
$y_{\beta}\in\mathbb{K}^s$ --- векторы выходных сигналов, $A_i$,
$B_{l\alpha}$, $C_{\nu\beta}\in M_{s}(\mathbb{K})$, $i=\overline{1,n}$,
$l=\overline{p,n}$, $\nu=\overline{1,p}$, $\alpha=\overline{1,m}$,
$\beta=\overline{1,k}$. Построим векторы $u={\rm
col}(u_1,\ldots,u_m)\in\mathbb{K}^{ms}$, $y={\rm
col}(y_1,\ldots,y_k)\in\mathbb{K}^{ks}$. 
Будем строить управление в системе \eqref{010}, \eqref{020} по принципу линейной статической обратной связи по выходу
\begin{gather}\label{025}
u=Qy.
\end{gather}
Здесь $Q=\{Q_{\alpha\beta}\}\in M_{ms,ks}(\mathbb{K})$,
$Q_{\alpha\beta}\in M_s(\mathbb{K})$, $\alpha=\overline{1,m}$,
$\beta=\overline{1,k}$.


{\bf Определение 1.} Скажем, что для системы
\eqref{010}, \eqref{020} {\it разрешима задача назначения
произвольного матричного спектра посредством линейной статической
обратной связи по выходу} \eqref{025}, если для любых матриц
$\Gamma_i\in M_s(\mathbb{K})$, $i=\overline{1,n}$, существует
матрица обратной связи $Q\in M_{ms,ks}(\mathbb{K})$ такая, что
замкнутая система \eqref{010}, \eqref{020}, \eqref{025}
имеет вид 
\begin{equation*}\label{53-050}
x^{(n)}+\Gamma_{1}x^{(n-1)}+ \ldots  + \Gamma_{n}x  = 0.
\end{equation*}

По системе \eqref{010}, \eqref{020} построим блочные матрицы $B=\{B_{l\alpha}\}\in
M_{ns,ms}(\mathbb{K})$, $l=\overline{1,n}$, $\alpha=\overline{1,m}$,
$C=\{C_{\nu \beta}\}\in M_{ns,ks}(\mathbb{K})$, $\nu=\overline{1,n}$,
$\beta=\overline{1,k}$, где $B_{l\alpha}=0\in
M_s(\mathbb{K})$ при $l<p$ и $C_{\nu \beta}=0\in
M_s(\mathbb{K})$ при
$\nu>p$. 

Обозначим  $\mathcal{J}:=J\otimes I\in M_{ns}(\mathbb{K})$, где
$I\in M_s(\mathbb{K})$, $J\in M_n(\mathbb{K})$~--- первый единичный
косой ряд, $\otimes$ --- прямое (кронекерово) произведение матриц. 

Пусть $X$, $Y$ --- блочные
матрицы с блоками размерности $s$ такие, что число (блочных)
столбцов матрицы $X$ совпадает с числом (блочных) строк матрицы $Y$:
\begin{gather*}
\begin{aligned}
X&=\{X_{ij}\}\in M_{qs,rs}(\mathbb{K}), && X_{ij}\in M_s(\mathbb{K}), && i=\overline{1,q}, && j=\overline{1,r}; \\
Y&=\{Y_{j\nu}\}\in M_{rs,ts}(\mathbb{K}), && Y_{j \nu}\in
M_s(\mathbb{K}), && j=\overline{1,r}, && \nu=\overline{1,t}.
\end{aligned}
\end{gather*}
 Для матриц $X$ и $Y$ определим операцию блочного умножения по следующему правилу:
$$
Z=X \star Y:=\{Z_{i\nu}\}, \quad Z_{i\nu}:=\sum_{j=1}^r
X_{ij}\otimes Y_{j \nu} \quad i=\overline{1,q},  \quad
\nu=\overline{1,t}.
$$
Имеем $Z_{i\nu}\in M_{s^2}(\mathbb{K})$, $i=\overline{1,q}$,
$\nu=\overline{1,t}$, поэтому $Z:=X \star Y \in M_{qs^2,
ts^2}(\mathbb{K})$. 

Введем отображение %${\rm VECCR}_s,\,
${\rm VECRR}_s:M_{qs,rs}(\mathbb{K})\to M_{s,qrs}(\mathbb{K})$,
которое разворачивает матрицу $X=\{X_{ij}\}\in
M_{qs,rs}(\mathbb{K})$ по
блочным строкам 
в блочную строку с блоками размерности $s$:
$
%{\rm VECCR}_s\,X=[X_{11},\ldots,X_{q1},\ldots,X_{1r},\ldots,X_{qr}], \\
{\rm VECRR}_s\,X=[X_{11},\ldots,X_{1r},\ldots,X_{q1},\ldots,X_{qr}].
$
 Рассмотрим матрицы
$C^T\star B$, $C^T\star \mathcal{J}B$, $\ldots$, 
$C^T\star \mathcal{J}^{n-1}B$.
%Имеем  $C^T\in M_{ks,ns}(\mathbb{K})$, $B\in M_{ns,ms}(\mathbb{K})$,
%следовательно, $C^T\star \mathcal{J}^{i-1}B \in$
%$M_{ks^2,ms^2}(\mathbb{K})$ для всех $i=\overline{1,n}$. 
Построим матрицы ${\rm VECRR}_{s^2}(C^T\star \mathcal{J}^{i-1}B) \in
M_{s^2,kms^2}(\mathbb{K})$, $i=\overline{1,n}$, и матрицу
\begin{equation*}\label{030}
\Theta=\left[\begin{matrix}
{\rm VECRR}_{s^2}(C^T\star B)\\
{\rm VECRR}_{s^2}(C^T\star \mathcal{J}B)\\
\ldots\ldots\ldots\ldots\ldots\ldots\ldots \\
{\rm VECRR}_{s^2}(C^T\star \mathcal{J}^{n-1}B)
\end{matrix}\right]\in M_{ns^2,kms^2}(\mathbb{K}).
\end{equation*}


{\bf Теорема 1.} {\it Для системы \eqref{010},
\eqref{020} разрешима задача назначения произвольного матричного
спектра посредством линейной статической обратной связи по выходу
\eqref{025} тогда и только тогда, когда}
\begin{equation*}
 {\rm rank}\,\Theta = ns^2.
\end{equation*}







% Современные издательства требуют использовать кавычки-елочки << >>.

% В конце текста можно выразить благодарности, если этого не было
% сделано в ссылке с заголовка статьи, например,
%Работа выполнена при поддержке Минобрнауки РФ в рамках государственного задания \textnumero~075-01265-22-00, проект FEWS-2020-0010.
%

% Список литературы оформляется подобно ГОСТ-2008.
% Примеры оформления находятся по этому адресу -
%     https://narfu.ru/agtu/www.agtu.ru/fad08f5ab5ca9486942a52596ba6582elit.html
%

\begin{thebibliography}{9} % или {99}, если ссылок больше десяти.
\bibitem{LAIA} Zaitsev~V., Kim~I. Matrix eigenvalue spectrum assignment for linear control systems by
static output feedback. Linear Algebra and its Applications. 2021.
Vol.~613. Pp.~115--150. 

\end{thebibliography}

% После библиографического списка в русскоязычных статьях необходимо оформить
% англоязычный заголовок.




%\end{document}

%%% Local Variables:
%%% mode: latex
%%% TeX-master: t
%%% End:
