\begin{englishtitle} % Настраивает LaTeX на использование английского языка
% Этот титульный лист верстается аналогично.
\title{Grid Algorithm for Computing Reachability Sets with a Modified Reduction Procedure\thanks{Исследование выполнено за счет средств гранта Российского научного фонда \textnumero~19-11-00105.}}
% First author
\author{Igor' Izmest'ev
}
\institute{N.N. Krasovskii Institute of Mathematics and Mechanics UB RAS, Yekaterinburg, Russia\\
  \and
Chelyabinsk State University, Chelyabinsk, Russia\\
\email{j748e8@gmail.com}}
% etc

\maketitle

\begin{abstract}
The report considers the problem of constructing reachable sets of nonlinear control systems. To solve this problem, a grid algorithm is proposed in which the procedures for calculating the next reachable set and reduction are combined. The proposed algorithm allows more efficient use of computer resources.

\keywords{control, reachability set, algorithm} % в конце списка точка не ставится
\end{abstract}
\end{englishtitle}

\iffalse



%%%%%%%%%%%%%%%%%%%%%%%%%%%%%%%%%%%%%%%%%%%%%%%%%%%%%%%%%%%%%%%%%%%%%%%%
%
%  This is the template file for the 6th International conference
%  NONLINEAR ANALYSIS AND EXTREMAL PROBLEMS
%  June 25-30, 2018
%  Irkutsk, Russia
%
%%%%%%%%%%%%%%%%%%%%%%%%%%%%%%%%%%%%%%%%%%%%%%%%%%%%%%%%%%%%%%%%%%%%%%%%

%  Верстка статьи осуществляется на основе стандартного класса llncs
%  (Lecture Notes in Computer Sciences), который корректируется стилевым
%  файлом конференции.
%
%  Скомпилировать файл в PDF можно двумя способами:
%  1. Использовать pdfLaTeX (pdflatex), (LaTeX+DVIPS не работает);
%  2. Использовать LuaLaTeX (XeLaTeX будет работать тоже).
%  При использовании LuaLaTeX потребуются TTF- или OTF-шрифты CMU
%  (Computer Modern Unicode). Шрифты устанавливаются либо пакетом
%  дистрибутива LaTeX cm-unicode
%              (https://www.ctan.org/tex-archive/fonts/cm-unicode),
%  либо загрузкой и установкой в операционной системе, адрес страницы:
%              http://canopus.iacp.dvo.ru/%7Epanov/cm-unicode/
%  Второй вариант не будет работать в XeLaTeX.
%
%  В MiKTeX (дистрибутив LaTeX для ОС Windows):
%  1. Пакет cm-unicode устанавливается вручную в программе MiKTeX Console.
%  2. Для верстки данного примера, а именно, картинки-заглушки необходимо,
%     также вручную, загрузить пакет pgf. Этот пакет используется популярным
%     пакетом tikz.
%  3. Тест показал, что остальные пакеты MiKTeX грузит автоматически (если
%     ему разрешено автоматически грузить пакеты). Режим автозагрузки
%     настраивается в разделе Settings в MiKTeX Console.
%
%
%  Самый простой способ сверстать статью - использовать pdfLaTeX, но
%  окончательный вариант верстки сборника будет собран в LuaLaTeX,
%  так как результат получится лучшего качества, благодаря пакету microtype и
%  использованию векторных шрифтов OTF вместо растровых pdfLaTeX.
%
%  В случае возникновения вопросов и проблем с версткой статьи,
%  пишите письма на электронную почту: eugeneai@irnok.net, Черкашин Евгений.
%
%  Новые варианты корректирующего стиля будут доступны на сайте:
%        https://github.com/eugeneai/nla-style
%        файл - nla.sty
%
%  Дальнейшие инструкции - в тексте данного шаблона. Он одновременно
%  является примером статьи.
%
%  Формат LaTeX2e!

\documentclass[12pt]{llncs}  % Необходимо использовать шрифт 12 пунктов.

% При использовании pdfLaTeX добавляется стандартный набор русификации babel.
% Если верстка производится в LuaLaTeX, то следующие три строки надо
% закомментировать, русификация будет произведена в корректирующем стиле автоматом.
\usepackage{iftex}

\ifPDFTeX
\usepackage[T2A]{fontenc}
\usepackage[utf8]{inputenc} % Кодировка utf-8, cp1251 и т.д.
\usepackage[english,russian]{babel}
\fi

% Для верстки в LuaLaTeX текст готовится строго в utf-8!

% В операционной системе Windows для редактирования в кодировке utf-8
% можно использовать программы notepad++ https://notepad-plus-plus.org/,
% techniccenter http://www.texniccenter.org/,
% SciTE (самая маленькая по объему программа) http://www.scintilla.org/SciTEDownload.html
% Подойдет также и встроенный в свежий дистрибутив MiKTeX редактор
% TeXworks.

% Добавляется корректирующий стилевой файл строго после babel, если он был включен.
% В параметре необходимо указать russian, что установит не совсем стандартные названия
% разделов текста, настроит переносы для русского языка как основного и т.п.

\usepackage{todonotes} % Этот пакет нужен для верстки данного шаблона, его
                       % надо убрать из вашей статьи.

\usepackage[russian]{nla}

% Многие популярные пакеты (amsXXX, graphicx и т.д.) уже импортированы в корректирующий стиль.
% Если возникнут конфликты с вашими пакетами, попробуйте их отключить и сверстать
% текст как есть.
%
%


% Было б удобно при верстке сборника, чтобы названия рисунков разных авторов не пересекались.
% Чтоб минимизировать такое пересечение, рисунки можно поместить в отдельную подпапку с
% названием - фамилией автора или названием статьи.
%
% \graphicspath{{ivanov-petrov-pics/}} % Указание папки с изображениями в форматах png, pdf.
% или
% \graphicspath{{great-problem-solving-paper-pics/}}.


\begin{document}

% Текст оформляется в соответствии с классом article, используя дополнения
% AMS.
%
\fi

\title{Сеточный алгоритм вычисления множеств достижимости с модифицированной процедурой прореживания}
% Первый автор
\author{И.~В.~Изместьев  % \inst ставит циферку над автором.
} % обязательное поле

% Аффилиации пишутся в следующей форме, соединяя каждый институт при помощи \and.
\institute{ИММ УрО РАН, Екатеринбург, Россия \\
  \and   % Разделяет институты и присваивает им номера по порядку.
ЧелГУ, Челябинск, Россия\\
  \email{j748e8@gmail.com}
% \and Другие авторы...
}

\maketitle

\begin{abstract}
В докладе рассматривается задача о построении множеств достижимости нелинейных управляемых систем. Для решения этой задачи предлагается сеточный алгоритм, в котором совмещены процедуры вычисления следующего по времени множества достижимости и прореживания. Предложенный алгоритм позволяет более эффективно использовать ресурсы ЭВМ.

\keywords{управление, множество достижимости, алгоритм} % в конце списка точка не ставится
\end{abstract}

%\section{Постановка задачи} 

На промежутке времени $[t_0, \vartheta]$ ($\vartheta<+\infty$) задана нелинейная управляемая система
\begin{equation}\label{Izmest'evIV:eqn1}
\frac{dx}{dt}=f(t,x,u), \quad x(t_0)=x_0.	
\end{equation}
Здесь $t$ --- время, $x \in \mathbb{R}^n$ --- фазовый вектор системы, $x_0 \in \mathbb{R}^n$ --- начальное состояние, $u$ --- вектор управления, удовлетворяющий включению $u \in P$, где $P$ --- компакт в $\mathbb{R}^p$.

Предполагаются выполненными следующие условия:

\textbf{Условие A.} Вектор-функция $f(t,x,u)$ ограничена и непрерывна на $[t_0, +\infty) \times \mathbb{R}^n \times P$, и для любой ограниченной и замкнутой области $D \subset [t_0, +\infty) \times \mathbb{R}^n$ найдётся такая константа $L = L(D) \in (0,\infty)$, что
$$
||f(t,x^{(1)},u)-f(t,x^{(2)},u)|| \leq L||x^{(1)}-x^{(2)}||, \quad (t,x^{(i)}, u) \in D \times P,\quad i =1, 2.
$$
Здесь $||f||$ --- норма вектора $f$ в $\mathbb{R}^n$.

\textbf{Условие B.} Существует такая константа $\gamma\in (0, \infty)$, что
$$
||f(t,x,u)|| \leq \gamma(1+||x||), \quad (t,x,u) \in [t_0, +\infty) \times \mathbb{R}^n \times P.
$$

\textbf{Условие C.} Множество $F(t,x) = f(t,x,P) = \{f(t,x,u): u \in P\} \subset \mathbb{R}^n$ выпукло при любых $(t,x) \in [t_0, +\infty) \times \mathbb{R}^n$.

Допустимым управлением является измеримая по Лебегу вектор-функция $u(t)$, $t \in [t_0, \vartheta]$,  со значениями в $P$. Обозначим через $X(t^*,t_{*},x_*)$ ($t_0\leq t_{*} <t^*\leq \vartheta$) множество достижимости системы (1), отвечающее моменту времени $t^*$ и начальному условию $x(t_{*})=x_*\in \mathbb{R}^n$. Рассматривается задача о построении множества $X(\vartheta,t_{0},x_0)$.
%\section{Основной результат} % не обязательное поле

Для решения данной задачи применяется модификация сеточного алгоритма с динамической сеткой \cite{Ushak, Izm1,Izm2}. Исходный алгоритм можно описать следующим образом. Дискретизируем отрезок $[t_0,\vartheta]$, выбрав разбиение $\Gamma: t_0<t_1<\ldots<t_n=\vartheta$
с постоянным шагом $t_{i+1}-t_i=\delta>0$, $i=\overline{0,n-1}$.  Тогда для каждого $t_i$ множество достижимости из точки $x_0$ может быть найдено по рекуррентной формуле $X(t_i)=X(t_i,t_{i-1},X(t_{i-1}))$. Поскольку множество $X(t^*,t_*, x_*)$ нельзя вычислить точно, то делаем это приближенно $X^{(\delta)}(t^*,t_*, x_*)=x_*+\delta F^{(\delta)}(t_*,x_*)$.
Здесь $F^{(\delta)}(t_*,x_*)=f(t_*,x_*, P^{(\delta)})$, $P^{(\delta)}$~--- заданное конечное подмножество $P$. 
Если количество точек в множестве $X^{(\delta)}(t_i)$ превысило заданное значение, то проводится процедура прореживания, суть которой состоит в следующем. Около множества $X^{(\delta)}(t_i)$ описывается оценочный $n$-мерный параллелепипед, внутри которого строится равномерная по каждой координате сетка. В множестве $X^{(\delta)}(t_i)$ остаются те точки, которые являются ближайшими к какому-то узлу сетки.  Итак, чтобы проредить множество $X^{(\delta)}(t_i)$, оно должно быть вычислено и хранится в памяти ЭВМ.

В докладе предлагается модифицированная процедура прореживания. В ней строятся оценочные параллелепипеды для множеств $X^{(\delta)}(t_{i-1})$ и $F^{(\delta)}(t_{i-1},X^{(\delta)}(t_{i-1}))$. Используя их, строятся параллелепипед, содержащий еще не построенное множество $X^{(\delta)}(t_{i})$, и соответствующая сетка. Далее,  вычисляя по очереди точки множества $X^{(\delta)}(t_{i})$, определяем ближайшие к узлам сетки.
 Таким образом, новый подход позволяет прореживать множество $X^{(\delta)}(t_i)$, не записывая его полностью в память ЭВМ. На языке программирования C++ с использованием технологии параллельных вычислений OpenMP написана программа, реализующая предложенный алгоритм. Проведены модельные расчеты.
% Рисунки и таблицы оформляются по стандарту класса article. Например,


% Современные издательства требуют использовать кавычки-елочки << >>.

% В конце текста можно выразить благодарности, если этого не было
% сделано в ссылке с заголовка статьи, например,


% Список литературы оформляется подобно ГОСТ-2008.
% Примеры оформления находятся по этому адресу -
%     https://narfu.ru/agtu/www.agtu.ru/fad08f5ab5ca9486942a52596ba6582elit.html
%

\begin{thebibliography}{9} % или {99}, если ссылок больше десяти.

\bibitem{Ushak} Ушаков~В.Н., Ухоботов~В.И., Ушаков~А.В., Паршиков~Г.В.  К решению задач о сближении управляемых систем~// Труды Математического института им. В.А. Стеклова. 2015. Т.~291, \textnumero~1. С.~276--291.

\bibitem{Izm1} Изместьев~И.В., Анфалов~Е.Д., Ушаков~А.В. Параллельная реализация одного алгоритма построения множества достижимости нелинейной управляемой системы~//  Теория управления и математическое моделирование. Матер. Всерос. конф. с междунар. участием, посв. памяти проф. Н.В. Азбелева и проф. Е.Л. Тонкова.  Ижевск, 2020. С.~177--178.

\bibitem{Izm2} Изместьев~И.В., Ушаков~В.Н. Параллельная реализация одного алгоритма решения задачи быстродействия для нелинейной управляемой системы~// Теория управления и теория обобщенных решений уравнений Гамильтона-Якоби. Матер. III Междунар. семинара, посв. 75-летию акад. А.И. Субботина. Екатеринбург, 2020. С.~171--174.

\end{thebibliography}

% После библиографического списка в русскоязычных статьях необходимо оформить
% англоязычный заголовок.




%\end{document}

%%% Local Variables:
%%% mode: latex
%%% TeX-master: t
%%% End:
