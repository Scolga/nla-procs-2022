\begin{englishtitle} % Настраивает LaTeX на использование английского языка
% Этот титульный лист верстается аналогично.
\title{Combined Algorithms Based on Bioinspired and Local Search Methods for Solving Multiextremal Optimization Problems}
% First author
\author{Pavel Sorokovikov
}
\institute{Matrosov Institute for System Dynamics and Control Theory of SB RAS, Irkutsk, Russia\\
  \email{pavel@sorokovikov.ru}
}
% etc

\maketitle

\begin{abstract}
We offer an approach to the numerical study of the problems of finding the absolute optimum of multiextremal function, based on the use of globalized nature-inspired and local descent methods. Six hybrid non-convex optimization algorithms are developed and implemented. Modifications of differential evolution, genetic search, particle swarm, harmony search, biogeography, and firefly algorithms are used as nature-inspired methods. We have performed the numerical study of the properties of the proposed algorithms using a representative collection of multiextremal problems. The reached research results confirm the performance of the suggested algorithms.

\keywords{non-convex optimization, multiextremal function, global optimum, bioinspired algorithms} % в конце списка точка не ставится
\end{abstract}
\end{englishtitle}

\iffalse

%%%%%%%%%%%%%%%%%%%%%%%%%%%%%%%%%%%%%%%%%%%%%%%%%%%%%%%%%%%%%%%%%%%%%%%%
%
%  This is the template file for the 6th International conference
%  NONLINEAR ANALYSIS AND EXTREMAL PROBLEMS
%  June 25-30, 2018
%  Irkutsk, Russia
%
%%%%%%%%%%%%%%%%%%%%%%%%%%%%%%%%%%%%%%%%%%%%%%%%%%%%%%%%%%%%%%%%%%%%%%%%

%  Верстка статьи осуществляется на основе стандартного класса llncs
%  (Lecture Notes in Computer Sciences), который корректируется стилевым
%  файлом конференции.
%
%  Скомпилировать файл в PDF можно двумя способами:
%  1. Использовать pdfLaTeX (pdflatex), (LaTeX+DVIPS не работает);
%  2. Использовать LuaLaTeX (XeLaTeX будет работать тоже).
%  При использовании LuaLaTeX потребуются TTF- или OTF-шрифты CMU
%  (Computer Modern Unicode). Шрифты устанавливаются либо пакетом
%  дистрибутива LaTeX cm-unicode
%              (https://www.ctan.org/tex-archive/fonts/cm-unicode),
%  либо загрузкой и установкой в операционной системе, адрес страницы:
%              http://canopus.iacp.dvo.ru/%7Epanov/cm-unicode/
%  Второй вариант не будет работать в XeLaTeX.
%
%  В MiKTeX (дистрибутив LaTeX для ОС Windows):
%  1. Пакет cm-unicode устанавливается вручную в программе MiKTeX Console.
%  2. Для верстки данного примера, а именно, картинки-заглушки необходимо,
%     также вручную, загрузить пакет pgf. Этот пакет используется популярным
%     пакетом tikz.
%  3. Тест показал, что остальные пакеты MiKTeX грузит автоматически (если
%     ему разрешено автоматически грузить пакеты). Режим автозагрузки
%     настраивается в разделе Settings в MiKTeX Console.
%
%
%  Самый простой способ сверстать статью - использовать pdfLaTeX, но
%  окончательный вариант верстки сборника будет собран в LuaLaTeX,
%  так как результат получится лучшего качества, благодаря пакету microtype и
%  использованию векторных шрифтов OTF вместо растровых pdfLaTeX.
%
%  В случае возникновения вопросов и проблем с версткой статьи,
%  пишите письма на электронную почту: eugeneai@irnok.net, Черкашин Евгений.
%
%  Новые варианты корректирующего стиля будут доступны на сайте:
%        https://github.com/eugeneai/nla-style
%        файл - nla.sty
%
%  Дальнейшие инструкции - в тексте данного шаблона. Он одновременно
%  является примером статьи.
%
%  Формат LaTeX2e!

\documentclass[12pt]{llncs}  % Необходимо использовать шрифт 12 пунктов.

% При использовании pdfLaTeX добавляется стандартный набор русификации babel.
% Если верстка производится в LuaLaTeX, то следующие три строки надо
% закомментировать, русификация будет произведена в корректирующем стиле автоматом.
\usepackage{iftex}

\ifPDFTeX
\usepackage[T2A]{fontenc}
\usepackage[utf8]{inputenc} % Кодировка utf-8, cp1251 и т.д.
\usepackage[english,russian]{babel}
\fi

% Для верстки в LuaLaTeX текст готовится строго в utf-8!

% В операционной системе Windows для редактирования в кодировке utf-8
% можно использовать программы notepad++ https://notepad-plus-plus.org/,
% techniccenter http://www.texniccenter.org/,
% SciTE (самая маленькая по объему программа) http://www.scintilla.org/SciTEDownload.html
% Подойдет также и встроенный в свежий дистрибутив MiKTeX редактор
% TeXworks.

% Добавляется корректирующий стилевой файл строго после babel, если он был включен.
% В параметре необходимо указать russian, что установит не совсем стандартные названия
% разделов текста, настроит переносы для русского языка как основного и т.п.

%\usepackage{todonotes} % Этот пакет нужен для верстки данного шаблона, его
                       % надо убрать из вашей статьи.

\usepackage[russian]{nla}

% Многие популярные пакеты (amsXXX, graphicx и т.д.) уже импортированы в корректирующий стиль.
% Если возникнут конфликты с вашими пакетами, попробуйте их отключить и сверстать
% текст как есть.
%
%


% Было б удобно при верстке сборника, чтобы названия рисунков разных авторов не пересекались.
% Чтоб минимизировать такое пересечение, рисунки можно поместить в отдельную подпапку с
% названием - фамилией автора или названием статьи.
%
% \graphicspath{{ivanov-petrov-pics/}} % Указание папки с изображениями в форматах png, pdf.
% или
% \graphicspath{{great-problem-solving-paper-pics/}}.


\begin{document}

% Текст оформляется в соответствии с классом article, используя дополнения
% AMS.
%
\fi

\title{Комбинированные алгоритмы на основе биоинспирированных методов и методов локального поиска для решения многоэкстремальных задач оптимизации}
% Первый автор
\author{П.~С.~Сороковиков
} % обязательное поле

% Аффилиации пишутся в следующей форме, соединяя каждый институт при помощи \and.
\institute{Институт динамики систем и теории управления имени В.М. Матросова СО РАН, Иркутск, Россия\\
  \email{pavel@sorokovikov.ru}
}

\maketitle

\begin{abstract}
Предложен подход к численному исследованию задач поиска абсолютного оптимума многоэкстремальной функции, основанный на применении глобализованных биоинспирированных методов и методов локального спуска. Разработаны и реализованы шесть гибридных алгоритмов невыпуклой оптимизации. Модификации алгоритмов дифференциальной эволюции, генетического поиска, роя частиц, гармонического поиска, биогеографии и роя светлячков используются в качестве методов, инспирированных природой. Выполнено численное исследование свойств предложенных алгоритмов с использованием репрезентативной коллекции многоэкстремальных задач. Полученные результаты исследований подтверждают работоспособность разработанных алгоритмов.

\keywords{невыпуклая оптимизация, многоэкстремальная функция, глобальный оптимум, биоинспирированные алгоритмы} % в конце списка точка не ставится
\end{abstract}

Задача поиска абсолютного оптимума многоэкстремального целевого функционала остается одной из самых сложных и актуальных в теории и приложениях математического программирования и оптимизации динамических систем. Как правило, надежные процедуры нелокальной оптимизации основаны на балансе между глобальным исследованием пространства поиска и локальным улучшением полученных приближений. Соответственно, одна итерация алгоритма нахождения абсолютного экстремума должна включать два этапа: глобальный поиск в допустимом множестве и локальное уточнение градиентными методами в областях, где вероятно наличие глобального оптимума. В работе предложен подход, использующий преимущества биоинспирированных алгоритмов для исследования допустимого множества и градиентных методов для локальной оптимизации, позволяющий строить вычислительные схемы, лежащие в основе эффективных методов исследования задач нелокального поиска.

Предложенный подход к численному исследованию задач поиска абсолютного экстремума мультимодальных целевых функций основан на применении алгоритмов дифференциальной эволюции \cite{Storn}, генетического поиска \cite{Whitley}, роя частиц \cite{Liu}, гармонического поиска \cite{Geem}, биогеографии \cite{Simon}, роя светлячков \cite{Yang} и L-BFGS \cite{Mokhtari}. Предложены и реализованы в виде библиотеки алгоритмов шесть двухметодных вычислительных схем решения многоэкстремальных задач оптимизации.

Разработанные алгоритмы исследовались на наборе тестовых задач и <<задач с содержательным смыслом>>, характеризующихся различным уровнем сложности. Все варианты алгоритмов, основанные на комбинациях с методом локального поиска L-BFGS, показали существенные улучшения по сравнению с исходными биоинспирированными алгоритмами. Предложенный подход позволяет строить вычислительные схемы, лежащие в основе эффективных методов исследования задач нелокального поиска. Приводятся результаты вычислительных экспериментов.


% Список литературы оформляется подобно ГОСТ-2008.
% Примеры оформления находятся по этому адресу -
%     https://narfu.ru/agtu/www.agtu.ru/fad08f5ab5ca9486942a52596ba6582elit.html
%

\begin{thebibliography}{9} % или {99}, если ссылок больше десяти.

\bibitem{Storn} Storn~R., Price~K. Differential evolution -- a simple and efficient heuristic for global optimization over continuous spaces. J.~of~Global~Optimization. 1997. Vol.~11. Pp.~341--359.

\bibitem{Whitley} Whitley~D. A genetic algorithm tutorial. Statistics~and~computing. 1994. Vol.~4. Pp.~65--85.

\bibitem{Liu} Liu~Y., Ling~X., Shi~Z., Lv~M., Fang~J., Zhang~L. A survey on particle swarm optimization algorithms for multimodal function optimization. J.~of~Software. 2011. Vol.~6. Pp.~2449--2455.

\bibitem{Geem} Geem~Z.W., Kim~J.H., Loganathan~G.V. A new heuristic optimization algorithm: harmony search. Simulation. 2001. Vol.~76. Pp.~60--68.

\bibitem{Simon} Simon~D. Biogeography-based optimization. IEEE~Transactions~on~Evolutionary Computation. 2008. Vol.~12. Pp.~702--713.

\bibitem{Yang} Yang~X.S., He~X. Firefly algorithm: recent advances and applications. International~J.~of~Swarm Intelligence. 2013. Vol.~1. Pp.~36--50.

\bibitem{Mokhtari} Mokhtari~A., Ribeiro~A. Global convergence of online limited memory BFGS. J.~of~Machine~Learning~Research. 2015. Vol.~16. Pp.~3151--3181.

\end{thebibliography}

% После библиографического списка в русскоязычных статьях необходимо оформить
% англоязычный заголовок.




%\end{document}

%%% Local Variables:
%%% mode: latex
%%% TeX-master: t
%%% End:
