\begin{englishtitle}
\title{Exact Solutions of a Nonclassical Nonlinear Partial Equation\thanks{Исследование выполнено за счет гранта Российского научного фонда № 22-21-00449.}}
% First author
\author{Anatoly Aristov}
\institute{MSU, Moscow, Russia\\
   \and
FRCCSC, Moscow, Russia\\
\email{ai\_aristov@mail.ru}}
% etc

\maketitle

\begin{abstract}
A nonclassical nonlinear partial equation is studied. 15 sets of exact solutions of it are built. Their properties are analysed.

\keywords{nonlinear partial equations, exact solutions}
\end{abstract}
\end{englishtitle}

\iffalse
\documentclass[12pt]{llncs}  % Необходимо использовать шрифт 12 пунктов.

% При использовании pdfLaTeX добавляется стандартный набор русификации babel.
% Если верстка производится в LuaLaTeX, то следующие три строки надо
% закомментировать, русификация будет произведена в корректирующем стиле автоматом.
\usepackage[T2A]{fontenc}
\usepackage[cp1251]{inputenc} % Кодировка utf-8, win1251 (cp1251) не тестировалась.
\usepackage[english,russian]{babel}

% Для верстки в LuaLaTeX текст готовится строго в utf-8!

% В операционной системе Windows для редактирования в кодировки utf-8
% можно использовать программы notepad++ https://notepad-plus-plus.org/,
% techniccenter http://www.texniccenter.org/,
% SciTE (самая маленькая по объему программа) http://www.scintilla.org/SciTEDownload.html
% Подойдет также и встроенный в свежий дистрибутив MiKTeX редактор
% TeXworks.

% Добавляется корректирующий стилевой файл строго после babel, если он был включен.
% В параметре необходимо указать russian, что установит не совсем стандартные названия
% разделов текста, настроит переносы для русского языка как основного и т.п.

\usepackage{todonotes} % Удрать из вашей статьи, нужен для верстки данного шаблона.

\usepackage[russian]{nla}
%\usepackage[english,russian]{nla}

% \graphicspath{{pics/}} %Set the subfolder with figures (png, pdf).

%\usepackage{showframe}
\begin{document}
%\selectlanguage{russian}
\fi

\title{Точные решения неклассического нелинейного уравнения в частных производных}
% Первый автор
\author{А.~И.~Аристов
  } % обязательное поле
\institute{МГУ им. М.В. Ломоносова, Москва, Россия\\
    \and
ФИЦ ИУ РАН, Москва, Россия\\
\email{ai\_aristov@mail.ru}
}




\maketitle

\begin{abstract}
В работе построено 15 классов точных решений одного нелинейного уравнения в частных производных. Проанализированы их качественные свойства.

\keywords{нелинейные уравнения в частных производных, точные решения}
\end{abstract}

%\section{Основные результаты} % не обязательное поле

Работа посвящена построению точных решений уравнения
$$
\frac{\partial}{\partial t}\Delta u+D[u]=0,\label{1}
$$
где
$$
D[u]=\alpha\frac{\partial}{\partial x}\left(\frac{\partial u}{\partial y}\frac{\partial u}{\partial z}\right)+
\beta\frac{\partial}{\partial y}\left(\frac{\partial u}{\partial x}\frac{\partial u}{\partial z}\right)+
\gamma\frac{\partial}{\partial z}\left(\frac{\partial u}{\partial x}\frac{\partial u}{\partial y}\right),
$$
$u$ --- действительнозначная функция трех координат и времени.
Предполагается, что $\alpha+\beta+\gamma=0$, причём эти коэффициенты не все равны нулю.

Нелинейные уравнения в частных производных, содержащие смешанные производные высоких порядков по времени и по пространственным
переменным, редко встречаются в литературе о точных решениях (например, [1]), тогда как имеются обширные исследования их качественных
свойств (например, [2]). Данное уравнение предложено в [1, гл. 3, \S 7]: оно может описывать спиновые волны в магнетиках при отсутствии
внешнего магнитного поля.

Здесь построено 15 классов точных решений данного уравнения. Проанализированы их качественные свойства. При этом применялись
методы обобщенного и функционального разделения переменных, поиск решений специального вида.


% Список литературы.
\begin{thebibliography}{99}
\bibitem{1}
Полянин А.Д., Зайцев В.Ф., Журов А.И. Метода решения нелинейных уравнений математической физики и механики. М.: Физматлит, 2005.
% Format for books
\bibitem{2}
Свешников А.Г., Альшин А.Б., Корпусов М.О., Плетнер Ю.Д. Линейные и нелинейные уравнения соболевского типа. М.: Физматлит, 2007.

\end{thebibliography}






%\end{document}

%%% Local Variables:
%%% mode: latex
%%% TeX-master: t
%%% End:
