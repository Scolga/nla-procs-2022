
\begin{englishtitle} % Настраивает LaTeX на использование английского языка
% Этот титульный лист верстается аналогично.
\title{Estimates for Solutions to Some Classes of Nonautonomous Nonlinear Time-Delay Systems\thanks{Работа
выполнена в рамках государственного задания
Института математики им.~С.\,Л.~Соболева СО РАН 
(проект \textnumero~FWNF-2022-0008).
}}
% First author
\author{Inessa Matveeva%\inst{1,2}
}
\institute{Sobolev Institute of Mathematics SB RAS, Novosibirsk, Russia \\
%  \and
\email{matveeva@math.nsc.ru}}
% etc

\maketitle

\begin{abstract}
Some classes of nonautonomous time-delay systems are considered,
where a time-varying delay can be unbounded.
Using Lyapunov~-- Krasovskii functionals of a special type,
estimates for solutions to these systems are established on the right half-axis.
The obtained estimates allow us to conclude whether the solutions are stable.
In the case of asymptotic stability, we estimate the attraction domains
and the stabilization rate of the solutions at infinity.

\keywords{delay differential equations, estimates for solutions,
stability, attraction domains, Lyapunov--Krasovskii functionals} % в конце списка точка не ставится
\end{abstract}
\end{englishtitle}

\iffalse
%%%%%%%%%%%%%%%%%%%%%%%%%%%%%%%%%%%%%%%%%%%%%%%%%%%%%%%%%%%%%%%%%%%%%%%%
%
%  This is the template file for the 6th International conference
%  NONLINEAR ANALYSIS AND EXTREMAL PROBLEMS
%  June 25-30, 2018
%  Irkutsk, Russia
%
%%%%%%%%%%%%%%%%%%%%%%%%%%%%%%%%%%%%%%%%%%%%%%%%%%%%%%%%%%%%%%%%%%%%%%%%

%  Верстка статьи осуществляется на основе стандартного класса llncs
%  (Lecture Notes in Computer Sciences), который корректируется стилевым
%  файлом конференции.
%
%  Скомпилировать файл в PDF можно двумя способами:
%  1. Использовать pdfLaTeX (pdflatex), (LaTeX+DVIPS не работает);
%  2. Использовать LuaLaTeX (XeLaTeX будет работать тоже).
%  При использовании LuaLaTeX потребуются TTF- или OTF-шрифты CMU
%  (Computer Modern Unicode). Шрифты устанавливаются либо пакетом
%  дистрибутива LaTeX cm-unicode
%              (https://www.ctan.org/tex-archive/fonts/cm-unicode),
%  либо загрузкой и установкой в операционной системе, адрес страницы:
%              http://canopus.iacp.dvo.ru/%7Epanov/cm-unicode/
%  Второй вариант не будет работать в XeLaTeX.
%
%  В MiKTeX (дистрибутив LaTeX для ОС Windows):
%  1. Пакет cm-unicode устанавливается вручную в программе MiKTeX Console.
%  2. Для верстки данного примера, а именно, картинки-заглушки необходимо,
%     также вручную, загрузить пакет pgf. Этот пакет используется популярным
%     пакетом tikz.
%  3. Тест показал, что остальные пакеты MiKTeX грузит автоматически (если
%     ему разрешено автоматически грузить пакеты). Режим автозагрузки
%     настраивается в разделе Settings в MiKTeX Console.
%
%
%  Самый простой способ сверстать статью - использовать pdfLaTeX, но
%  окончательный вариант верстки сборника будет собран в LuaLaTeX,
%  так как результат получится лучшего качества, благодаря пакету microtype и
%  использованию векторных шрифтов OTF вместо растровых pdfLaTeX.
%
%  В случае возникновения вопросов и проблем с версткой статьи,
%  пишите письма на электронную почту: eugeneai@irnok.net, Черкашин Евгений.
%
%  Новые варианты корректирующего стиля будут доступны на сайте:
%        https://github.com/eugeneai/nla-style
%        файл - nla.sty
%
%  Дальнейшие инструкции - в тексте данного шаблона. Он одновременно
%  является примером статьи.
%
%  Формат LaTeX2e!

\documentclass[12pt]{llncs}  % Необходимо использовать шрифт 12 пунктов.

% При использовании pdfLaTeX добавляется стандартный набор русификации babel.
% Если верстка производится в LuaLaTeX, то следующие три строки надо
% закомментировать, русификация будет произведена в корректирующем стиле автоматом.
%\usepackage{iftex}

%\ifPDFTeX
\usepackage[T2A]{fontenc}
\usepackage[utf8]{inputenc} % Кодировка utf-8, cp1251 и т.д.
\usepackage[english,russian]{babel}
%\fi

% Для верстки в LuaLaTeX текст готовится строго в utf-8!

% В операционной системе Windows для редактирования в кодировке utf-8
% можно использовать программы notepad++ https://notepad-plus-plus.org/,
% techniccenter http://www.texniccenter.org/,
% SciTE (самая маленькая по объему программа) http://www.scintilla.org/SciTEDownload.html
% Подойдет также и встроенный в свежий дистрибутив MiKTeX редактор
% TeXworks.

% Добавляется корректирующий стилевой файл строго после babel, если он был включен.
% В параметре необходимо указать russian, что установит не совсем стандартные названия
% разделов текста, настроит переносы для русского языка как основного и т.п.

%\usepackage{todonotes} % Этот пакет нужен для верстки данного шаблона, его
                       % надо убрать из вашей статьи.

\usepackage[russian]{nla}

% Многие популярные пакеты (amsXXX, graphicx и т.д.) уже импортированы в корректирующий стиль.
% Если возникнут конфликты с вашими пакетами, попробуйте их отключить и сверстать
% текст как есть.
%
%


% Было б удобно при верстке сборника, чтобы названия рисунков разных авторов не пересекались.
% Чтоб минимизировать такое пересечение, рисунки можно поместить в отдельную подпапку с
% названием - фамилией автора или названием статьи.
%
% \graphicspath{{ivanov-petrov-pics/}} % Указание папки с изображениями в форматах png, pdf.
% или
% \graphicspath{{great-problem-solving-paper-pics/}}.


\begin{document}

% Текст оформляется в соответствии с классом article, используя дополнения
% AMS.
%
\fi

\title{Оценки решений некоторых классов
неавтономных нелинейных систем с запаздыванием}
% Первый автор
\author{И.~И.~Матвеева%\inst{1,2}  % \inst ставит циферку над автором.
} % обязательное поле

% Аффилиации пишутся в следующей форме, соединяя каждый институт при помощи \and.
\institute{Институт математики им. С.\,Л. Соболева СО РАН, Новосибирск, Россия \\
%  \and   % Разделяет институты и присваивает им номера по порядку.
  \email{matveeva@math.nsc.ru}
% \and Другие авторы...
}

\maketitle

\begin{abstract}

Рассматриваются некоторые классы систем неавтономных уравнений с 
запаздыванием, при этом запаздывание может быть неограниченным. 
Используя функционалы Ляпунова~-- Красовского специального вида, установлены 
оценки решений этих систем на правой полуоси. 
Полученные оценки позволяют сделать вывод об устойчивости решений. 
В случае асимптотической устойчивости указаны оценки на области притяжения 
и на скорость стабилизации решений на бесконечности.

\keywords{дифференциальные уравнения с запаздыванием,
оценки решений, устойчивость, области притяжения,
функционалы Ляпунова~-- Красовского} % в конце списка точка не ставится
\end{abstract}

%\section{Основные результаты} % не обязательное поле

Рассматриваются некоторые классы систем неавтономных уравнений с 
запаздыванием следующего вида:
$$
\dot y(t) = A(t) y(t) + B(t) y(t-\tau(t)) + C(t) \dot y(t-\tau(t)) 
$$
$$
+ F(t, y(t), y(t-\tau(t)), \dot y(t-\tau(t))), 
\qquad  
t > 0,
\eqno(1)
$$
где 
$A(t)$, $B(t)$, $C(t)$~--- 
матрицы размера 
$n \times n$, 
запаздывание 
$\tau(t)$
может быть неограниченной функцией, 
вектор-функция 
$F(t,u_1,u_2,u_3)$
определяет нелинейные члены.
Используя функционалы Ляпунова~-- Красовского специального вида, установлены 
оценки решений систем вида (1) на полуоси
$\{t > 0\}$. 
Полученные оценки позволяют сделать вывод об устойчивости решений. 
В случае асимптотической устойчивости указаны 
оценки на области притяжения и на скорость стабилизации решений на бесконечности.
Работа продолжает наши исследования устойчивости решений 
неавтономных уравнений с запаздыванием (см., например, [1--9]).

% Современные издательства требуют использовать кавычки-елочки << >>.

% В конце текста можно выразить благодарности, если этого не было
% сделано в ссылке с заголовка статьи, например,
%Работа выполнена при поддержке РФФИ (РНФ, другие фонды), проект \textnumero~00-00-00000.
%

% Список литературы оформляется подобно ГОСТ-2008.
% Примеры оформления находятся по этому адресу -
%     https://narfu.ru/agtu/www.agtu.ru/fad08f5ab5ca9486942a52596ba6582elit.html
%

\begin{thebibliography}{9} % или {99}, если ссылок больше десяти.
\bibitem{mmmm1}
Демиденко~Г.В., Матвеева~И.И.
Устойчивость решений дифференциальных уравнений с запаздывающим аргументом и 
периодическими коэффициентами в линейных членах~//
Сибирский математический журнал. 2007. Т.~48, \textnumero~5. С.~1025--1040.

\bibitem{mmmm2}
Демиденко~Г.В., Матвеева~И.И.
Об оценках решений систем дифференциальных уравнений нейтрального типа 
с периодическими коэффициентами~//
Сибирский математический журнал. 2014. Т.~55, \textnumero~5. С.~1059--1077.

\bibitem{mmmm3}
Матвеева И.И.
Об экспоненциальной устойчивости решений периодических систем нейтрального типа~//
Сибирский математический журнал. 2017. Т.~58, \textnumero~2. С.~344--352.

\bibitem{mmmm4}
Матвеева И.И.
Об экспоненциальной устойчивости решений периодических систем нейтрального 
типа с несколькими запаздываниями~//
Дифференциальные уравнения. 2017. Т.~53, \textnumero~6. С.~730--740.

\bibitem{mmmm5}
Демиденко Г.В., Матвеева И.И., Скворцова М.А.
Оценки решений дифференциаль
ных уравнений нейтрального типа с периодическими 
коэффициентами в линейных членах~//
Сибирский математический журнал. 2019. Т.~60, \textnumero~5. С.~1063--1079.

\bibitem{mmmm6}
Матвеева И.И.
Оценки экспоненциального убывания решений линейных систем нейтрального типа 
с периодическими коэффициентами~//
Сибирский журнал индустриальной математики. 2019. Т.~22, \textnumero~3. С.~96--103.

\bibitem{mmmm7}
Matveeva I.I.
Exponential stability of solutions to nonlinear time-varying
delay systems of neutral type equations with periodic coefficients~//
Electronic Journal of Differential Equations. 2020. Vol.~2020, No.~20. Pp.~1--12.

\bibitem{mmmm8}
Матвеева И.И.
Оценки экспоненциального убывания решений одного класса 
нелинейных систем нейтрального типа с периодическими коэффициентами~//
Журнал вычислительной математики и математической физики.
2020. Т.~60, \textnumero~4. С.~612--620.

\bibitem{mmmm9}
Матвеева И.И.
Оценки решений класса неавтономных систем нейтрального типа с неограниченным запаздыванием~//
Сибирский математический журнал. 2021. Т.~62, \textnumero~3. С.~583--598.

\end{thebibliography}

% После библиографического списка в русскоязычных статьях необходимо оформить
% англоязычный заголовок.



%\end{document}

%%% Local Variables:
%%% mode: latex
%%% TeX-master: t
%%% End:
