\begin{englishtitle} % Настраивает LaTeX на использование английского языка
% Этот титульный лист верстается аналогично.
\title{Laplace Cascade Method}
% First author
\author{E.~I.~Zotova\inst{1}
  \and
  R.~D.~Murtazina\inst{2}
 }
\institute{USATU, Ufa, Russia\\
  \email{zot-kate83@yandex.ru}
  \and
USATU, Ufa, Russia\\
\email{reginaufa@yandex.ru}}
% etc

\maketitle

\begin{abstract}
The equation $z_{xy}+mxz_{x}+nyz_{y}+(2m-n+mnxy)z=0$ is integrated by the Laplace cascade method.

\keywords{invariant, cascade integration method, Laplace transforms} % в конце списка точка не ставится
\end{abstract}
\end{englishtitle}

\iffalse
%%%%%%%%%%%%%%%%%%%%%%%%%%%%%%%%%%%%%%%%%%%%%%%%%%%%%%%%%%%%%%%%%%%%%%%%
%
%  This is the template file for the 6th International conference
%  NONLINEAR ANALYSIS AND EXTREMAL PROBLEMS
%  June 25-30, 2018
%  Irkutsk, Russia
%
%%%%%%%%%%%%%%%%%%%%%%%%%%%%%%%%%%%%%%%%%%%%%%%%%%%%%%%%%%%%%%%%%%%%%%%%

%  Верстка статьи осуществляется на основе стандартного класса llncs
%  (Lecture Notes in Computer Sciences), который корректируется стилевым
%  файлом конференции.
%
%  Скомпилировать файл в PDF можно двумя способами:
%  1. Использовать pdfLaTeX (pdflatex), (LaTeX+DVIPS не работает);
%  2. Использовать LuaLaTeX (XeLaTeX будет работать тоже).
%  При использовании LuaLaTeX потребуются TTF- или OTF-шрифты CMU
%  (Computer Modern Unicode). Шрифты устанавливаются либо пакетом
%  дистрибутива LaTeX cm-unicode
%              (https://www.ctan.org/tex-archive/fonts/cm-unicode),
%  либо загрузкой и установкой в операционной системе, адрес страницы:
%              http://canopus.iacp.dvo.ru/%7Epanov/cm-unicode/
%  Второй вариант не будет работать в XeLaTeX.
%
%  В MiKTeX (дистрибутив LaTeX для ОС Windows):
%  1. Пакет cm-unicode устанавливается вручную в программе MiKTeX Console.
%  2. Для верстки данного примера, а именно, картинки-заглушки необходимо,
%     также вручную, загрузить пакет pgf. Этот пакет используется популярным
%     пакетом tikz.
%  3. Тест показал, что остальные пакеты MiKTeX грузит автоматически (если
%     ему разрешено автоматически грузить пакеты). Режим автозагрузки
%     настраивается в разделе Settings в MiKTeX Console.
%
%
%  Самый простой способ сверстать статью - использовать pdfLaTeX, но
%  окончательный вариант верстки сборника будет собран в LuaLaTeX,
%  так как результат получится лучшего качества, благодаря пакету microtype и
%  использованию векторных шрифтов OTF вместо растровых pdfLaTeX.
%
%  В случае возникновения вопросов и проблем с версткой статьи,
%  пишите письма на электронную почту: eugeneai@irnok.net, Черкашин Евгений.
%
%  Новые варианты корректирующего стиля будут доступны на сайте:
%        https://github.com/eugeneai/nla-style
%        файл - nla.sty
%
%  Дальнейшие инструкции - в тексте данного шаблона. Он одновременно
%  является примером статьи.
%
%  Формат LaTeX2e!

\documentclass[12pt]{llncs}  % Необходимо использовать шрифт 12 пунктов.

% При использовании pdfLaTeX добавляется стандартный набор русификации babel.
% Если верстка производится в LuaLaTeX, то следующие три строки надо
% закомментировать, русификация будет произведена в корректирующем стиле автоматом.
\usepackage{iftex}

\ifPDFTeX
\usepackage[T2A]{fontenc}
\usepackage[utf8]{inputenc} % Кодировка utf-8, cp1251 и т.д.
\usepackage[english,russian]{babel}
\fi

% Для верстки в LuaLaTeX текст готовится строго в utf-8!

% В операционной системе Windows для редактирования в кодировке utf-8
% можно использовать программы notepad++ https://notepad-plus-plus.org/,
% techniccenter http://www.texniccenter.org/,
% SciTE (самая маленькая по объему программа) http://www.scintilla.org/SciTEDownload.html
% Подойдет также и встроенный в свежий дистрибутив MiKTeX редактор
% TeXworks.

% Добавляется корректирующий стилевой файл строго после babel, если он был включен.
% В параметре необходимо указать russian, что установит не совсем стандартные названия
% разделов текста, настроит переносы для русского языка как основного и т.п.

\usepackage{todonotes} % Этот пакет нужен для верстки данного шаблона, его
                       % надо убрать из вашей статьи.

\usepackage[russian]{nla}

% Многие популярные пакеты (amsXXX, graphicx и т.д.) уже импортированы в корректирующий стиль.
% Если возникнут конфликты с вашими пакетами, попробуйте их отключить и сверстать
% текст как есть.
%
%


% Было б удобно при верстке сборника, чтобы названия рисунков разных авторов не пересекались.
% Чтоб минимизировать такое пересечение, рисунки можно поместить в отдельную подпапку с
% названием - фамилией автора или названием статьи.
%
% \graphicspath{{ivanov-petrov-pics/}} % Указание папки с изображениями в форматах png, pdf.
% или
% \graphicspath{{great-problem-solving-paper-pics/}}.


\begin{document}

% Текст оформляется в соответствии с классом article, используя дополнения
% AMS.
%
\fi

\title{Каскадный метод Лапласа}
% Первый автор
\author{Е.~И.~Зотова\inst{1}  % \inst ставит циферку над автором.
  \and  % разделяет авторов, в тексте выглядит как запятая.
% Второй автор
  Р.~Д.~Муртазина\inst{2}
  % \and
} % обязательное поле

% Аффилиации пишутся в следующей форме, соединяя каждый институт при помощи \and.
\institute{УГАТУ, Уфа, Россия \\
  \email{zot-kate83@yandex.ru}
  \and   % Разделяет институты и присваивает им номера по порядку.
УГАТУ, Уфа, Россия\\
  \email{reginaufa@yandex.ru}
% \and Другие авторы...
}

\maketitle

\begin{abstract}
В работе проинтегрировано уравнение $$z_{xy}+mxz_{x}+nyz_{y}+(2m-n+mnxy)z=0$$ каскадным методом Лапласа.

\keywords{инвариант, метод каскадного интегрирования, преобразования Лапласа} % в конце списка точка не ставится
\end{abstract}

%\section{Основные результаты} % не обязательное поле



Уравнение $		u_{xy}+a(x, y)u_{x}+b(x, y)u_{y}+c(x, y)u=0$
 можно записать в двух равносильных формах
	$$
		\begin{array}{l}
			u_{xy}+au_{x}+bu_{y}+(a_{x}+ab-h)u=\left( \dfrac{\partial}{\partial x}+b\right) \left(\dfrac{\partial}{\partial y}+a\right)u-hu=0,\\
			u_{xy}+au_{x}+bu_{y}+(a_{x}+ab-k)u=\left(\dfrac{\partial}{\partial y}+a\right)\left(\dfrac{\partial}{\partial x}+b\right)u-ku=0.
		\end{array}
	$$
	Поэтому исходное уравнение  эквивалентно каждой из систем
$$
\begin{array}{l}
		\left(\dfrac{\partial}{\partial y}+a\right)u=u_{1},\qquad \left(\dfrac{\partial}{\partial x}+b\right)u_{1}-hu=0,\\
		\left(\dfrac{\partial}{\partial x}+b\right)u=u_{-1},\qquad \left(\dfrac{\partial}{\partial y}+a\right)u_{-1}-ku=0,
		\end{array}
$$
показыващих, что если хотя бы один из инвариантов $h$ или $k$ тождественно равен нулю, то исходное уравнение интегрируется в квадратурах.

Таким образом, $X$- преобразование Лапласа генерирует из исходного уравения $E_{0}$ уравнения $E_{i}$ вида
$$
	 	\dfrac{\partial^{2}u_{i}}{\partial x\partial y}+a_{i}\dfrac{\partial u_{i}}{\partial x}+\dfrac{\partial u_{i}}{\partial y}+c_{i}u_{i}=0,\qquad i= 1, 2,\ldots,
$$
коэффициенты и инварианты которых связаны между собой соотношениями
$$
	  	\begin{array}{l}
	  		a_{i}=a_{i-1}-(\ln h_{i-1})_{y},\quad b_{i}=b_{i-1},\quad c_{i}=a_{i}b_{i}-(b_{i})_{y}-h_{i-1},\\
	  		h_{i}=2h_{i-1}-k_{i-1}-(\ln h_{i-1})_{xy},\quad 	k_{i}=h_{i-1}.
	  	\end{array}
$$
Здесь $a_{0}=a$, $b_{0}=b$, $c_{0}=c$. Аналогично $y$- преобразование Лапласа дает цепочку уравнений $E_{-i}$
$$
  	 	\dfrac{\partial^{2}u_{-i}}{\partial x\partial y}+a_{-i}\dfrac{\partial u_{-i}}{\partial x}+\dfrac{\partial u_{-i}}{\partial y}+c_{-i}u_{-i}=0,\qquad i= 1, 2,\ldots,
$$
	  где
$$
	  	\begin{array}{l}
	  		a_{-i-1}=a_{-i},\quad b_{-i-1}=b_{-i}-(\ln k_{-i})_{x},\quad c_{-i-1}=a_{-i-1}b_{-i-1}-(a_{-i})_{x}-k_{-i},\\
	  		h_{-i-1}=k_{-i},\quad 	k_{-i-1}=2k_{-i}-h_{-i}-(\ln k_{-i})_{xy}.
	  	\end{array}
$$
	  
		Для уравнений $E_i$ и $E_{-i}$ системы первого порядка имеют вид
$$
\begin{array}{l}
	  		\left(\dfrac{\partial}{\partial y}+a_{i}\right)u_{i}=u_{i+1},\qquad  \left(\dfrac{\partial}{\partial x}+b\right)u_{i+1}-h_{i}u_{i}=0,\\
  	  	\left(\dfrac{\partial}{\partial x}+b_{-i}\right)u_{-i}=u_{-i-1},\qquad  \left(\dfrac{\partial}{\partial y}+a\right)u_{-i-1}-k_{-i}u_{-i}=0,
				\end{array}
$$
 а  их инварианты $h_i$, $k_i$ и $h_{-i}$, $k_{-i}$ выражаются через инварианты предыдущих уравнений 
$$
\begin{array}{l}
 	  	h_i=(i+1)h-ik-\dfrac{\partial^{2}}{\partial x\partial y}\ln (h^ih_1^{i-1}\ldots h_{i-2}^{2} h_{i-1}),\\
 	  	k_{i}=ih-(i-1)k-\dfrac{\partial^{2}}{\partial x\partial y}\ln (h^{i-1} h_1^{i-2}\ldots h_{i-3}^{2} h_{i-2}),\\
 	  		h_{-i}=ih_{-1}-(i-1)-\dfrac{\partial^{2}}{\partial x\partial y}\ln (h^{i-1}_{-1}h_{-2}^{i-2}\ldots h_{-i+2}^{2} h_{-i+1}),\\
 	  		k_{-i}=h_{-i-1}=(i+1)h_{-1}-ih-\dfrac{\partial^{2}}{\partial x\partial y}\ln (h^i_{-1}h_{-2}^{i-1}\ldots h_{-i+1}^{2} h_{-i}).
 	  	\end{array}
 	  $$ 

Общее решение уравнения
$$
z_{xy}+mxz_{x}+nyz_{y}+(2m-n+mnxy)z=0
$$
имеет вид 
$$
z=\dfrac{e^{-nxy}}{n-m}\dfrac{\partial}{\partial x}\left(e^{(n-m)xy}\left(X(x)+\int Y(y)e^{(m-n)xy}dy \right) \right).
$$


\begin{thebibliography}{9} % или {99}, если ссылок больше десяти.

\bibitem{Zot_Sokolov} Соколов~В.В. Алгебраические структуры в теории интегрируемых систем. Новосибирск:~Наука,~2021.

\bibitem{Zot_Murt}	Муртазина~Р.Д., Кудашева~Е.Г., Низамова~А.Д., Сидельникова~Н.А. Дифференциальные уравнения в частных производных второго порядка. Устойчивость течения жидкостей в канале с линейным профилем температуры. М.:~Русайнс,~2021.


\bibitem{Zot_Zhiber}	Жибер~А.В., Муртазина~Р.Д., Хабибуллин~И.Т., Шабат~А.Б.  Характеристические кольца Ли и нелинейные интегрируемые уравнения. М.:~Юрайт,~2022.


\end{thebibliography}

% После библиографического списка в русскоязычных статьях необходимо оформить
% англоязычный заголовок.




%\end{document}

%%% Local Variables:
%%% mode: latex
%%% TeX-master: t
%%% End:
