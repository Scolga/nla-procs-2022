\begin{englishtitle} % Настраивает LaTeX на использование английского языка
% Этот титульный лист верстается аналогично.
\title{On the Theory of Game Problems with Connected  Variables}
% First author
\author{Akmal Mamatov %\inst{1}
 % \and
 % Name FamilyName2\inst{2}
 % \and
 % Name FamilyName3\inst{1}
}
\institute{Samarkand state university, Samarkand, Uzbekistan\\
  \email{email}
%  \and
%Affiliation, City, Country\\
\email{akmm1964@rambler.ru}}
% etc

\maketitle

\begin{abstract}
The paper presents the necessary and sufficient conditions for the nonemptiness of the solution set of a system of linear equations with parameters for any parameter from a given region, which arises in game problems with connected variables.

\keywords{game problem, players, first phase problem} % в конце списка точка не ставится
\end{abstract}
\end{englishtitle}

\iffalse

%%%%%%%%%%%%%%%%%%%%%%%%%%%%%%%%%%%%%%%%%%%%%%%%%%%%%%%%%%%%%%%%%%%%%%%%
%
%  This is the template file for the 6th International conference
%  NONLINEAR ANALYSIS AND EXTREMAL PROBLEMS
%  June 25-30, 2018
%  Irkutsk, Russia
%
%%%%%%%%%%%%%%%%%%%%%%%%%%%%%%%%%%%%%%%%%%%%%%%%%%%%%%%%%%%%%%%%%%%%%%%%

%  Верстка статьи осуществляется на основе стандартного класса llncs
%  (Lecture Notes in Computer Sciences), который корректируется стилевым
%  файлом конференции.
%
%  Скомпилировать файл в PDF можно двумя способами:
%  1. Использовать pdfLaTeX (pdflatex), (LaTeX+DVIPS не работает);
%  2. Использовать LuaLaTeX (XeTeX не будет работать).
%  При использовании LuaLaTeX потребуются TTF- или OTF-шрифты CMU
%  (Computer Modern Unicode). Шрифты устанавливаются либо пакетом
%  дистрибутива LaTeX cm-unicode
%              (https://www.ctan.org/tex-archive/fonts/cm-unicode),
%  либо загрузкой и установкой в операционной системе, адрес страницы:
%  http://canopus.iacp.dvo.ru/%7Epanov/cm-unicode/
%
%  В MiKTeX (дистрибутив LaTeX для ОС Windows):
%  1. Пакет cm-unicode устанавливается вручную в программе MiKTeX Console.
%  2. Для верстки данного примера, а именно, картинки-заглушки необходимо,
%     также вручную, загрузить пакет pgf. Этот пакет используется популярным
%     пакетом tikz.
%  3. Тест показал, что остальные пакеты MiKTeX грузит автоматически (если
%     ему разрешено автоматически грузить пакеты). Режим автозагрузки
%     настраивается в разделе Settings в MiKTeX Console.
%
%
%  Самый простой способ сверстать статью - использовать pdfLaTeX, но
%  окончательный вариант верстки сборника будет собран в LuaLaTeX,
%  так как результат получится лучшего качества.
%
%  В случае возникновения вопросов и проблем с версткой статьи,
%  пишите письма на электронную почту: eugeneai@irnko.net, Черкашин Евгений
%
%  Новые варианты корректирующего стиля будут доступны на сайте:
%        https://github.com/eugeneai/nla-style
%        файл - nla.sty
%
%  Дальнейшие инструкции - в тексте данного шаблона. Он одновременно
%  является примером статьи.
%
%  Формат LaTeX2e!

\documentclass[12pt]{llncs}  % Необходимо использовать шрифт 12 пунктов.

% При использовании pdfLaTeX добавляется стандартный набор русификации babel.
% Если верстка производится в LuaLaTeX, то следующие три строки надо
% закомментировать, русификация будет произведена в корректирующем стиле автоматом.
\usepackage[T2A]{fontenc}
\usepackage[cp1251]{inputenc} % Кодировка utf-8, win1251 (cp1251) не тестировалась.
\usepackage[english,russian]{babel}

% Для верстки в LuaLaTeX текст готовится строго в utf-8!

% В операционной системе Windows для редактирования в кодировки utf-8
% можно использовать программы notepad++ https://notepad-plus-plus.org/,
% techniccenter http://www.texniccenter.org/,
% SciTE (самая маленькая по объему программа) http://www.scintilla.org/SciTEDownload.html
% Подойдет также и встроенный в свежий дистрибутив MiKTeX редактор
% TeXworks.

% Добавляется корректирующий стилевой файл строго после babel, если он был включен.
% В параметре необходимо указать russian, что установит не совсем стандартные названия
% разделов текста, настроит переносы для русского языка как основного и т.п.

%\usepackage{todonotes} % Удрать из вашей статьи, нужен для верстки данного шаблона.

\usepackage[russian]{nla}

% Многие популярные пакеты (amsXXX, graphicx и т.д.) уже импортированы в корректирующий стиль.
% Если возникнут конфликты с вашими пакетами, попробуйте их отключить и сверстать
% текст как есть.
%
%


% Было б удобно при верстке сборника, чтобы названия рисунков разных авторов не пересекались.
% Чтоб минимизировать такое пересечение, рисунки можно поместить в отдельную подпапку с
% названием - фамилией автора или названием статьи.
%
% \graphicspath{{ivanov-petrov-pics/}} % Указание папки с изображениями в форматах png, pdf.
% или
% \graphicspath{{great-problem-solving-paper-pics/}} % Указание папки с изображениями в форматах png, pdf.


\begin{document}

% Текст оформляется в соответствии с классом article, используя дополнения
% AMS.
%
\fi

\title{К теории игровых задач со связанными переменными}
% Первый автор
\author{А.~Р.~Маматов} % обязательное поле

% Аффилиации пишутся в следующей форме, соединяя каждый институт при помощи \and.
\institute{СамГУ имени Ш. Рашидова, Самарканд, Узбекистан\\
  \email{akmm1964@rambler.ru}}

\maketitle

\begin{abstract}
Приведены необходимые и достаточные условия непустоты множества решений системы линейных уравнения с параметрами для любого параметра из заданной области,  возникающей в игровых задачах со связанными переменными.
\keywords{Игровая задача, игроки, задача первой фазы} % в конце списка точка не ставится
\end{abstract}

\section{Основные результаты} % не обязательное поле

Как известно \cite{mmGabasov1}, задачу о непустоте множества планов задачи
линейного программирования, можно решить с помощью специальной
задачи линейного программирования (задача первой фазы). В данной
работе исследуется аналогичная задача, возникающая в игровых задачах со связанными переменными: задача определения непустоты множеств решений систем линейных уравнений с параметрами для любого параметра из заданной области \cite{mmIvanilov1972,mmGermeyer1,mmFalk1973,mmMamatov2006}.

Пусть имеется два игрока, которые выбирают векторы $x$ и $y$ соответственно из множеств
$$\quad X=\{x\mid f_*\leq x \leq f^*\}, Y(x)=\{y\mid g_* \leq y \leq g^*,\quad Ax+By = b\},$$
поочередно, сначала первый игрок выбирает $x$, затем, зная $x$,
второй игрок выбирает $y$.
Здесь \begin{math} x=x(J), \quad f_*=f_*(J),\quad f^*=f^*(J) \in \mathbb{R}^n, y=y(K),g_{*}=g_{*}(K), g^{*}=g^{*}(K) \in \mathbb{R}^l, b\in \mathbb{R}^m, A=A(I,J) \in \mathbb{R}^{mxn}, B=B(I,K)\in \mathbb{R}^{mxl},  rankB<l, I=\{1,2,...,m\}, J=\{1,2,...,n\},
K=\{1,2,...,l\}.\end{math}

Вектор$\quad x\in X$ называется стратегией первого игрока, а вектор $y\in Y(x)$ ---$x$-стратегией второго игрока.

Задача (1): требуется определить, для любой стратегии первого игрока $x \in X$, существует ли соответствующий $x$-стратегия второго игрока, т.е. $\forall x \in X, Y(x)\ne \emptyset$?

Наряду с задачей (1) рассмотрим максиминную
 задачу (2):
  $$f(x)=\min_{g_*\leq y \leq g^*}\sum_{i\in I}|A(i,J)x(J)+B(i,K)y(K)-b(i)|\rightarrow \max_{x\in X}. \quad \quad \quad  \quad     $$


 \begin{theorem}{\cite{mmMamatov2006}.}  Оптимальные значения целевых функций задачи 
(2) и задачи (3):
$$F(x)=\min_{(y,\xi,\eta)\in H(x)}(e'\xi+e'\eta)\rightarrow \max_{x\in X},\quad \quad \quad \quad \quad $$
$H(x)=\{(y,\xi,\eta)\mid Ax+By-\xi+\eta = b,g_* \leq y \leq
g^*, \xi\geq0,\eta\geq0\},e'=(1,1,...,1),$
 совпадают. 
\end{theorem}

Задача (3) является линейной максиминной задачей со связанными переменными, в
которой $\forall x \in X,H(x) \neq \emptyset$.


\begin{lemma} Для  непустоты множество $Y(x), \forall x \in X$ необходимо и достаточно, чтобы в решении задачи (3)
$(x^0, y^0,\xi^0,\eta^0)$ равнялись нулю компоненты $\xi^0,\eta^0$.
\end{lemma}

 Пусть  $K_{op}=\{k_{1},k_2,...,k_m\} \subset K$,
опора \cite{mmGabasov2} задачи (3), т.е. $detB(I,K_{op})\neq0.$

Определим:
$$\gamma^{*}_k=\max_{f_*\leq x \leq f^*,g_*(K_n)\leq y(K_n) \leq
g^*(K_n)}h_k(b(I)-A(I,J)x(J)-B(I,K_n)y(K_n)),$$
$$\gamma_{*k}=\min_{f_*\leq x \leq f^*,g_*(K_n)\leq y(K_n) \leq
g^*(K_n)}h'_k(b(I)-A(I,J)x(J)-B(I,K_n)y(K_n)),$$
$k\in K_{op}.$

Здесь $h_k, k$-ая строка матрицы $[{B}(I,K_{op})]^{-1}, K_n=K \backslash K_{op}.$

 \begin{theorem} Если
  $${g}_{*k}\leq  \gamma_{*k} \leq \gamma^{*}_k \leq g^{*}_k, k\in
K_{op}, $$ 
 то $Y(x)\ne \emptyset \quad \forall x \in X.$
\end{theorem}



\begin{thebibliography}{9} % или {99}, если ссылок больше десяти.
\bibitem{mmGabasov1} Габасов Р. и др. Методы оптимизации. - Мн.:  Издательство
``Четыре четверти'', 2011.

\bibitem{mmIvanilov1972} Иванилов Ю.П. Двойственные полуигры~//Известия АН СССР. Серия
техническая кибернетика.1972. \textnumero~4. С.~3--9.

\bibitem{mmGermeyer1} Гермейер Ю.Б. Игры с непротивоположными интересами.--М.:Наука,1976.


\bibitem{mmFalk1973} Falk J.E. Linear maxmin problem~// J.~Math. Program. 1973. Vol.~5.№ 2. Pp.~169--188.

\bibitem{mmMamatov2006} Маматов А.Р.Алгоритм решения одной игры двух лиц
с передачей информации~// ЖВМиМФ, 2006, Т.46, №10, С. 1784-1789.

\bibitem{mmGabasov2} Габасов Р., Кириллова Ф.М., Тятюшкин А. И. Конструктивные
методы оптимизации. Ч.1. Линейные задачи.-Минск:Университетское.1984.


\end{thebibliography}

% После библиографического списка в русскоязычных статьях необходимо оформить
% англоязычный заголовок.




%\end{document}

%%% Local Variables:
%%% mode: latex
%%% TeX-master: t
%%% End:
