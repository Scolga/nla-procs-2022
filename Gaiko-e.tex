



\iffalse
\documentclass[12pt]{llncs}

\usepackage{todonotes}

\usepackage{nla}

\begin{document}
\fi

\title{Catastrophe Theory and Global Bifurcations of~Limit~Cycles}

\author{Valery Gaiko}
\institute{United~Institute~of~Informatics~Problems, National~Academy~of~Sciences~of~Belarus, Minsk, Belarus\\
\email{valery.gaiko@gmail.com}}

\maketitle

\begin{abstract}
Applying Catastrophe Theory, we carry out a qualitative analysis of multi-parameter
polynomial dynamical systems and study global bifurcations of limit cycles.

\keywords{Catastrophe Theory, Wintner--Perko termination principle, polynomial dynamical
system, bifurcation, limit cycle, strange attractor}
\end{abstract}

\section{Introduction}

Applying Catastrophe Theory, we carry out a global qualitative analysis of multi-parameter
polynomial dynamical systems. To~control all their limit cycle bifurcations, especially,
bifurcations of multiple limit cycles, it~is necessary to know the properties and combine
the effects of all their rotation parameters. It can be done by means of the development of
new bifurcation geometric methods based on Perko's planar termination principle~\cite{gaiko1}.
This principle is a consequence of the principle of natural termination which
was applied by A.\,Wintner for studying one-parameter families of periodic
orbits of the restricted three-body problem to show that in the analytic case any
one-parameter family of periodic orbits can be uniquely continued through any bifurcation
except a period-doubling bifurcation. Such a bifurcation can happen, e.\,g., in a
three-dimensional Lorenz system. But this cannot happen for planar systems. That is why
the Wintner--Perko termination principle is applied for studying multiple limit cycle
bifurcations of planar polynomial dynamical systems~\cite{gaiko1}.
If we do not know the cyclicity of the termination points, then, applying canonical systems with
field rotation parameters, we use geometric properties of the spirals filling the interior and
exterior domains of limit cycles.

\section{The main results}

Using this approach, we have solved, e.\,g., {\it Hilbert's Sixteenth Problem}
on the maximum number and distribution of limit cycles for the general Li\'{e}nard
polynomial system with an arbitrary number of singular points~\cite{gaiko2}, the
Kukles cubic-linear system~\cite{gaiko3}, the Euler--Lagrange--Li\'{e}nard polynomial
mechanical system~\cite{gaiko4}, Leslie--Gower systems which model the population dynamics
in real ecological or biomedical systems~\cite{gaiko5} and a~reduced planar quartic Topp system
which models the dy\-na\-mics of diabetes~\cite{gaiko6}. Finally, applying a similar approach,
we have considered various applications of three-dimensional polynomial dynamical systems and,
in~particular, completed the strange attractor bifurcation scenario in Lorenz type systems
globally connecting the homoclinic, period-doubling, Andronov--Shilnikov, and period-halving
bifurcations of their limit cycles~\cite{gaiko7}.

\begin{thebibliography}{9}
\bibitem{gaiko1}
Gaiko~V.\,A. Global Bifurcation Theory and Hilbert's Sixteenth Problem. Boston, Kluwer
Academic Publishers, 2003.

\bibitem{gaiko2}
Gaiko~V.\,A. Maximum number and distribution of limit cycles in the general Li\'{e}nard
polynomial system. Adv. Dyn. Syst. Appl. 2015. Vol.~10, no~2. Pp.~177--188.

\bibitem{gaiko3}
Gaiko~V.\,A. Global bifurcation analysis of the Kukles cubic system. Int. J.~Dyn. Syst.
Differ. Equ. 2018. Vol.~8, no~4. Pp.~326--336.

\bibitem{gaiko4}
Gaiko~V.\,A. Limit cycles of multi-parameter polynomial dynamical systems. J.~Math. Sci.
2022. Vol.~260, no.~5. Pp.~662--677.

\bibitem{gaiko5}
Gaiko~V.\,A., Vuik~C. Global dynamics in the Leslie--Gower model with the Allee effect.
Int. J.~Bifurcation Chaos. 2018. Vol.~28. Pp.~1850151.

\bibitem{gaiko6}
Gaiko~V.\,A., Broer~H.\,W., Sterk~A.\,E. Global bifurcation analysis of Topp system.
Cyber. Phys. 2019. Vol.~8, no.~4. Pp.~244--250.

\bibitem{gaiko7}
Gaiko~V.\,A. Global bifurcation analysis of the Lorenz system. J.~Nonlinear Sci.
Appl. 2014. Vol.~7, no.~6. Pp.~429--434.

\end{thebibliography}
%\end{document}
