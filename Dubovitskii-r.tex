\begin{englishtitle} % Настраивает LaTeX на использование английского языка
% Этот титульный лист верстается аналогично.
\title{On the Right Invertibility of the Differential for the Equality Constraint Operator and the Implicit Function Theorem in a General Optimal Control Problem\thanks{Работа выполнена по теме Государственного задания, номер гос. регистрации ЦИТИС -AAAA-A19-119022690098-3.
}}
% First author
\author{Vladimir A. Dubovitskij
}
\institute{Institute of Problems of Chemical Physics, Chernogolovka, Russia \\
%  \and
\email{dubv@icp.ac.ru}}
% etc

\maketitle

\begin{abstract}
 The property of the right invertibility of the Frechet differential of the equality-type constraint operator in the general optimal control problem is proved. It follows from this that the linear subspace tangent to the constraint has a topological complement and the implicit function theorem applies to the description of the constraint

\keywords{optimal control problem, equality constraints, Frechet differential, right invertibility, topological direct sum of subspaces, implicit function} % в конце списка точка не ставится
\end{abstract}
\end{englishtitle}

\iffalse

%%%%%%%%%%%%%%%%%%%%%%%%%%%%%%%%%%%%%%%%%%%%%%%%%%%%%%%%%%%%%%%%%%%%%%%%
%
%  This is the template file for the 6th International conference
%  NONLINEAR ANALYSIS AND EXTREMAL PROBLEMS
%  June 25-30, 2018
%  Irkutsk, Russia
%
%%%%%%%%%%%%%%%%%%%%%%%%%%%%%%%%%%%%%%%%%%%%%%%%%%%%%%%%%%%%%%%%%%%%%%%%

%  Верстка статьи осуществляется на основе стандартного класса llncs
%  (Lecture Notes in Computer Sciences), который корректируется стилевым
%  файлом конференции.
%
%  Скомпилировать файл в PDF можно двумя способами:
%  1. Использовать pdfLaTeX (pdflatex), (LaTeX+DVIPS не работает);
%  2. Использовать LuaLaTeX (XeLaTeX будет работать тоже).
%  При использовании LuaLaTeX потребуются TTF- или OTF-шрифты CMU
%  (Computer Modern Unicode). Шрифты устанавливаются либо пакетом
%  дистрибутива LaTeX cm-unicode
%              (https://www.ctan.org/tex-archive/fonts/cm-unicode),
%  либо загрузкой и установкой в операционной системе, адрес страницы:
%              http://canopus.iacp.dvo.ru/%7Epanov/cm-unicode/
%  Второй вариант не будет работать в XeLaTeX.
%
%  В MiKTeX (дистрибутив LaTeX для ОС Windows):
%  1. Пакет cm-unicode устанавливается вручную в программе MiKTeX Console.
%  2. Для верстки данного примера, а именно, картинки-заглушки необходимо,
%     также вручную, загрузить пакет pgf. Этот пакет используется популярным
%     пакетом tikz.
%  3. Тест показал, что остальные пакеты MiKTeX грузит автоматически (если
%     ему разрешено автоматически грузить пакеты). Режим автозагрузки
%     настраивается в разделе Settings в MiKTeX Console.
%
%
%  Самый простой способ сверстать статью - использовать pdfLaTeX, но
%  окончательный вариант верстки сборника будет собран в LuaLaTeX,
%  так как результат получится лучшего качества, благодаря пакету microtype и
%  использованию векторных шрифтов OTF вместо растровых pdfLaTeX.
%
%  В случае возникновения вопросов и проблем с версткой статьи,
%  пишите письма на электронную почту: eugeneai@irnok.net, Черкашин Евгений.
%
%  Новые варианты корректирующего стиля будут доступны на сайте:
%        https://github.com/eugeneai/nla-style
%        файл - nla.sty
%
%  Дальнейшие инструкции - в тексте данного шаблона. Он одновременно
%  является примером статьи.
%
%  Формат LaTeX2e!

\documentclass[12pt]{llncs}  % Необходимо использовать шрифт 12 пунктов.

% При использовании pdfLaTeX добавляется стандартный набор русификации babel.
% Если верстка производится в LuaLaTeX, то следующие три строки надо
% закомментировать, русификация будет произведена в корректирующем стиле автоматом.
%\usepackage{iftex}

%\ifPDFTeX
\usepackage[T2A]{fontenc}
\usepackage[utf8]{inputenc} % Кодировка utf-8, cp1251 и т.д.
\usepackage[english,russian]{babel}
%\fi

% Для верстки в LuaLaTeX текст готовится строго в utf-8!

% В операционной системе Windows для редактирования в кодировке utf-8
% можно использовать программы notepad++ https://notepad-plus-plus.org/,
% techniccenter http://www.texniccenter.org/,
% SciTE (самая маленькая по объему программа) http://www.scintilla.org/SciTEDownload.html
% Подойдет также и встроенный в свежий дистрибутив MiKTeX редактор
% TeXworks.

% Добавляется корректирующий стилевой файл строго после babel, если он был включен.
% В параметре необходимо указать russian, что установит не совсем стандартные названия
% разделов текста, настроит переносы для русского языка как основного и т.п.

%\usepackage{todonotes} % Этот пакет нужен для верстки данного шаблона, его
                       % надо убрать из вашей статьи.

\usepackage[russian]{nla}

% Многие популярные пакеты (amsXXX, graphicx и т.д.) уже импортированы в корректирующий стиль.
% Если возникнут конфликты с вашими пакетами, попробуйте их отключить и сверстать
% текст как есть.
%
%


% Было б удобно при верстке сборника, чтобы названия рисунков разных авторов не пересекались.
% Чтоб минимизировать такое пересечение, рисунки можно поместить в отдельную подпапку с
% названием - фамилией автора или названием статьи.
%
% \graphicspath{{ivanov-petrov-pics/}} % Указание папки с изображениями в форматах png, pdf.
% или
% \graphicspath{{great-problem-solving-paper-pics/}}.


\begin{document}

% Текст оформляется в соответствии с классом article, используя дополнения
% AMS.
%
\fi

\title{О правой обратимости дифференциала для оператора равенственных ограничений и теорема о неявной функции в общей задаче оптимального управления}
% Первый автор
\author{B.~A.~Дубовицкий %\inst{1,2}  % \inst ставит циферку над автором.
} % обязательное поле

% Аффилиации пишутся в следующей форме, соединяя каждый институт при помощи \and.
\institute{ФГБУН Институт Проблем Химической Физики РАН, г.Черноголовка, Россия \\
%  \and   % Разделяет институты и присваивает им номера по порядку.
  \email{dubv@icp.ac.ru}
% \and Другие авторы...
}

\maketitle

\begin{abstract}

Доказано свойство правой обратимости дифференциала Фреше оператора ограничений типа равенства в общей задаче оптимального управления. Из этого следует, что касательное к ограничению линейное подпространство имеет топологическое дополнение и к описанию ограничения применима теорема о неявной функции.

\keywords{задача оптимального управления, ограничения равенства, дифференциал Фреше, правая обратимость, топологическая прямая сумма подпространств, неявная функция} % в конце списка точка не ставится
\end{abstract}

%\section{Основные результаты} % не обязательное поле

Важным  элементом математического аппарата теории оптимального управления (ОУ) является анализ равенственного ограничения, которое для задачи с закреплённым интервалом времени $\Delta=[t_0,t_1]$ можно задать операторным уравнением $P(w)=0$.
Здесь $P(w)=(K(x),\dot{x}-f(x,u,t),g(x,u,t))$
  есть составной нелинейный оператор равенственных ограничений, отображающий пространство всех траекторий $w=(x(\cdot),u(\cdot))$ $W=A^{d_x}(\Delta)\times L_{\infty}^{d_u}(\Delta)$
   в пространство $Z=\mathbb{R}^{d_K}\times L_1^{d_x}(\Delta)\ times L_{\infty}^{d_g}(\Delta)$.
  Здесь символы $d_x$, $d_u$, $d_g$, $d_K$,
 обозначают целочисленные положительные размерности соответствующих векторных функций, $A^{d_x}(\Delta)$ --- Банахово пространство абсолютно непрерывных фазовых компонент траектории $w$. Предполагается, что $f,g$  непрерывно дифференцируемые по $x,u$  функции, измеримые по $t$, оператор $K(x)$, отображающий $C^{d_x}(\Delta)$  в  $\mathbb{R}^{d_K}$ непрерывно дифференцируем. Считаем, как обычно, что якобиан $\partial g(x,u,t) / \partial u$ смешанного ограничения $g(x,u,t)=0$  имеет локально равномерно полный ранг. При сделанных предположениях оператор $P(w)$  непрерывно дифференцируем по Фреше. Краеугольным камнем теории ОУ \cite{Dub1}  является то обстоятельство, что образ дифференциала Фреше оператора $P(w)$ замкнут, благодаря чему в соответствующей абстрактной задаче на условный экстремум с ограничением $P(w)=0$  применима теорема Люстерника \cite{Dub2} и может быть доказано необходимое условие минимума в форме существования нетривиального набора сопряженных множителей Лагранжа. В работе установлено существенно более сильное свойство дифференциала оператора ограничения равенства в общей задаче ОУ.

\begin{theorem} Линейный оператор --- дифференциал $P’(w)$ для любой допустимой траектории $w$ имеет замкнутый образ $Im\,P’(w)$ конечной коразмерности, причём $P’(w)$ правообратим на $Im\,P’(w)$. То есть существует непрерывный оператор $B:Z\to W$ такой, что произведение $P’(w)B$  есть проектор пространства $Z$  на  подпространство $Im\,P’(w)$.  
\end{theorem}

Из этой теоремы вытекают важные следствия

{\bf Следствие 1.} Нулевое подпространство дифференциала $Ker\, P’(w)$, согласно \cite{Dub3}(Предложение 13), имеет линейное топологическое дополнение $Im\, B$ в пространстве $W$, т.е Банахово пространство траекторий разлагается в прямую сумму замкнутых подпространств $W=Ker\,P’(w) \oplus Im\, B$. 

{\bf Следствие 2.} Если для траектории $w_0$ дифференциал $P’(w_0)$ имеет полный образ, т.е. $Im\, P’(w_0)=Z$, то на основе разложения $W=Ker\, P’(w)\oplus Im\, B$  в окрестности траектории $w_0$  к уравнению $P(w)=P(w_0)$ применима классическая теорема о неявной функции, и поэтому множество уровня, заданное ограничением типа равенства,  является гладким многообразием в $W$. 

Теорема о неявной функции в приведённой форме существенно проясняет локальную геометрию равенственного ограничения в задачах ОУ и полезна для развития теории таких задач, в том числе для теории принципа максимума. Ранее доступная  информация об ограничении типа равенства фактически сводилась к описанию касательных вариаций, как подпространства нулей дифференциала $P’(w_0)$.






\begin{thebibliography}{9} % или {99}, если ссылок больше десяти.


\bibitem{Dub1} Дубовицкий А.Я., Милютин А.А. Задачи на экстремум при наличии ограничений // ЖВМ и МФ. 1965. T.~5, \textnumero~3. C.~395--453.

\bibitem{Dub2} Дмитрук А.В., Милютин А.А., Осмоловский Н.П. Теорема Люстерника и теория экстремума // УМН. 1980. T.35,  6:216. C~11--46.

\bibitem{Dub3}  Бурбаки Н. Топологические векторные пространстваю Москва: ИЛ, 1959.


\end{thebibliography}

% После библиографического списка в русскоязычных статьях необходимо оформить
% англоязычный заголовок.



%\end{document}

%%% Local Variables:
%%% mode: latex
%%% TeX-master: t
%%% End:
