
\iffalse
%%%%%%%%%%%%%%%%%%%%%%%%%%%%%%%%%%%%%%%%%%%%%%%%%%%%%%%%%%%%%%%%%%%%%%%%
%
% This is the template file for the 6th International conference
% NONLINEAR ANALYSIS AND EXTREMAL PROBLEMS
% June 25-30, 2018
% Irkutsk, Russia
%
%%%%%%%%%%%%%%%%%%%%%%%%%%%%%%%%%%%%%%%%%%%%%%%%%%%%%%%%%%%%%%%%%%%%%%%%
% The preparation of the article is based on the standard llncs class
% (Lecture Notes in Computer Sciences), which is adjusted with style
% file of the conference.
%
% There are two ways of compilation of the file into PDF
% 1. Use pdfLaTeX (pdflatex), (LaTeX+DVIPS will not work);
% 2. Use LuaLaTeX (XeLaTeX will work too).
% When using LuaLaTeX You will need TTF or OTF CMU fonts
% (Computer Modern Unicode). The fonts are installed with 'cm-unicode' package in
% a distribution of LaTeX % (https://www.ctan.org/tex-archive/fonts/cm-unicode),
% either by downloading and installing these fonts system wide, the address of their page is
% http://canopus.iacp.dvo.ru/%7Epanov/cm-unicode/
% The second option won't work in XeLaTeX.
%
% For MiKTeX (LaTeX distribution for Windows),
%  1. Package 'cm-unicode' is installed manually with the MiKTeX administration Console.
%  2. For the compilation of this example, namely, the stub figure, one will also need to
% download package 'pgf' manually. This package uses in the popular
% package tikz.
%  3. Tests showed that the rest of the required packages MiKTeX loads automatically (if
%     it is allowed). The 'auto download' option is
%     configured in 'Settings' section in MiKTeX Console.
%
%
% The easiest way to compile an article is to use pdfLaTeX, but
% the final layout of the book will be compiled with LuaLaTeX,
% as a result will be of better quality thanks to the package 'microtype' and
% use vector OTF instead of standard raster fonts of pdfLaTeX.
%
% In the case of questions and problems with the article compilation,
% write letters to e-mail: eugeneai@irnok.net, Cherkashin Evgeny.
%
% New version of the correcting style file will be available at the website:
%     https://github.com/eugeneai/nla-style
%     file - nla.sty
%
% Further instructions are in the text body of the template. The template itself
% is an article example.
%
% The LaTeX2e format is used!

% 12 points font size is used.
\documentclass[12pt]{llncs}

% The correcting style file is added.
\usepackage{todonotes}

\usepackage{nla} % This package is needed for compiling
                 % this template, it should be removed
                 % from your article.

% Many popular packages (amsXXX, graphicx, etc.) are already imported in the style file.
% If there is a conflict with your packages, try disabling them and compile
% the text.
%
% It would be convenient in the layout of the proceedings if the file names
% of the figures of different authors do not clash.
% To minimize the clash, the drawings can be placed in a separate subfolder
% named after the author or the title of the paper.
%
% \graphicspath{{ivanov-petrov-pics/}} % specifies the folder with images in png, pdf formats.
% or
% \graphicspath{{great-problem-solving-paper-pics/}}.

\begin{document}

% Text should be formatted in accordance with the 'article' class, using extensions like
% AMS.
%
\fi
\title{On a Local Search Method for Bilevel Optimization Problems with an Equilibrium at the Lower Level\thanks{The research is funded by the fundamental research project no.~121041300065-9 (code FWEW-2021-0003).}}
% First author
\author{Andrei V. Orlov%\inst{}\orcidID{0000-0003-1593-9347} 
}
\institute{ISDCT  SB   RAS,  Irkutsk, Russia \\
\email{anor@icc.ru}}
% etc

\maketitle

\begin{abstract}
This work addresses the developing Special Local Search Method for one of the classes of the bilevel optimization problems (BOPs) with equilibrium at the lower level. 
\keywords{bilevel optimization, bilevel problems with equilibrium at the lower level, bimatrix game, d.c. constraint problem, local search method}

\end{abstract}

% at the end of the list, there should be no final dot
\section{The main results}

Nowadays, bilevel optimization is a highly developing area of optimization \cite{NewBilevelBook}.
Especially, the study of bilevel problems with many players at the lower (Single-Leader-Multi-Follower-Problem (SLMFP))
or at the upper level (Multi-Leader-Single-Follower-Problem (MLSFP)), or even Multi-Leader-Fol\-lo\-wer-Prob\-lems (MLFPs), where there are one or more Nash games at each level, is gaining popularity (see Chapter 3 in \cite{NewBilevelBook}).

In this connection, the work addresses one of the classes of the bilevel optimization problems (BOPs) with equilibrium at the lower level. We study quadratic BOPs with a bimatrix game at the lower level in their optimistic statement \cite{OrlovMOTOR2021}.
$$
\left.
  \begin{array}{c}
   \langle x, Cx \rangle + \langle c, x \rangle + \langle y, D_1y \rangle + \langle d_1, y \rangle +
    \langle z, D_2z \rangle + \langle d_2, z \rangle  \uparrow \max\limits_{x,y,z}, \\
    x \in X , \;
    (y,z) \in NE(\Gamma B(x)),
  \end{array}
\right \}
\eqno{({\cal BP}_{\Gamma B})}
$$
 where $X = \{ x \in I\!\!R^m \;|\; Ax \leq a, \;x \geq 0, \;\; \langle b_1, x \rangle + \langle b_2, x \rangle =1 \}$, $NE(\Gamma B(x))$ is {a} set of Nash
  equilibrium points of the game
$$
\left.
  \begin{array}{c}
   \langle y, B_1 z \rangle \uparrow \max\limits_{y},  \;\;\;\;\;y \in Y(x) = \{y \mid y \geq 0, \;\;\; \langle e_{n_1},y \rangle = \langle b_1, x \rangle \}, \\
   \langle y, B_2 z \rangle \uparrow \max\limits_{z},  \;\;\;\;\;z \in Z(x) = \{z \mid z \geq 0,\;\;\; \langle e_{n_2},z \rangle = \langle b_2, x \rangle \};
  \end{array}
\right \}  \eqno{(\Gamma B(x))}
$$
$c,\; b_1,\; b_2 \in I\!\!R^m$; $ y,\; d_1 \in I\!\!R^{n_1}$; $ z,\; d_2 \in I\!\!R^{n_2}$; $ a \in I\!\!R^{m_1}$;
$b_1 \geq 0, \;b_1 \ne 0$, \linebreak $b_2 \geq 0,\; b_2 \ne 0;$ $A, B_1, B_2, C, D_1, D_2$ are matrices of appropriate dimension.
$C=C^T$, $D_1=D_1^T$, $D_2=D_2^T$ are positive semidefinite matrices, so, the objective function of the leader is convex.

First of all, we transform the problem $({\cal BP}_{\Gamma B})$ to the following single-level optimization problem which is equivalent to it
from the global solutions point of view by special optimality conditions \cite{OrlovMOTOR2021}:
$$
\left.
  \begin{array}{c}
    - f_0(x,y,z)\stackrel\triangle{=}  \langle x, Cx \rangle + \langle c, x \rangle + \langle y, D_1y \rangle + \langle d_1, y \rangle + \\ +
    \langle z, D_2z \rangle + \langle d_2, z \rangle \uparrow \max\limits_{x,y,z,\alpha,\beta}, \\
    (x,y,z) \in S \stackrel\triangle{=} \{ x,y,z \mid Ax \leq a, \;\; x \geq 0, \;\; \langle b_1, x \rangle + \langle b_2, x \rangle =1, \\
     y \geq 0, \;\; \langle e_{n_1}, y \rangle = \langle b_1, x \rangle, \;\;\;\;\;
     z \geq 0, \;\; \langle e_{n_2}, z \rangle = \langle b_2, x \rangle \}, \\
    \langle b_1, x \rangle (B_1z) \leq \alpha e_{n_1}, \;\;\;
    \langle b_2, x \rangle (yB_2) \leq \beta e_{n_2}, \\
    \langle y, (B_1+B_2) z \rangle = \alpha + \beta.
  \end{array}
\right \}  \eqno{({\cal PB})}
$$

It is easy to prove that all of nonconvex functions from the formulation $({\cal PB})$ can be represented as a difference of two convex functions \cite{OrlovMOTOR2021}.
So, we can formulate the problem $({\cal  PB})$ as the minimization
problem with a convex quadratic objective function and $(n_1+n_2+1)$ d.c. constraints:
$$%\begin{equation}\label{eq:p}
%\lefteqn{\hspace{-1.9cm}({\cal P})\colon}
\left.
\begin{array}{c}
f_0(x,y,z) \downarrow \min\limits_{x,y,z,v},\quad (x,y,z)\in S,\\
f_i(x,z,\alpha):=g_i(x,z,\alpha)-h_i(x,z)\leq 0,\;\;\;\;\;i \in\{1,\dots,n_1\}=: {\cal I}, \\
f_j(x,y,\beta):=g_j(x,y,\beta)-h_j(x,y)\leq 0,\;\;\;j \in \{n_1+1,\dots,n_1+n_2\} =: {\cal J}, \\
f_{n_1+n_2+1}(y,z,\alpha,\beta):=g_{n_1+n_2+1}(y,z,\alpha,\beta)-h_{n_1+n_2+1}(y,z) = 0,
\end{array}
\right\}
\eqno{({\cal DCC})}
$$%\end{equation}
where the functions $f_0$; $g_i,\;h_i \;\forall i\in {\cal I} = \{1,...,n_1\}$; $g_j,\;h_j $\linebreak $\forall j\in {\cal J} = \{n_1+1,...,n_1+n_2\}$;
$g_{n_1+n_2+1}$, and $h_{n_1+n_2+1}$, are convex with respect to the aggregate of all their variables; and the set
$$  %\begin{array}{c}
S\!=\!\left\{x,y,z\! \geq 0\;|\;Ax \leq a,\; \langle b_1, x \rangle \! +\! \langle b_2, x \rangle \!=\!1,\;
       \langle e_{n_1}, y \rangle\! =\! \langle b_1, x \rangle,\;
    \langle e_{n_2}, z \rangle \!= \!\langle b_2, x \rangle\!\right\},  %\end{array}
 $$ is, obviously, convex too.

Now we can apply the special Global Search Theory (GST) for general d.c. optimization problems to the problem $({\cal DCC})$ 
\cite{StrekBook,StrekPardalos,StrekOptima2020}.

Following this theory, the algorithms for solving a nonconvex problem in question consists of two stages: local search which take into account the features of the problem under study and global searches procedures that help to jump out the points provided by the local search based on Global Optimality Conditions proved by A.S. Strekalovsky \cite{StrekBook,StrekPardalos,StrekOptima2020}. In this work we study the issue of developing Special Local Search Method.

In order to do it we propose the further transformation of the problem using the Exact Penalization Theory \cite{StrekOptima2020,Nocedal,Bonnans} and
elaborate the Special Penalty Local Search Method based on the results from \cite{StrLocNew}.



\begin{thebibliography}{9} % or {99}, if there is more than ten references.

\bibitem{NewBilevelBook}
Bilevel Optimization: Advances and Next Challenges~/ Ed. by S.~Dempe, A.~Zemkoho. Cham: Springer, 2020.

\bibitem{OrlovMOTOR2021}
Orlov A.V. On Solving Bilevel Optimization Problems with a Nonconvex Lower Level: The Case of a Bimatrix Game.
Lecture Notes in Computer Science. MOTOR 2021 / Ed. by P.~Pardalos, M.~Khachay, A.~Kazakov. Vol. 12755. Cham: Springer, 2021. P.~235--249.

\bibitem{StrekBook}
Strekalovsky A.S. Elements of nonconvex optimization. Nauka, Novosibirsk, 2003.~[In Russian]

\bibitem{StrekPardalos}
Strekalovsky~A.S. On Solving Optimization Problems with Hidden Nonconvex Structures. Optimization
in Science and Engineering / Ed. by T.M.~Rassias, C.A.~Floudas, S.~Butenko. N.Y.: Springer, 2014. P.~465--502.

\bibitem{StrekOptima2020}
Strekalovsky~A.S. On a Global Search in D.C. Optimization Problems. Optimization and Applications. OPTIMA 2019. Communications in Computer and Information Science / Ed. by M.~Jacimovic et al. Vol. 1145. Cham: Springer, 2020. P.~222--236.

\bibitem{Nocedal}
Nocedal~J. and Wright~S.J.  Numerical optimization. Springer-Verlag, New York-Berlin-Heidelberg, 2000.

\bibitem{Bonnans}
Bonnans~J.-F., Gilbert J.C., Lemarechal C., Sagastizabal C.A. Numerical optimization: theoretical
and practical aspects. Springer, Berlin-Heidelberg, 2006.

\bibitem{StrLocNew}
Strekalovsky A.S. Local search for nonsmooth DC optimization with DC equality and inequality constraints~// Numerical Nonsmooth Optimization -- State of the Art Algorithms / Ed. by A.~Bagirov et al. Numerical Nonsmooth Optimization -- State of the Art Algorithms. Cham: Springer, 2020. P.~229--261.

\end{thebibliography}
%\end{document}

%%% Local Variables:
%%% mode: latex
%%% TeX-master: t
%%% End:
