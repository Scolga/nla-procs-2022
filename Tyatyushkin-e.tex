
\iffalse
%%%%%%%%%%%%%%%%%%%%%%%%%%%%%%%%%%%%%%%%%%%%%%%%%%%%%%%%%%%%%%%%%%%%%%%%
%
% This is the template file for the 6th International conference
% NONLINEAR ANALYSIS AND EXTREMAL PROBLEMS
% June 25-30, 2018
% Irkutsk, Russia
%
%%%%%%%%%%%%%%%%%%%%%%%%%%%%%%%%%%%%%%%%%%%%%%%%%%%%%%%%%%%%%%%%%%%%%%%%
% The preparation of the article is based on the standard llncs class
% (Lecture Notes in Computer Sciences), which is adjusted with style
% file of the conference.
%
% There are two ways of compilation of the file into PDF
% 1. Use pdfLaTeX (pdflatex), (LaTeX+DVIPS will not work);
% 2. Use LuaLaTeX (XeLaTeX will work too).
% When using LuaLaTeX You will need TTF or OTF CMU fonts
% (Computer Modern Unicode). The fonts are installed with 'cm-unicode' package in
% a distribution of LaTeX % (https://www.ctan.org/tex-archive/fonts/cm-unicode),
% either by downloading and installing these fonts system wide, the address of their page is
% http://canopus.iacp.dvo.ru/%7Epanov/cm-unicode/
% The second option won't work in XeLaTeX.
%
% For MiKTeX (LaTeX distribution for Windows),
%  1. Package 'cm-unicode' is installed manually with the MiKTeX administration Console.
%  2. For the compilation of this example, namely, the stub figure, one will also need to
% download package 'pgf' manually. This package uses in the popular
% package tikz.
%  3. Tests showed that the rest of the required packages MiKTeX loads automatically (if
%     it is allowed). The 'auto download' option is
%     configured in 'Settings' section in MiKTeX Console.
%
%
% The easiest way to compile an article is to use pdfLaTeX, but
% the final layout of the book will be compiled with LuaLaTeX,
% as a result will be of better quality thanks to the package 'microtype' and
% use vector OTF instead of standard raster fonts of pdfLaTeX.
%
% In the case of questions and problems with the article compilation,
% write letters to e-mail: eugeneai@irnok.net, Cherkashin Evgeny.
%
% New version of the correcting style file will be available at the website:
%     https://github.com/eugeneai/nla-style
%     file - nla.sty
%
% Further instructions are in the text body of the template. The template itself
% is an article example.
%
% The LaTeX2e format is used!

% 12 points font size is used.
\documentclass[12pt]{llncs}

% The correcting style file is added.
\usepackage{todonotes}

\usepackage{nla} % This package is needed for compiling
                 % this template, it should be removed
                 % from your article.

% Many popular packages (amsXXX, graphicx, etc.) are already imported in the style file.
% If there is a conflict with your packages, try disabling them and compile
% the text.
%
% It would be convenient in the layout of the proceedings if the file names
% of the figures of different authors do not clash.
% To minimize the clash, the drawings can be placed in a separate subfolder
% named after the author or the title of the paper.
%
% \graphicspath{{ivanov-petrov-pics/}} % specifies the folder with images in png, pdf formats.
% or
% \graphicspath{{great-problem-solving-paper-pics/}}.

\begin{document}

% Text should be formatted in accordance with the 'article' class, using extensions like
% AMS.
%
\fi
\title{Control Optimization in Systems \\with Phase Constraints	%\thanks{The research is supported by RFBR (RNF, other funds), project No.~00-00-00000.}
}
% First author
\author{Alexander Tyatyushkin%\inst{1}
%  \and
%  Alexander Gornov%\inst{2}
%  \and
%  Name FamilyName3\inst{1}
}
\institute{Matrosov Institute for System Dynamics and Control Theory of SB RAS, Irkutsk, Russia\\
  \email{tjat@icc.ru}
 % \and
%Affiliation, City, Country\\
%\email{email@example.com}
}
% etc

\maketitle

\begin{abstract}
The optimal control problem with phase constraints is considered. A method proposed for solving this problem is based on using a modified Lagrange function and a procedure for constraints linearization.

\keywords{Optimal control problem, phase constraints, Lagrange function}
\end{abstract}
% at the end of the list, there should be no final dot
%\section{The main results}
The paper proposes a numerical method for solving the optimal control problems with phase constraints. The gradients of the functionals are calculated using the Pontryagin function and solutions of the conjugate system with boundary conditions. To solve the considered problem, some initial approximation of the control and the corresponding phase trajectory are selected, and in its vicinity the linearization of terminal and phase constraints is carried out.

The main idea of the approach is to use the modified Lagrange function to construct a minimized functional and apply a procedure for linearization of terminal and phase constraints. Iterations of the proposed method solve the problem of minimizing the corresponding functional on the solutions of the original controlled system with linearized constraints on phase coordinates and controls. The results contain the values of Lagrange multipliers, the relationship between the values of which on neighboring iterations is established by the necessary optimality condition.


% At the end of the text, acknowledgments are expressed, if you haven't
% made a footnote from the title. For example, we can write
%The research is carried on with support of RFBR (RNF, other funds), project No.~00-00-00000.

%\begin{thebibliography}{9} % or {99}, if there is more than ten references.
%\bibitem{Pub1} Gornov A.Y., Zarodnyuk T.S., Madzhara T.I., Daneyeva A.V., Veyalko I.A. A collection of test multiextremal optimal control problems. Springer Optimization and Its Applications.~2013. Vol.~76. Pp.~257--274.

%\end{thebibliography}
%\end{document}

%%% Local Variables:
%%% mode: latex
%%% TeX-master: t
%%% End:
