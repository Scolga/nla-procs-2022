\iffalse

%%%%%%%%%%%%%%%%%%%%%%%%%%%%%%%%%%%%%%%%%%%%%%%%%%%%%%%%%%%%%%%%%%%%%%%%
%
% This is the template file for the 6th International conference
% NONLINEAR ANALYSIS AND EXTREMAL PROBLEMS
% June 25-30, 2018
% Irkutsk, Russia
%
%%%%%%%%%%%%%%%%%%%%%%%%%%%%%%%%%%%%%%%%%%%%%%%%%%%%%%%%%%%%%%%%%%%%%%%%
% The preparation of the article is based on the standard llncs class
% (Lecture Notes in Computer Sciences), which is adjusted with style
% file of the conference.
%
% There are two ways of compilation of the file into PDF
% 1. Use pdfLaTeX (pdflatex), (LaTeX+DVIPS will not work);
% 2. Use LuaLaTeX (XeLaTeX will work too).
% When using LuaLaTeX You will need TTF or OTF CMU fonts
% (Computer Modern Unicode). The fonts are installed with 'cm-unicode' package in
% a distribution of LaTeX % (https://www.ctan.org/tex-archive/fonts/cm-unicode),
% either by downloading and installing these fonts system wide, the address of their page is
% http://canopus.iacp.dvo.ru/%7Epanov/cm-unicode/
% The second option won't work in XeLaTeX.
%
% For MiKTeX (LaTeX distribution for Windows),
%  1. Package 'cm-unicode' is installed manually with the MiKTeX administration Console.
%  2. For the compilation of this example, namely, the stub figure, one will also need to
% download package 'pgf' manually. This package uses in the popular
% package tikz.
%  3. Tests showed that the rest of the required packages MiKTeX loads automatically (if
%     it is allowed). The 'auto download' option is
%     configured in 'Settings' section in MiKTeX Console.
%
%
% The easiest way to compile an article is to use pdfLaTeX, but
% the final layout of the book will be compiled with LuaLaTeX,
% as a result will be of better quality thanks to the package 'microtype' and
% use vector OTF instead of standard raster fonts of pdfLaTeX.
%
% In the case of questions and problems with the article compilation,
% write letters to e-mail: eugeneai@irnok.net, Cherkashin Evgeny.
%
% New version of the correcting style file will be available at the website:
%     https://github.com/eugeneai/nla-style
%     file - nla.sty
%
% Further instructions are in the text body of the template. The template itself
% is an article example.
%
% The LaTeX2e format is used!

% 12 points font size is used.
\documentclass[12pt]{llncs}

% The correcting style file is added.
\usepackage{todonotes}

\usepackage{nla} % This package is needed for compiling
                 % this template, it should be removed
                 % from your article.

% Many popular packages (amsXXX, graphicx, etc.) are already imported in the style file.
% If there is a conflict with your packages, try disabling them and compile
% the text.
%
% It would be convenient in the layout of the proceedings if the file names
% of the figures of different authors do not clash.
% To minimize the clash, the drawings can be placed in a separate subfolder
% named after the author or the title of the paper.
%
% \graphicspath{{ivanov-petrov-pics/}} % specifies the folder with images in png, pdf formats.
% or
% \graphicspath{{great-problem-solving-paper-pics/}}.

\begin{document}

% Text should be formatted in accordance with the 'article' class, using extensions like
% AMS.
%

\fi 

\title{On Generalized Solutions of the Second Boundary Value Problem for Differential-difference Equations with Variable Coefficients\thanks{The research is supported by  the Ministry of Science and Higher Education of the Russian Federation: agreement no. 075-03-2020-223/3 (FSSF-2020-0018)}}
% First author
\author{Nikita O. Ivanov}
\institute{RUDN University, Moscow, Russia\\
  \email{noivanov1@gmail.com}}
% etc


\maketitle

\begin{abstract}
The second boundary value problem for a second order differential-difference equation with variable coefficients is considered. The question of the existence of a generalized solution is investigated. The conditions for the right-hand side of the equation that ensure smoothness of generalized solutions on the whole interval $(0, d)$, $d \notin \mathbb{N}$ are obtained.

\keywords{difference-differential equation, second boundary value problem, generalized solutions}
\end{abstract}

% at the end of the list, there should be no final dot
\section{The main results}

We consider the second boundary value problem for the second order differential-difference equation with variable coefficients on the finite interval of non-integer length. We assume that the Hermitian part of the difference operator is a positive definite operator. We have proved that the corresponding differential-difference operator is Fredholm operator. It is shown that the smoothness of generalized solutions holds on subintervals. The specified subintervals obtained by deleting the orbits for the ends of the interval $(0, d)$, $d =N+\theta$, $0<\theta<1$, generated by a group of integer shifts. On the other hand, smoothness of generalized solutions of the boundary value problem can be violated at the points of the mentioned orbits. It is proved that the smoothness of generalized solutions on the whole interval $(0, d)$, $d \notin \mathbb{N}$,  is preserved if the condition of orthogonality of the right-hand side of the differential-difference equation to a finite number of linearly independent functions is fulfilled.

\par For the first boundary value problem similar results were obtained in \cite{ivan_1}.
% At the end of the text, acknowledgments are expressed, if you haven't
% made a footnote from the title. For example, we can write
\bigskip
\par This is a joint work with Alexander L. Skubachevskii
\bigskip
\par This work is supported by  the Ministry of Science and Higher Education of the Russian Federation: agreement no. 075-03-2020-223/3 (FSSF-2020-0018).

\begin{thebibliography}{4} % or {99}, if there is more than ten references.
\bibitem{ivan_1} {\it Skubachevskii~A.\,L.} Elliptic Functional
Differential Equations and Applications.    Basel--Boston--Berlin:
Birkh\"auser, 1997.   298 p.

\end{thebibliography}
%\end{document}

%%% Local Variables:
%%% mode: latex
%%% TeX-master: t
%%% End:
