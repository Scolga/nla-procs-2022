\begin{englishtitle} % Настраивает LaTeX на использование английского языка
% Этот титульный лист верстается аналогично.
\title{On the Solvability of a Nonlocal Boundary Value Problem for Fractional Differential Inclusions with Causal Multioperators\thanks{Исследование выполнено при финансовой поддержке РФФИ в рамках научного проекта \textnumero~19-31-60011 и гранта Президента РФ для государственной поддержки молодых российских ученых – кандидатов наук, проект МК-338.2021.1.1.}}
% First author
\author{V. Obukhovskii\inst{1}
  \and
  G. Petrosyan\inst{1,2}
  \and
  M. Soroka\inst{1}
}
\institute{Voronezh State Pedagogical University,  Voronezh, Russia\\
  \email{valerio-ob2000@mail.ru, marya.afanasowa@yandex.ru}
  \and
Voronezh State University of Engineering Technologies, Voronezh, Russia\\
\email{garikpetrosyan@yandex.ru}}
% etc

\maketitle

\begin{abstract}
We consider the solvability of a nonlocal boundary value problem for fractional differential inclusions  with causal multioperators based on the topological degree theory for multivalued condensing maps.

\keywords{fractional derivative, differential inclusion, boundary value problem, condensing multioperator, measure of noncompactness, causal multioperator} % в конце списка точка не ставится
\end{abstract}
\end{englishtitle}

\iffalse
%%%%%%%%%%%%%%%%%%%%%%%%%%%%%%%%%%%%%%%%%%%%%%%%%%%%%%%%%%%%%%%%%%%%%%%%
%
%  This is the template file for the 6th International conference
%  NONLINEAR ANALYSIS AND EXTREMAL PROBLEMS
%  June 25-30, 2018
%  Irkutsk, Russia
%
%%%%%%%%%%%%%%%%%%%%%%%%%%%%%%%%%%%%%%%%%%%%%%%%%%%%%%%%%%%%%%%%%%%%%%%%

%  Верстка статьи осуществляется на основе стандартного класса llncs
%  (Lecture Notes in Computer Sciences), который корректируется стилевым
%  файлом конференции.
%
%  Скомпилировать файл в PDF можно двумя способами:
%  1. Использовать pdfLaTeX (pdflatex), (LaTeX+DVIPS не работает);
%  2. Использовать LuaLaTeX (XeLaTeX будет работать тоже).
%  При использовании LuaLaTeX потребуются TTF- или OTF-шрифты CMU
%  (Computer Modern Unicode). Шрифты устанавливаются либо пакетом
%  дистрибутива LaTeX cm-unicode
%              (https://www.ctan.org/tex-archive/fonts/cm-unicode),
%  либо загрузкой и установкой в операционной системе, адрес страницы:
%              http://canopus.iacp.dvo.ru/%7Epanov/cm-unicode/
%  Второй вариант не будет работать в XeLaTeX.
%
%  В MiKTeX (дистрибутив LaTeX для ОС Windows):
%  1. Пакет cm-unicode устанавливается вручную в программе MiKTeX Console.
%  2. Для верстки данного примера, а именно, картинки-заглушки необходимо,
%     также вручную, загрузить пакет pgf. Этот пакет используется популярным
%     пакетом tikz.
%  3. Тест показал, что остальные пакеты MiKTeX грузит автоматически (если
%     ему разрешено автоматически грузить пакеты). Режим автозагрузки
%     настраивается в разделе Settings в MiKTeX Console.
%
%
%  Самый простой способ сверстать статью - использовать pdfLaTeX, но
%  окончательный вариант верстки сборника будет собран в LuaLaTeX,
%  так как результат получится лучшего качества, благодаря пакету microtype и
%  использованию векторных шрифтов OTF вместо растровых pdfLaTeX.
%
%  В случае возникновения вопросов и проблем с версткой статьи,
%  пишите письма на электронную почту: eugeneai@irnok.net, Черкашин Евгений.
%
%  Новые варианты корректирующего стиля будут доступны на сайте:
%        https://github.com/eugeneai/nla-style
%        файл - nla.sty
%
%  Дальнейшие инструкции - в тексте данного шаблона. Он одновременно
%  является примером статьи.
%
%  Формат LaTeX2e!

\documentclass[12pt]{llncs}  % Необходимо использовать шрифт 12 пунктов.

% При использовании pdfLaTeX добавляется стандартный набор русификации babel.
% Если верстка производится в LuaLaTeX, то следующие три строки надо
% закомментировать, русификация будет произведена в корректирующем стиле автоматом.
\usepackage{iftex}

\ifPDFTeX
\usepackage[T2A]{fontenc}
\usepackage[utf8]{inputenc} % Кодировка utf-8, cp1251 и т.д.
\usepackage[english,russian]{babel}
\fi

% Для верстки в LuaLaTeX текст готовится строго в utf-8!

% В операционной системе Windows для редактирования в кодировке utf-8
% можно использовать программы notepad++ https://notepad-plus-plus.org/,
% techniccenter http://www.texniccenter.org/,
% SciTE (самая маленькая по объему программа) http://www.scintilla.org/SciTEDownload.html
% Подойдет также и встроенный в свежий дистрибутив MiKTeX редактор
% TeXworks.

% Добавляется корректирующий стилевой файл строго после babel, если он был включен.
% В параметре необходимо указать russian, что установит не совсем стандартные названия
% разделов текста, настроит переносы для русского языка как основного и т.п.

\usepackage{todonotes} % Этот пакет нужен для верстки данного шаблона, его
                       % надо убрать из вашей статьи.

\usepackage[russian]{nla}

% Многие популярные пакеты (amsXXX, graphicx и т.д.) уже импортированы в корректирующий стиль.
% Если возникнут конфликты с вашими пакетами, попробуйте их отключить и сверстать
% текст как есть.
%
%


% Было б удобно при верстке сборника, чтобы названия рисунков разных авторов не пересекались.
% Чтоб минимизировать такое пересечение, рисунки можно поместить в отдельную подпапку с
% названием - фамилией автора или названием статьи.
%
% \graphicspath{{ivanov-petrov-pics/}} % Указание папки с изображениями в форматах png, pdf.
% или
% \graphicspath{{great-problem-solving-paper-pics/}}.


\begin{document}

% Текст оформляется в соответствии с классом article, используя дополнения
% AMS.
%
\fi

\title{О разрешимости нелокальной краевой задачи для дифференциальных включений дробного порядка с каузальными мультиоператорами}
% Первый автор
\author{В.~В.~Обуховский\inst{1}  % \inst ставит циферку над автором.
  \and  % разделяет авторов, в тексте выглядит как запятая.
% Второй автор
  Г.~Г.~Петросян\inst{1,2}
  \and
% Третий автор
  М.~С.~Сорока\inst{1}
} % обязательное поле

% Аффилиации пишутся в следующей форме, соединяя каждый институт при помощи \and.
\institute{Воронежский государственный педагогический университет (ВГПУ), Воронеж, Россия \\
  \email{valerio-ob2000@mail.ru, marya.afanasowa@yandex.ru}
  \and   % Разделяет институты и присваивает им номера по порядку.
Воронежский государственный университет инженерных технологий (ВГУИТ), Воронеж, Россия\\
  \email{garikpetrosyan@yandex.ru}
% \and Другие авторы...
}

\maketitle

\begin{abstract}
Рассматривается разрешимость нелокальной краевой задачи для дифференциальных включений дробного порядка с каузальными мультиоператорами на основе теории топологической степени для многозначных уплотняющих отображений.

\keywords{дробная производная, дифференциальное включение, краевая задача, уплотняющий мультиоператор, мера некомпактности, каузальный мультиоператор} % в конце списка точка не ставится
\end{abstract}

\section{Основные результаты} % не обязательное поле

В сепарабельном банаховом пространстве $E$ мы рассматриваем нелокальную краевую задачу для дифференциальных включений дробного порядка следующего вида:
$$
^{C}D^{q}_{0}[x(t)-k(t,x_t)]\in Ax(t)+Q(x)(t), \, t\in [0,T],
$$
$$
x(s)+g(x)(s)=\upsilon(s), \, s\in (-\infty,0],
$$
где $^{C}D^{q}_{0}, 0<q<1,$ -- дробная производная Капуто, $A: D(A)\subset E\to E$ -- линейный замкнутый оператор, порождающий ограниченную $C_0$-полугруппу. Обозначим через $C=C([0,T];E)$ пространство всех непрерывных функций на отрезке $[0,T],$ $\mathcal{B}$ -- фазовое пространство бесконечных запаздываний Хейла--Като и будем считать $\mathcal{C}=\mathcal{C}((-\infty,T];E)$ нормированным пространством функций, сужение которых на $(-\infty,0]$ принадлежит $\mathcal{B},$ а сужение на $[0,T]$ принадлежит $C.$ Мы полагаем, что $Q:\mathcal{C} \Rightarrow L^p ([0,T];E)$ -- многозначный мультиоператор, $k:[0,T]\times \mathcal{B} \to E$ и $g: \mathcal{C} \to \mathcal{B},$ $x_t\in \mathcal{B}$ -- предыстория функции до момента $t$ и $\upsilon \in \mathcal{B}$ -- заданная функция.

Для разрешения задачи мы в пространстве $\mathcal{C}$ конструируем разрешающий многозначный интегральный оператор и сводим исходную задачу к задаче о существовании неподвижных точек построенного мультиоператора. Для доказательства существования неподвижных точек используется теория мер некомпактности и уплотняющих многозначных отображений.









 

\begin{thebibliography}{9} % или {99}, если ссылок больше десяти.

\bibitem{pg 1} Афанасова~М.С., Обуховский~В.В., Петросян~Г.Г. Об обобщенной краевой задаче для управляемой системы с обратной связью и бесконечным запаздыванием~// Вестник Удмуртского университета. Математика. Механика. Компьютерные науки. 2021. Т.~31, №~2. С.~167--185.

\bibitem{pg 2} Ахмеров~Р.Р., Каменский~М.И.  Ко второй теореме Н.Н. Боголюбова в принципе усреднения для функционально-дифференциальных уравнений нейтрального типа~// Дифференциальные уравнения. 1974. Т.~10, №~3. С.~537--540.

\bibitem{pg 3} Каменский~М.И., Макаренков~О.Ю., Нистри~П. Об одном подходе в теории обыкновенных дифференциальных уравнений с малым параметром~// Доклады Академии наук. 2003. Т.~388, №~4. С.~439--442.

\bibitem{pg 4} Каменский~М.И., Пенкин~О.М., Покорный Ю.В. О полугруппе в задаче диффузии на пространственной сети~// Доклады Академии наук. 1999. Т.~368, №~2. С.~157--159.


\bibitem{pg 5} Gurova~I.N., Kamenskii~M.I. On the method of semidiscre\-ti\-za\-tion in the problem on periodic solutions to quasilinear autonomous parabolic equations~// Differential Equations. 1996. Vol.~32. Pp.~106--112.


\bibitem{pg 6} Johnson~R., Nistri~P., Kamenskii~M. On periodic solutions of a damped wave equation in a thin domain using degree theoretic methods~// Journal of Diffe\-ren\-tial Equations. 1997. Vol.~140. Pp.~186--208.

\bibitem{pg 7} Kamenskii~M., Makarenkov~O., Nistri~P. An alternative approa\-ch to study bifurcation from a limit cycle in periodically perturbed autonomous systems~// Journal of Dynamics and Diffe\-ren\-tial Equations. 2011. Vol.~23. Pp.~425--435.


\bibitem{pg 8} Kamenskii~M.I., Obukhovskii~V.V. Condensing multi\-ope\-ra\-tors and periodic solutions of parabolic functional-differe\-ntial inc\-lu\-sions in Banach spaces~// Nonlinear Analysis. 1993. Vol.~20. Pp.~781--792.

\bibitem{pg 9} Kamenskii~M.I., Nistri~P., Zecca~P., Obukhovskii V.V. Optimal feedback control for a semilinear evolution equation~// Journal of Optimi\-za\-tion Theory and Applications. 1994. Vol.~82. Pp.~503--517.


\end{thebibliography}

% После библиографического списка в русскоязычных статьях необходимо оформить
% англоязычный заголовок.



%\end{document}

%%% Local Variables:
%%% mode: latex
%%% TeX-master: t
%%% End:
