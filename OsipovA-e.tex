\iffalse
%%%%%%%%%%%%%%%%%%%%%%%%%%%%%%%%%%%%%%%%%%%%%%%%%%%%%%%%%%%%%%%%%%%%%%%%
%
% This is the template file for the 6th International conference
% NONLINEAR ANALYSIS AND EXTREMAL PROBLEMS
% June 25-30, 2018
% Irkutsk, Russia
%
%%%%%%%%%%%%%%%%%%%%%%%%%%%%%%%%%%%%%%%%%%%%%%%%%%%%%%%%%%%%%%%%%%%%%%%%
% The preparation of the article is based on the standard llncs class
% (Lecture Notes in Computer Sciences), which is adjusted with style
% file of the conference.
%
% There are two ways of compilation of the file into PDF
% 1. Use pdfLaTeX (pdflatex), (LaTeX+DVIPS will not work);
% 2. Use LuaLaTeX (XeLaTeX will work too).
% When using LuaLaTeX You will need TTF or OTF CMU fonts
% (Computer Modern Unicode). The fonts are installed with 'cm-unicode' package in
% a distribution of LaTeX % (https://www.ctan.org/tex-archive/fonts/cm-unicode),
% either by downloading and installing these fonts system wide, the address of their page is
% http://canopus.iacp.dvo.ru/%7Epanov/cm-unicode/
% The second option won't work in XeLaTeX.
%
% For MiKTeX (LaTeX distribution for Windows),
%  1. Package 'cm-unicode' is installed manually with the MiKTeX administration Console.
%  2. For the compilation of this example, namely, the stub figure, one will also need to
% download package 'pgf' manually. This package uses in the popular
% package tikz.
%  3. Tests showed that the rest of the required packages MiKTeX loads automatically (if
%     it is allowed). The 'auto download' option is
%     configured in 'Settings' section in MiKTeX Console.
%
%
% The easiest way to compile an article is to use pdfLaTeX, but
% the final layout of the book will be compiled with LuaLaTeX,
% as a result will be of better quality thanks to the package 'microtype' and
% use vector OTF instead of standard raster fonts of pdfLaTeX.
%
% In the case of questions and problems with the article compilation,
% write letters to e-mail: eugeneai@irnok.net, Cherkashin Evgeny.
%
% New version of the correcting style file will be available at the website:
%     https://github.com/eugeneai/nla-style
%     file - nla.sty
%
% Further instructions are in the text body of the template. The template itself
% is an article example.
%
% The LaTeX2e format is used!

% 12 points font size is used.
\documentclass[12pt]{llncs}

% The correcting style file is added.
\usepackage{todonotes}

\usepackage{nla} % This package is needed for compiling
                 % this template, it should be removed
                 % from your article.

% Many popular packages (amsXXX, graphicx, etc.) are already imported in the style file.
% If there is a conflict with your packages, try disabling them and compile
% the text.
%
% It would be convenient in the layout of the proceedings if the file names
% of the figures of different authors do not clash.
% To minimize the clash, the drawings can be placed in a separate subfolder
% named after the author or the title of the paper.
%
% \graphicspath{{ivanov-petrov-pics/}} % specifies the folder with images in png, pdf formats.
% or
% \graphicspath{{great-problem-solving-paper-pics/}}.

\begin{document}

% Text should be formatted in accordance with the 'article' class, using extensions like
% AMS.
%
\fi

\title{On an Inverse Spectral Problem for Band Operators and Nnonlinear Lattices\thanks{This work is done at SRISA according to the project No
0580-2021-007 (Reg. No 121031300051-3).}}
% First author
\author{Andrey Osipov 
}
\institute{Federal State Institution \\
``Scientific-Research Institute for System Analysis\\
of the Russian Academy of Sciences", Moscow,  Russia\\
  \email{osipa68@yahoo.com}
}
% etc

\maketitle

\begin{abstract}
We study some links between an inverse spectral problem for
certain classes of operators generated by (possibly infinite) band
matrices and nonlinear dynamical systems such as e. g. Volterra
and Toda lattices. \keywords{Band operators, Inverse spectral
problems, Nonlinear lattices.}
\end{abstract}

% at the end of the list, there should be no final dot
\section{The main results}
As known, the inverse spectral problem method for differential and
difference operators consists in reconstructing of a given
operator from its spectral characteristics. Since the works of
Kac, van Moerbeke and Moser the inverse spectral problems for
operators generated by infinite band matrices (called band
operators; in particular, the Jacobi operators belong to this
class) have been widely applied to the integration via Lax pair
formalism of certain nonlinear dynamical systems (nonlinear
lattices). Here we consider a version of the inverse spectral
problem method for band operators, where a key role is played by
the moments of the Weyl matrix (or Weyl function) of a given band
operator which are used for unique reconstruction of the latter,
see [1]-[3] for details.

Out of the nonlinear lattices, the most well studied ones are the
Toda lattice and the Volterra lattice (also known as Kac - van
Moerbeke system, Langmuir chain or discrete KdV equation). In
[1]-[2] we found that a discrete Miura transformation which
relates these two systems to each other can be easily described in
terms of the above mentioned moments (in particular, such
description allows one to establish a bijection between the
semi-infinite Volterra lattices and the semi-infinite Toda
lattices characterized by positivity of Jacobi operators in their
Lax representation). Also, in the finite lattice case, due to the
special structure of the moments corresponding to a finite Jacobi
operator, some non-trivial first integrals for both Volterra and
Toda lattices were found.

Here we discuss similar issues for the Bogoyavlensky lattices
BL1({\it p}):
\begin{equation*}
\overset \cdot
a_i=a_i\left(\sum_{j=1}^{p}a_{i+j}-\sum_{j=1}^{p}a_{i-j}\right );
\end{equation*}
and BL2({\it p}):
\begin{equation*}
\overset \cdot
b_i=b_i\left(\prod_{j=1}^{p}b_{i+j}-\prod_{j=1}^{p}b_{i-j}\right
);
\end{equation*}
for some fixed $p\ge 1$, [4] (for $p=1$ both BL1({\it p}) and BL2({\it p}) coincide with the Volterra lattice).

For the lattice BL3({\it p}):
\begin{equation*}
\overset \cdot
c_i=c_i^2\left(\sum_{j=1}^{p}c_{i+j}-\sum_{j=1}^{p}c_{i-j}\right
);
\end{equation*}
the band operator which appears in its Lax representation has a structure which is different from the cases of BL1({\it
p}) and  BL2({\it
p}), so our inverse problem method needs to be
modified before it can be applied to the integration of BL3({\it
p}). This modification as well as the whole integration procedure
of the finite lattices BL3({\it p}) are considered for the case
$p=1$.

\begin{thebibliography}{9} % or {99}, if there is more than ten references.
\bibitem{os1}  Osipov A. S. Inverse Spectral Problems for
Second-Order Difference Operators and Their Application to the
Study of Volterra Type Systems. Rus. J. Nonlin. Dyn. 2020. Vol.~
16. no~3. Pp.~397--419.
\bibitem{os2} A. Osipov. Inverse spectral problem for Jacobi operators
and Miura transformation. Concr. Oper. 2021. Vol.~8. Pp.~77--89.
\bibitem{os3}  A. S. Osipov. Inverse spectral problem for band
operators and their sparsity criterion in terms of inverse problem
data. Russ. J. Math. Phys. 2022. Vol.~29. no.~2.
\bibitem{sur}Yu.B. Suris. The Problem of Integrable Discretization:
Hamiltonian Approach. Progress in Mathematics, vol. 219.
Birkh\"auser, Basel, 2003.

\end{thebibliography}
%\end{document}

%%% Local Variables:
%%% mode: latex
%%% TeX-master: t
%%% End:
