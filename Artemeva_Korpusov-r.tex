\begin{englishtitle} % Настраивает LaTeX на использование английского языка
% Этот титульный лист верстается аналогично.
\title{Blow up of Solutions and Local Solvability of an Abstract Cauchy Problem for Second-order Differential Equation with a Non-coercive Source}
% First author
\author{M.~V.~Artemeva\inst{1}
  \and
  M.~O.~Korpusov\inst{2}
%  \and
%  Name FamilyName3\inst{1}
}
\institute{Lomonosov Moscow State University, Moscow, Russia\\
  \email{artemeva.mv14@physics.msu.ru}
  \and
Lomonosov Moscow State University, Moscow, Russia\\
\email{korpusov@gmail.com}}
% etc

\maketitle

\begin{abstract}
An abstract Cauchy problem for a second-order differential equation with nonlinear operator coefficients and a non-coercive source is considers. Local solvability is proved and sufficient conditions for the blow up of solutions in finite time are obtained. Estimates from above and below for the blow up time are found and sufficient conditions for global solvability in time are obtained.

\keywords{nonlinear Sobolev type equations, blow-up, local solvability, nonlinear capacity,  blow up time estimates} % в конце списка точка не ставится
\end{abstract}
\end{englishtitle}

\iffalse
%%%%%%%%%%%%%%%%%%%%%%%%%%%%%%%%%%%%%%%%%%%%%%%%%%%%%%%%%%%%%%%%%%%%%%%%
%
%  This is the template file for the 6th International conference
%  NONLINEAR ANALYSIS AND EXTREMAL PROBLEMS
%  June 25-30, 2018
%  Irkutsk, Russia
%
%%%%%%%%%%%%%%%%%%%%%%%%%%%%%%%%%%%%%%%%%%%%%%%%%%%%%%%%%%%%%%%%%%%%%%%%

%  Верстка статьи осуществляется на основе стандартного класса llncs
%  (Lecture Notes in Computer Sciences), который корректируется стилевым
%  файлом конференции.
%
%  Скомпилировать файл в PDF можно двумя способами:
%  1. Использовать pdfLaTeX (pdflatex), (LaTeX+DVIPS не работает);
%  2. Использовать LuaLaTeX (XeLaTeX будет работать тоже).
%  При использовании LuaLaTeX потребуются TTF- или OTF-шрифты CMU
%  (Computer Modern Unicode). Шрифты устанавливаются либо пакетом
%  дистрибутива LaTeX cm-unicode
%              (https://www.ctan.org/tex-archive/fonts/cm-unicode),
%  либо загрузкой и установкой в операционной системе, адрес страницы:
%              http://canopus.iacp.dvo.ru/%7Epanov/cm-unicode/
%  Второй вариант не будет работать в XeLaTeX.
%
%  В MiKTeX (дистрибутив LaTeX для ОС Windows):
%  1. Пакет cm-unicode устанавливается вручную в программе MiKTeX Console.
%  2. Для верстки данного примера, а именно, картинки-заглушки необходимо,
%     также вручную, загрузить пакет pgf. Этот пакет используется популярным
%     пакетом tikz.
%  3. Тест показал, что остальные пакеты MiKTeX грузит автоматически (если
%     ему разрешено автоматически грузить пакеты). Режим автозагрузки
%     настраивается в разделе Settings в MiKTeX Console.
%
%
%  Самый простой способ сверстать статью - использовать pdfLaTeX, но
%  окончательный вариант верстки сборника будет собран в LuaLaTeX,
%  так как результат получится лучшего качества, благодаря пакету microtype и
%  использованию векторных шрифтов OTF вместо растровых pdfLaTeX.
%
%  В случае возникновения вопросов и проблем с версткой статьи,
%  пишите письма на электронную почту: eugeneai@irnok.net, Черкашин Евгений.
%
%  Новые варианты корректирующего стиля будут доступны на сайте:
%        https://github.com/eugeneai/nla-style
%        файл - nla.sty
%
%  Дальнейшие инструкции - в тексте данного шаблона. Он одновременно
%  является примером статьи.
%
%  Формат LaTeX2e!

\documentclass[12pt]{llncs}  % Необходимо использовать шрифт 12 пунктов.

% При использовании pdfLaTeX добавляется стандартный набор русификации babel.
% Если верстка производится в LuaLaTeX, то следующие три строки надо
% закомментировать, русификация будет произведена в корректирующем стиле автоматом.
\usepackage{iftex}

\ifPDFTeX
\usepackage[T2A]{fontenc}
\usepackage[utf8]{inputenc} % Кодировка utf-8, cp1251 и т.д.
\usepackage[english,russian]{babel}
\fi

% Для верстки в LuaLaTeX текст готовится строго в utf-8!

% В операционной системе Windows для редактирования в кодировке utf-8
% можно использовать программы notepad++ https://notepad-plus-plus.org/,
% techniccenter http://www.texniccenter.org/,
% SciTE (самая маленькая по объему программа) http://www.scintilla.org/SciTEDownload.html
% Подойдет также и встроенный в свежий дистрибутив MiKTeX редактор
% TeXworks.

% Добавляется корректирующий стилевой файл строго после babel, если он был включен.
% В параметре необходимо указать russian, что установит не совсем стандартные названия
% разделов текста, настроит переносы для русского языка как основного и т.п.

\usepackage{todonotes} % Этот пакет нужен для верстки данного шаблона, его
                       % надо убрать из вашей статьи.

\usepackage[russian]{nla}

% Многие популярные пакеты (amsXXX, graphicx и т.д.) уже импортированы в корректирующий стиль.
% Если возникнут конфликты с вашими пакетами, попробуйте их отключить и сверстать
% текст как есть.
%
%


% Было б удобно при верстке сборника, чтобы названия рисунков разных авторов не пересекались.
% Чтоб минимизировать такое пересечение, рисунки можно поместить в отдельную подпапку с
% названием - фамилией автора или названием статьи.
%
% \graphicspath{{ivanov-petrov-pics/}} % Указание папки с изображениями в форматах png, pdf.
% или
% \graphicspath{{great-problem-solving-paper-pics/}}.


\begin{document}

% Текст оформляется в соответствии с классом article, используя дополнения
% AMS.
%
\fi

\title{Разрушение решений и локальная разрешимость абстрактной задачи Коши для дифференциального уравнения второго порядка с некоэрцитивным источником}
% Первый автор
\author{М.~В.~Артемьева\inst{1}  % \inst ставит циферку над автором.
  \and  % разделяет авторов, в тексте выглядит как запятая.
% Второй автор
  М.~О.~Корпусов \inst{2}
  \and
% Третий автор
%  И.~О.~Фамилия\inst{1}
} % обязательное поле

% Аффилиации пишутся в следующей форме, соединяя каждый институт при помощи \and.
\institute{МГУ им. М.~В.~Ломоносова, физический факультет, Москва, Россия \\
  \email{artemeva.mv14@physics.msu.ru}
  \and   % Разделяет институты и присваивает им номера по порядку.
%Институт (название в краткой форме), Город, Страна\\
МГУ им. М.~В.~Ломоносова, физический факультет, Москва, Россия \\
  \email{korpusov@gmail.com}
% \and Другие авторы...
}

\maketitle

\begin{abstract}
В работе рассматривается одна абстрактная задача Коши для дифференциального уравнения второго порядка с нелинейными операторными коэффициентами и некоэрцитивным источником. Доказана локальная разрешимость и получены достаточные условия разрушения решений за конечное время. Найдены оценки сверху и снизу на время разрушения и получены достаточные условия глобальной во времени разрешимости.

\keywords{нелинейные уравнения соболевского типа, разрушение, blow-up, локальная разрешимость, нелинейная емкость, оценки времени разрушения} % в конце списка точка не ставится
\end{abstract}

%\section{Основные результаты} % не обязательное поле

Математические модели реальных физических процессов в нелинейных электромагнитных средах сводятся к изучению абстрактных дифференциальных уравнений с нелинейными операторными коэффициентами и некоэрцитивными источниками~\cite{korpusov3,rab1}. Такие задачи стали активно исследоваться начиная с 1970-х годов~\cite{levine2,Straughan,Kalantar}. При этом существенный интерес представляют вопросы разрушения указанных задач за конечное время. С точки зрения физики эффект разрушения есть процесс электрического пробоя в электромагнитной среде.	

Рассмотрим абстрактную задачу Коши для дифференциального уравнения второго порядка следующего вида:
\begin{equation}\label{levine-eq-vved-10-2}
	\dfrac{d^2}{d t^2}\left(A_0u+\sum\limits_{j=1}^NA_j(u)\right)+\dfrac{d}{d t}DP(u)+Lu=\dfrac{d}{d t}F(u),\quad u(0)=u_0,\quad u'(0)=u_1
\end{equation}
где операторы $A_0$ и $L$ линейные, а операторы $A_j(u)$, $DP(u)$ и $F(u)$ нелинейные. Нелинейное слагаемое
$$
\dfrac{d}{d t}DP(u),
$$
в приложениях имеет следующий, например, вид:
$$
\dfrac{\partial^2 u^2(x,t)}{\partial t\partial x_1}.
$$

В этой работе доказывается существование непродолжаемого во времени классического решения задачи Коши  (\ref{levine-eq-vved-10-2}) при некоторых условиях на операторные коэффициенты. Для доказательства разрушения за конечное время используется модификация энергетического метода H.\,A.~Levine, изложенная в работе \cite{korpusov3}. Отметим, что уравнение (\ref{levine-eq-vved-10-2}) содержит некоэрцитивный источник
$$
\dfrac{d}{d t}F(u),
$$
что сильно усложняет получение достаточных условий разрушения задачи Коши (\ref{levine-eq-vved-10-2}) за конечное время. 

Оценки снизу и сверху на время разрушения, а также получим достаточные условия существования глобального решения задачи для произвольных начальных данных (не обязательно малых).

% Рисунки и таблицы оформляются по стандарту класса article. Например,

%\begin{figure}[htb]
%  \centering
  % Поддерживаются два формата:
  %\includegraphics[width=0.7\linewidth]{figure.pdf} % Растровый формат
  %\includegraphics[width=0.7\linewidth]{figure.png} % Векторный и растровый формат
  %
  % Векторные рисунки можно рисовать в редакторе Inkscape
  % https://inkscape.org/ru/download/
  % Основной формат этого редактора - SVG, поэтому рисунки необходимо экспортировать в
  % PDF или PNG (с разрешением - минимум 150 dpi, максимум - 300dpi).
%  \begin{center}
%    \missingfigure[figwidth=0.7\linewidth]{Уберите меня из статьи!}
%  \end{center}
%  \caption{Заголовок рисунка}\label{fig:example}
%\end{figure}

% Современные издательства требуют использовать кавычки-елочки << >>.

% В конце текста можно выразить благодарности, если этого не было
% сделано в ссылке с заголовка статьи, например,
%Работа выполнена при поддержке РФФИ (РНФ, другие фонды), проект \textnumero~00-00-00000.
%

% Список литературы оформляется подобно ГОСТ-2008.
% Примеры оформления находятся по этому адресу -
%     https://narfu.ru/agtu/www.agtu.ru/fad08f5ab5ca9486942a52596ba6582elit.html
%

\begin{thebibliography}{9} % или {99}, если ссылок больше десяти.
\bibitem{korpusov3}  Al'shin~A.B., Korpusov~M.O.,  Sveshnikov~A.G. Blow-up in nonlinear Sobolev type equations. De Gruyter Ser. Nonlinear Anal. Appl. 2011. Vol.~15. Pp.~648.

\bibitem{rab1}  Рабинович~М.И. Автоколебания распределенных систем~// Изв. вузов. Радиофизика. 1974. Т.~17. № 4. С.~477--510.		

\bibitem{levine2}   Levine~H.A. Instability and nonexistence of global solutions to nonlinear wave equations of the form $Pu_{tt}=-Au+F(u)$. Trans. Amer. Math. Soc. 1973. vol.~51. pp.~371--386.

\bibitem{Straughan} Straughan~B. Further global nonexistence theorems for abstract nonlinear wave equations. Proc. Amer. Math. Soc. 1975. Vol.~48. Issue 2. Pp.~381--390.

\bibitem{Kalantar} Калантаров~В.К., Ладыженская~О.А. О возникновении коллапсов для квазилинейных уравнений параболического и гиперболического типов~// Зап. научн. сем. ЛОМИ. 1977. Т.~69. №10. стр.~77--102.


%\bibitem{Gantmakher} Гантмахер~Ф.Р. Теория матриц. М.:~Наука,~1966.
%
%\bibitem{Kholl} Современные численные методы решения обыкновенных дифференциальных уравнений~/ Под~ред.~Дж.~Холл, Дж.~Уатт. М.:~Мир,~1979.
%
%\bibitem{Aleksandrov1} Александров~А.Ю. Об устойчивости сложных систем в критических случаях~// Автоматика и телемеханика. 2001. \textnumero~9. С.~3--13.
%
%\bibitem{Moreau1977} Moreau~J.-J. Evolution problem associated with a moving convex set in a Hilbert space~// J.~Differential~Eq. 1977. Vol.~26. Pp.~347--374.
%
%\bibitem{Semenov} Семенов~А.А. Замечание о вычислительной сложности известных предположительно односторонних функций~// Тр.~XII Байкальской междунар. конф. <<Методы оптимизации и их приложения>>. Иркутск, 2001. С.~142--146.

\end{thebibliography}

% После библиографического списка в русскоязычных статьях необходимо оформить
% англоязычный заголовок.




%\end{document}

%%% Local Variables:
%%% mode: latex
%%% TeX-master: t
%%% End:
