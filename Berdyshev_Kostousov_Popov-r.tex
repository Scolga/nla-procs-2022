\begin{englishtitle} % Настраивает LaTeX на использование английского языка
% Этот титульный лист верстается аналогично.
\title{Optimal Object Trajectories under Unfriendly Observation\thanks{Работа выполнена в рамках исследований, проводимых в Уральском математическом центре при финансовой поддержке Министерства науки и высшего образования Российской Федерации (номер соглашения \textnumero~075-02-2022-874). }}
% First author
\author{V.I.~Berdyshev 
  \and
  V.B.~Kostousov 
  \and
  A.A.~Popov 
}
\institute{IMM UrB RAS, Ekaterinburg, Russia\\
  \email{bvi@imm.uran.ru,  vkost@imm.uran.ru, aap@imm.uran.ru}
%  \and
%Affiliation, City, Country\\
%\email{email@example.com}}
% etc
}

\maketitle

\begin{abstract}
In applied problems associated with the construction of trajectories of autonomous moving objects, sometimes there are requirements for the maximum distance of the trajectory from undesirable stationary objects-observers.  Often observers have a means of observation that allows you to capture moving objects in a certain cone of observation, but outside the cone the observer does not see the object.
Similar situations lead to statement of the problems of constructing optimal trajectories in a given corridor $ Y\subset X $.
The report considers the problem of forming of such trajectory that maximizes the minimum distance from the observers in their field of view. The problem is investigated in the space $X=\mathbb{R}^n, (n=2,3)$.
For a point object, an optimal corridor~$Y^* \subset Y$ is constructed. It is such a connected set that any continuous trajectory that passes in this corridor and connects the given start and end points is optimal.
A similar problem is also solved on the plane for the case when the moving object is a solid, namely it is a circle \cite{Berdyshev}.
For practical calculations, the report proposes algorithms for constructing an optimal corridor and the shortest optimal trajectory for a solid object on a plane.
For these purposes the initial continuous conditions of the problem, such as the boundaries of the bounding corridor and observation cones, are projected onto a discrete regular grid.  After that a discrete implementation of the optimal corridor is constructed, its boundaries is built on the grid, and the shortest optimal trajectory of the solid object is found using the Dijkstra algorithm.

\keywords{route planning in the presence of obstacles, optimal trajectory, observers, shortest path} % в конце списка точка не ставится
\end{abstract}
\end{englishtitle}


\iffalse
%%%%%%%%%%%%%%%%%%%%%%%%%%%%%%%%%%%%%%%%%%%%%%%%%%%%%%%%%%%%%%%%%%%%%%%%
%
%  This is the template file for the 6th International conference
%  NONLINEAR ANALYSIS AND EXTREMAL PROBLEMS
%  June 25-30, 2018
%  Irkutsk, Russia
%
%%%%%%%%%%%%%%%%%%%%%%%%%%%%%%%%%%%%%%%%%%%%%%%%%%%%%%%%%%%%%%%%%%%%%%%%

%  Верстка статьи осуществляется на основе стандартного класса llncs
%  (Lecture Notes in Computer Sciences), который корректируется стилевым
%  файлом конференции.
%
%  Скомпилировать файл в PDF можно двумя способами:
%  1. Использовать pdfLaTeX (pdflatex), (LaTeX+DVIPS не работает);
%  2. Использовать LuaLaTeX (XeLaTeX будет работать тоже).
%  При использовании LuaLaTeX потребуются TTF- или OTF-шрифты CMU
%  (Computer Modern Unicode). Шрифты устанавливаются либо пакетом
%  дистрибутива LaTeX cm-unicode
%              (https://www.ctan.org/tex-archive/fonts/cm-unicode),
%  либо загрузкой и установкой в операционной системе, адрес страницы:
%              http://canopus.iacp.dvo.ru/%7Epanov/cm-unicode/
%  Второй вариант не будет работать в XeLaTeX.
%
%  В MiKTeX (дистрибутив LaTeX для ОС Windows):
%  1. Пакет cm-unicode устанавливается вручную в программе MiKTeX Console.
%  2. Для верстки данного примера, а именно, картинки-заглушки необходимо,
%     также вручную, загрузить пакет pgf. Этот пакет используется популярным
%     пакетом tikz.
%  3. Тест показал, что остальные пакеты MiKTeX грузит автоматически (если
%     ему разрешено автоматически грузить пакеты). Режим автозагрузки
%     настраивается в разделе Settings в MiKTeX Console.
%
%
%  Самый простой способ сверстать статью - использовать pdfLaTeX, но
%  окончательный вариант верстки сборника будет собран в LuaLaTeX,
%  так как результат получится лучшего качества, благодаря пакету microtype и
%  использованию векторных шрифтов OTF вместо растровых pdfLaTeX.
%
%  В случае возникновения вопросов и проблем с версткой статьи,
%  пишите письма на электронную почту: eugeneai@irnok.net, Черкашин Евгений.
%
%  Новые варианты корректирующего стиля будут доступны на сайте:
%        https://github.com/eugeneai/nla-style
%        файл - nla.sty
%
%  Дальнейшие инструкции - в тексте данного шаблона. Он одновременно
%  является примером статьи.
%
%  Формат LaTeX2e!

\documentclass[12pt]{llncs}  % Необходимо использовать шрифт 12 пунктов.

% При использовании pdfLaTeX добавляется стандартный набор русификации babel.
% Если верстка производится в LuaLaTeX, то следующие три строки надо
% закомментировать, русификация будет произведена в корректирующем стиле автоматом.
\usepackage{iftex}

\ifPDFTeX
\usepackage[T2A]{fontenc}
\usepackage[utf8]{inputenc} % Кодировка utf-8, cp1251 и т.д.
\usepackage[english,russian]{babel}
\fi

% Для верстки в LuaLaTeX текст готовится строго в utf-8!

% В операционной системе Windows для редактирования в кодировке utf-8
% можно использовать программы notepad++ https://notepad-plus-plus.org/,
% techniccenter http://www.texniccenter.org/,
% SciTE (самая маленькая по объему программа) http://www.scintilla.org/SciTEDownload.html
% Подойдет также и встроенный в свежий дистрибутив MiKTeX редактор
% TeXworks.

% Добавляется корректирующий стилевой файл строго после babel, если он был включен.
% В параметре необходимо указать russian, что установит не совсем стандартные названия
% разделов текста, настроит переносы для русского языка как основного и т.п.

%\usepackage{todonotes} % Этот пакет нужен для верстки данного шаблона, его
                       % надо убрать из вашей статьи.

\usepackage[russian]{nla}

% Многие популярные пакеты (amsXXX, graphicx и т.д.) уже импортированы в корректирующий стиль.
% Если возникнут конфликты с вашими пакетами, попробуйте их отключить и сверстать
% текст как есть.
%
%


% Было б удобно при верстке сборника, чтобы названия рисунков разных авторов не пересекались.
% Чтоб минимизировать такое пересечение, рисунки можно поместить в отдельную подпапку с
% названием - фамилией автора или названием статьи.
%
% \graphicspath{{ivanov-petrov-pics/}} % Указание папки с изображениями в форматах png, pdf.
% или
% \graphicspath{{great-problem-solving-paper-pics/}}.

\begin{document}

% Текст оформляется в соответствии с классом article, используя дополнения
% AMS.
%
\fi

\title{Оптимальные траектории при недружественных наблюдателях}
% Первый автор
\author{В.~И.~Бердышев   % \inst ставит циферку над автором.
  \and  % разделяет авторов, в тексте выглядит как запятая.
% Второй автор
  В.~Б.~Костоусов 
  \and
  А.~А.~Попов 
} % обязательное поле

% Аффилиации пишутся в следующей форме, соединяя каждый институт при помощи \and.
\institute{Институт математики и механики им. Н.Н.Красовского УрО РАН, Екатеринбург, Россия \\
  \email{bvi@imm.uran.ru,  vkost@imm.uran.ru, aap@imm.uran.ru}
%  \and   % Разделяет институты и присваивает им номера по порядку.
%Институт (название в краткой форме), Город, Страна\\
%  \email{email@example.com}
% \and Другие авторы...
}

\maketitle

\begin{abstract}

В прикладных задачах, связанных с построением траекторий автономных движущихся объектов, иногда возникают требования максимальной удаленности траектории от нежелательных неподвижных объектов-наблюдателей.
Нередко такие объекты имеют средство наблюдения, позволяющее фиксировать движущиеся объекты в некотором конусе наблюдения, а за пределами этого конуса наблюдатель объекта не видит.
Подобные ситуации приводят к постановкам задач построения оптимальных траекторий в заданном коридоре $Y\subset X$.
В  докладе рассматривается задача формирования траектории в пространстве $X=\mathbb{R}^n, (n=2,3)$, максимизирующей минимум расстояния от наблюдателей в их поле зрения. Для точечного объекта строится оптимальный коридор~$Y^*\subset Y$~--- это такое связное множество, что любая непрерывная траектория, проходящая в этом коридоре и соединяющая заданные начальную и конечную точки, является оптимальной.  Аналогичная задача решается  на плоскости и для случая, когда движущийся объект~--- телесный, а именно представляет собой круг \cite{Berdyshev}.
 В дискретной постановке для практических расчетов в докладе предлагаются алгоритмы построения оптимального коридора и кратчайшей оптимальной траектории для телесного объекта на плоскости. Исходные непрерывные условия задачи, такие как границы ограничивающего коридора и конусов наблюдения, проектируются на дискретную регулярную сетку, и на ней строятся дискретная реализация оптимального коридора, его границ, а также находится с помощью алгоритма Дейкстры кратчайшая оптимальная траектория телесного объекта.

\keywords{планирование маршрута при наличии препятствий, оптимальная траектория, наблюдатели, кратчайший путь} % в конце списка точка не ставится
\end{abstract}

%\section{Основные результаты} % не обязательное поле
%
% Рисунки и таблицы оформляются по стандарту класса article. Например,

%\begin{figure}[htb]
%  \centering
  % Поддерживаются два формата:
  %\includegraphics[width=0.7\linewidth]{figure.pdf} % Растровый формат
  %\includegraphics[width=0.7\linewidth]{figure.png} % Векторный и растровый формат
  %
  % Векторные рисунки можно рисовать в редакторе Inkscape
  % https://inkscape.org/ru/download/
  % Основной формат этого редактора - SVG, поэтому рисунки необходимо экспортировать в
  % PDF или PNG (с разрешением - минимум 150 dpi, максимум - 300dpi).
%  \begin{center}
%    \missingfigure[figwidth=0.7\linewidth]{Уберите меня из статьи!}
%  \end{center}
%  \caption{Заголовок рисунка}\label{fig:example}
%\end{figure}

% Современные издательства требуют использовать кавычки-елочки << >>.

% В конце текста можно выразить благодарности, если этого не было
% сделано в ссылке с заголовка статьи, например,
%Работа выполнена при поддержке РФФИ (РНФ, другие фонды), проект \textnumero~00-00-00000.
%

% Список литературы оформляется подобно ГОСТ-2008.
% Примеры оформления находятся по этому адресу -
%     https://narfu.ru/agtu/www.agtu.ru/fad08f5ab5ca9486942a52596ba6582elit.html
%

\begin{thebibliography}{9} % или {99}, если ссылок больше десяти.
\bibitem{Berdyshev} Berdyshev V.~I., Kostousov V.B., Popov A.A. Optimal Trajectory in $R^2$ under Observation. Proc.~Steklov~Inst.~Mathematics. 2019. Vol.~304, Suppl.~1. Pp.~S31–S43.

\end{thebibliography}

% После библиографического списка в русскоязычных статьях необходимо оформить
% англоязычный заголовок.




%\end{document}

%%% Local Variables:
%%% mode: latex
%%% TeX-master: t
%%% End:
